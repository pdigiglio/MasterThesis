\subsection{Single-resonance Dalitz plots}
\label{subsec:single_resonance_dalitz_plots}

    % -- f0(980) ----------------------------------------------------------
    \begin{figure}
        \centering
        \subfloat[][\label{fig:f0_only_phase_fit}]{\begin{tikzpicture}
    \begin{axis}[phase_plot]
        \addplot+ [guess] table [x index={0}, y index={5}] {data/rho0/par_guess.txt};
        \addplot+ [fit] table [x index={0}, y index={5}, y error index={6}]
                  {data/rho0/par_fit.txt};
    \end{axis}
\end{tikzpicture}
}

        \subfloat[][\label{fig:f0_only_magnitude}]{\begin{tikzpicture}
    \begin{axis}[amplitude_plot]
        \addplot+ [guess] table [x index={0}, y index={1}] {data/f0_f0-1500/par_guess.txt};
        \addplot+ [fit] table [x index={0}, y index={1}, y error index={2}]
                  {data/f0_f0-1500/par_fit.txt};
    \end{axis}
\end{tikzpicture}
}
        \caption{Freed-wave fit of the S wave of a $\PDplus \to \Ppiplus\Ppiminus\Ppiplus$ decay through the \Pfnez{} resonance.~\Square} 
        \label{fig:f0_only}
    \end{figure}
    \begin{figure}
        \centering
        \subfloat[][Source data: decay in the S wave through the \Pfnez{} resonance.\label{fig:dalitz_f0_980_only_source}]{\begin{tikzpicture}
    \begin{axis} [dalitz_plot]

        \addplot3 [dalitz]
          gnuplot [raw gnuplot] { splot "data/f2/f2_mcmc.txt" using 1:2:3 };
    \end{axis}
\end{tikzpicture}
}

        \subfloat[][Result of the model-independent fit to the plot shown above.\label{fig:dalitz_f0_980_only_result}]{\begin{tikzpicture}
    \begin{axis} [dalitz_plot]
        \addplot3 [dalitz]
          gnuplot [raw gnuplot] { splot "data/f0/f0_fit_result_mcmc.txt" using 1:2:3 };
    \end{axis}
\end{tikzpicture}
}
        \caption{\ac{mc}-generated Dalitz plots of a $\PDplus \to \Ppiplus\Ppiminus\Ppiplus$ decay.~\Square}
        \label{fig:f0_only_dalitz}
    \end{figure}
    % ---------------------------------------------------------------------
    In this section, I present the test fits I performed on data sets where the $\PDplus\to\Ppiplus\Ppiminus\Ppiplus$ decay proceeds through one intermediate state only. 
    In this case, the dynamic-shape phase is an increasing function of the resonance mass and ranges from $0$ to $\pi$, so I restricted the phase-motion parameter to the asymmetric range $[0,\pi]$.


    Figure~\ref{fig:f0_only} shows the fit performed on the decay proceeding through the $\Pfnez{}$ resonance in the S wave.
    I optimized the mass binning in the neighborhood of the resonance peak.
    This is the easiest possible fit as in the S wave both the spin-dependent part of the decay amplitude and the Blatt-Weisskopf factor are constant over the phase space.
    Figure~\ref{fig:f0_only_dalitz} shows the comparison between the Dalitz plots of the source data and the data generated with the fitted parameters.


    \begin{figure}
        \centering

        \subfloat[][]{\begin{tikzpicture}
    \begin{axis}[phase_plot]
        \addplot+ [guess] table [x index={0}, y index={5}] {data/rho0/par_guess.txt};
        \addplot+ [fit] table [x index={0}, y index={5}, y error index={6}]
                  {data/rho0/par_fit.txt};
    \end{axis}
\end{tikzpicture}
}

        \subfloat[][]{\begin{tikzpicture}
    \begin{axis}[amplitude_plot]
        \addplot+ [guess] table [x index={0}, y index={1}] {data/f0_f0-1500/par_guess.txt};
        \addplot+ [fit] table [x index={0}, y index={1}, y error index={2}]
                  {data/f0_f0-1500/par_fit.txt};
    \end{axis}
\end{tikzpicture}
}

        \caption{Freed-wave fit of the P wave of a $\PDplus \to \Ppiplus\Ppiminus\Ppiplus$ decay through the \Prhozero{} resonance.~\Star}
        \label{fig:rho0_only}
    \end{figure}
    \begin{figure}
        \centering
        \subfloat[\acs{mc}-generated Dalitz plot of a P-wave decay through the \Prhozero{} resonance.]%
                 [Source data: P-wave decay through the \Prhozero{} resonance.\label{fig:rho0_only_dalitz_source}]%
                 {\begin{tikzpicture}
    \begin{axis} [dalitz_plot]

        \addplot3 [dalitz]
          gnuplot [raw gnuplot] { splot "data/f2/f2_mcmc.txt" using 1:2:3 };
    \end{axis}
\end{tikzpicture}
}

        \subfloat[\acs{mc}-generated Dalitz plot of a P-wave decay through the \Prhozero{} resonance.]%
                 [Result of the model-independent fit to the plot shown above.\label{fig:rho0_only_dalitz_fitted}]%
                 {\begin{tikzpicture}
    \begin{axis} [dalitz_plot]
        \addplot3 [dalitz]
          gnuplot [raw gnuplot] { splot "data/f0/f0_fit_result_mcmc.txt" using 1:2:3 };
    \end{axis}
\end{tikzpicture}
}

        \caption{\ac{mc}-generated Dalitz plots of a P-wave $\PDplus \to \Ppiplus\Ppiminus\Ppiplus$ decay through the \Prhozero{} resonance.~\Star}
        \label{fig:rho0_only_dalitz}

    \end{figure}
    Figure~\ref{fig:rho0_only} shows the fit performed on the decay proceeding through the $\Prhozero{}$ resonance in the P wave.
    In this case, neither the spin-dependent part of the amplitude nor the Blatt-Weisskopf factor is constant: they suppress the probability of the events close to the phase-space edges.
    The scarcity of events leads to the ambiguities observed close to the upper boundaries of the mass range in fit~\ref{fig:rho0_only}.
    Please note that, even though it is not so evident as in figure~\ref{fig:rho0_only}, also fit~\ref{fig:f0_only} shows this ambiguity.
    Figure~\ref{fig:rho0_only_dalitz} shows the comparison between the Dalitz plots of the source data and the data generated with the fitted parameters.
    Regardless of the fit ambiguity, Dalitz plot~\ref{fig:rho0_only_dalitz_fitted} correctly reproduces the right edge of Dalitz plot~\ref{fig:rho0_only_dalitz_source}.


    \begin{figure}
        \centering

        \subfloat[][\label{fig:f2_only_phase_fit}]{\begin{tikzpicture}
    \begin{axis}[phase_plot]
        \addplot+ [guess] table [x index={0}, y index={5}] {data/rho0/par_guess.txt};
        \addplot+ [fit] table [x index={0}, y index={5}, y error index={6}]
                  {data/rho0/par_fit.txt};
    \end{axis}
\end{tikzpicture}
}

        \subfloat[][\label{fig:f2_only_amplitude}]{\begin{tikzpicture}
    \begin{axis}[amplitude_plot]
        \addplot+ [guess] table [x index={0}, y index={1}] {data/f0_f0-1500/par_guess.txt};
        \addplot+ [fit] table [x index={0}, y index={1}, y error index={2}]
                  {data/f0_f0-1500/par_fit.txt};
    \end{axis}
\end{tikzpicture}
}

        \caption{Freed-wave fit of the D wave of a $\PDplus \to \Ppiplus\Ppiminus\Ppiplus$ decay through the \Pfii{} resonance.~\Square}
        \label{fig:f2_only}
    \end{figure}
    \begin{figure}
        \centering
        \subfloat[]%
                 [Source data: D-wave decay through the \Pfii{} resonance.\label{fig:f2_dalitz_source}]%
                 {\begin{tikzpicture}
    \begin{axis} [dalitz_plot]

        \addplot3 [dalitz]
          gnuplot [raw gnuplot] { splot "data/f2/f2_mcmc.txt" using 1:2:3 };
    \end{axis}
\end{tikzpicture}
}

        \subfloat[]%
                 [Result of the model-independent fit to the plot shown above.]%
                 {\begin{tikzpicture}
    \begin{axis} [dalitz_plot]
        \addplot3 [dalitz]
          gnuplot [raw gnuplot] { splot "data/f0/f0_fit_result_mcmc.txt" using 1:2:3 };
    \end{axis}
\end{tikzpicture}
}

        \caption{\ac{mc}-generated Dalitz plots of a $\PDplus \to \Ppiplus\Ppiminus\Ppiplus$ decay.~\Square}
        \label{fig:f2_only_dalitz}

    \end{figure}
    Figure~\ref{fig:f2_only} shows the fit performed on the decay proceeding through the $\Pfii{}$ resonance in the D wave.
    Also in this case, the spin-dependent part of the amplitude and the Blatt-Weisskopf factor lead to the ambiguities observed at both edges of the phase space.
    \begin{figure}
        \centering
        \begin{tikzpicture}
    \begin{axis}[
            phase_plot,
            ylabel={Dynamic-shape phase motion $[\si{deg}]$}
        ]
        \addplot+ [guess] table [x index={0}, y index={3}] {data/f2/par_guess.txt};
        \addplot+ [fit] table [x index={0}, y index={3}, y error index={4}]
                  {data/f2/par_fit.txt};

    \end{axis}
\end{tikzpicture}

        \caption{Fitted phase motion of the \Pfii{} with $\uD\phi_0 \equiv 0$.~\Square}
        \label{fig:fit_f2_phase_motion}
    \end{figure}
    The systematic phase displacement in fit~\ref{fig:f2_only_phase_fit} is due to the second bin's phase motion in figure~\ref{fig:fit_f2_phase_motion}, which shows the fitted phase motion of the \Pfii{}---the actual fit parameter.
    The first bin contains few events and is ambiguous, so setting $\uD \phi_0 \equiv 0$ does not fix the overall phase.
    \begin{figure}
        \centering
        \input{fig/f2/decay_amplitude_zoom.pgf}
        \caption{Detail of fit~\ref{fig:f2_only_amplitude}: I hide the last bin to highlight the fit close to the resonance peak.~\Square}
        \label{fig:f2_only_amplitude_zoom}
    \end{figure}
    Figure~\ref{fig:f2_only_amplitude_zoom} shows a detail of fit~\ref{fig:f2_only_amplitude} close to the resonance peak.
    Figure~\ref{fig:f2_only_dalitz} shows the comparison between the Dalitz plots of the source data and the data generated with the fitted parameters, which again correctly reproduce the source data.

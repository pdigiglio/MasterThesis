\begin{abstract}
    Partial-wave analysis is currently the standard analysis technique in the study of hadronic heavy-meson decays.
    In this context, to find an explicit form of the decay amplitude, most analyses exploit the isobar model, which assumes that the decay of the parent particle proceeds through subsequent two-body decays involving well-defined intermediate states.
    The success of the isobar model in providing a good description of the decay depends on the assumptions about these intermediate states.


    Due to the availability of increasingly large experimental data sets, the systematic uncertainty introduced by partial or incorrect knowledge of the intermediate states may dominate the statistical uncertainty.
    A possible extension of the current analysis techniques is the model-independent approach to partial-wave analysis:
    It exploits the increase of the experimental data samples to get rid of unjustified assumptions of the isobar model, as the properties of the isobars are extracted from the data itself.


    Because of the large experimental data sets and the large number of fit parameters, partial-wave analyses involve expensive calculations.
    This motivates the development of \pacs{yap}, a novel toolkit for partial-wave analysis.


    Here, after introducing the partial-wave-analysis formalism, I describe the main features of \pacs{yap} and present my \pacs{yap}-based implementation of a model-independent partial-wave-analysis fit utility.
    I also show the test fits I performed on several \acs{mc} data sets of $\PDplus\to\Ppiplus\Ppiminus\Ppiplus$ decays with increasing number of waves.
    The \acs{mc} data generated according to the fitted parameters correctly reproduce the fit source data.

\end{abstract}

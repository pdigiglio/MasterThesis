\section{The isobar decomposition}

    In quantum mechanics, the quantity that describes the distribution of some observed events is the event intensity, $\Intensity(\tau)$, being $\tau$ the coordinate of the event in its phase space.
    The intensity is defined as the squared magnitude of the event probability amplitude:
    \begin{equation}
        \Intensity(\tau) \coloneqq \abs{\A(\tau)}^2.
    \end{equation}
    In this section I will show how to write down the expression for the probability amplitude of a decay by means of the \ac{pwa} formalism, which---in this context---I will also call \emph{isobar formalism}\index{formalism!isobar}\index{isobar!formalism}.


    In the isobar formalism, a resonance\index{resonance} is a bound state of particles with well-defined quantum numbers.\footnote{Please note that a bound state of resonances is a resonance too.}
    Decaying particles can associate in resonances in more than one way, \ie~the decay can proceed through different topologies\index{decay topology}.
    The quantum numbers that characterize a resonance are its isospin, its spin, $J$, its parity, $P$, and its charge-conjugation parity, $C$.
    The notation usually adopted to indicate the resonance quantum numbers is $J^{PC}$.


    A set of bound states with the same quantum numbers is called \emph{isobar}\index{isobar}.
    Consequently, the decomposition of a decay amplitude in terms of isobars is called \emph{isobar model}\index{isobar!model} or formalism.
    In this formalism, the decay amplitude of an initial-state particle, $X$, into a set of final state particles---whose coordinate in the phase space is $\tau$---can be written as the following coherent sum of all the contributions from the isobars that are allowed by the conservation laws:
    \begin{equation}\label{eq:isobar_decomposition}
        \A_X(m_X,\tau) = \sum_{I\in\set{(J, P, C)}} c_I \Psi_I(m_X, \tau),
    \end{equation}
    being $c_I$ a complex parameter quantifying the relative weight of the $I$-th isobar amplitude, $\Psi_I$.
    I'd like to stress that in the equation~\eqref{eq:isobar_decomposition} I omitted the sum over the decay topologies, as I will do in the following discussion.
    Please also note that, in the context of the isobar decomposition, the partial waves correspond to the isobars.


%    In heavy-meson decays, a known initial-state particle decays to a known final state.
%    As the initial-state particle has a well-defined invariant mass, and fixes the $C$ and $P$ quantum numbers of the decay, the isobar decomposition can be simplified as follows:
%    \begin{equation}\label{eq:heavy_meson_isobar_decomposition}
%        \A(\tau) = \sum_{J} c_J \Psi_J(\tau),
%    \end{equation}
%    where I dropped the dependence on the initial-particle mass; and parity, and charge-conjugation parity quantum numbers.
%    The next sections will be on how to find a mathematical expression for the isobar amplitude $\Psi_J(\tau)$.
%    {\color{red} which angular momentum/spin quantum numbers are allowed?}

    \begin{figure}
        \centering
        \begin{tikzpicture}[
        particle/.style = {circle},
        P/.style        = {particle, ball color=black!20},
        a/.style        = {particle, ball color=blue!20},
        b/.style        = {particle, ball color=green!20},
        c/.style        = {particle, ball color=red!20},
    ]

    % Parent particle
    \node[P] (P) at (0,0) {$P$};

    \pgfmathsetmacro{\xDist}{6}
    \pgfmathsetmacro{\yDist}{3}

    \node[a] (a) at ($(P) + (\xDist, \yDist)$) {$a$};
    \node[b] (b) at ($(P) + (\xDist, 0)$)      {$b$};
    \node[c] (c) at ($(P) + (\xDist,-\yDist)$) {$c$};
    % Resonance
    \node[ ] (r) at ($(P) + (.5*\xDist, .5*\yDist)$) {$r$};

    % Segments
    \draw[->] (P) -- (r);
    \draw[->] (r) -- (a);
    \draw[->] (r) -- (b);
    \draw[->] (P) -- (c);
\end{tikzpicture}

        \caption{Decay of a spinless parent particle, $X$, to pseudo-scalar final-state particles, $a$, $b$, and $c$, through a resonance, $\xi$, in the $ab$ channel. Alternative topologies for the decay would have a resonance in the $bc$ channel or $ac$ channel.}
        \label{fig:isobar_three_body_decay}
    \end{figure}
    To reduce the complexity of the decay model, I will assume that the initial-state particle decays to the final-state particles via subsequent two-body decays, as schematically depicted in figure~\ref{fig:isobar_three_body_decay}.
    This assumption has been verified in non-leptonic three-body \PD{} and \PB{} particle decays~\cite[\S~13.2]{Bevan:2014iga}.


    Allowing only two-body sub-decays, the $\PDplus \to \Ppiplus\Ppiminus\Ppiplus$ decay will proceed as:
    \begin{equation}
        \PDplus\to\xi\Ppiplus,\qquad
        \xi\to\Ppiplus\Ppiminus,
    \end{equation}
    being $\xi$ and unknown resonance.
    The allowed resonances include, for example, the \Pfnez{} in the $0^{++}$ state, such that $\Pfnez\Ppiplus$ is in the $0^{-+}$ state;
    the \Prhozero{} in the $1^{--}$ state, such that $\Prhozero\Ppiplus$ is in the $1^{--}$ state;
    and the \Pfii{} in the $2^{++}$ state, such that $\Pfii\Ppiplus$ is in the $2^{++}$ state.


    I have taken the $J^{PC}$ values of the previous resonances from~\cite{chinese_phisics}.
    For the composite state of the resonance and the \emph{spectator}\index{spectator particle} (or \emph{bachelor}\index{bachelor particle}) particle, $\xi\Ppiplus$, I evaluate the state as follows. 
    The angular-momentum conservation reads
    \begin{equation}
        \vec{J}_{\PD} = \vec{L} + \vec{J}_{\xi} + \vec{J}_{\Ppi},
    \end{equation}
    where, as shown in the figure~\ref{fig:isobar_three_body_decay}, $\vec{L}$ is the operator for the orbital angular momentum between the bachelor $\Ppiplus$ and $\xi$, and I have dropped the particle superscripts to make the formula less cumbersome.
    As both the \Ppiplus{} and the \PDplus are spinless particles, $\vec{J}_{\PD} = \vec{J}_{\Ppi} = 0$; thus
    \begin{equation}
        \abs{J_{\xi} - L} \le 0 \le J_{\xi} + L.
    \end{equation}
    In particular, the left-hand inequality implies that---only for the specific decay I am considering---the resonance spin must be equal to the angular momentum quantum number of the system $\xi\Ppiplus$, namely $L$.


    The parity is a multiplicative quantum number, and couples with the angular momentum.
    So the parity of the $\xi\Ppiplus$ system is\marginpar{\color{red} must this be equal to $P_{\PD}$ in weak decays?}
    \begin{equation}
        P_{\xi\Ppi} = (-1)^L P_{\xi} P_{\Ppi}.
    \end{equation}


    Finally, the charge-conjugation parity is undefined for charged particles, so the $\xi\Ppiplus$ state will have the same value of $C$ as the resonance's. 


%    {\color{red}
%    All contributions from the possible associations of the final-state particles---\ie~which particles form the resonance and which is the \emph{spectator}, or \emph{bachelor}.
%    Each isobar amplitude $\Psi_J(\tau)$ can be factorized into a spin-dependent and a dynamic part.
%    }

    Having set the context of the isobar formalism, I will now discuss how to write down the explicit form of the contributions to the decay amplitude~\eqref{eq:isobar_decomposition}.
    Each isobar amplitude can be be decomposed as follows:\marginpar{\color{red}arguments!}
    \begin{equation}\label{eq:isobar_amplitude}
        \Psi_I(\tau) = \psi_J(\tau)\,\mathcal{F\!}_{J}(\tau)\,\Delta_I(s)\,\A_{d_1}\!(m_d; \tau)\,\A_{d_2}\!(m_d; \tau).
    \end{equation}
    Here $\psi_J$ is the spin-dependent part of the decay amplitude;
    $\mathcal{F\!}_J$ is the Blatt-Wei\ss{}kopf penetration factor;
    $\Delta_I$ is the dynamic shape of the isobar;
    and $\A_{d_1}$ and $\A_{d_2}$ are the amplitudes of the decays of $d_1$ and $d_2$, the daughter particles (or resonances) of the sub-decay.
    These amplitudes are to be evaluated recursively as in the equation~\eqref{eq:isobar_decomposition}, which is the constant function $\A_d = 1$ when $d$ is a final-state particle.
    Please note again that I am only allowing the decay to proceed via two-body sub-decays.


    In what follows, when not explicitly written, I will imply the sum over the Bose-Einstein symmetrizations of the indistinguishable particles, \ie~the \Ppiplus{} in the \PDplus{} decay under consideration.

    {\color{red}
    \begin{itemize}
        \item Each sub-decay amplitude can be decomposed as the~\eqref{eq:isobar_decomposition}; for example, in the $\PDplus\to\Ppiplus\Ppiminus\Ppiplus$\dots

            In this case, the first step is $\PDplus \to\xi\Ppiplus$, so
            \begin{equation}
                \Psi = \psi_L \Delta(s_{\PD})\A_\xi
                \quad
                \Delta(s_{\PD}) = \delta(s_{\PD} - m_{\PD}^2)
            \end{equation}
            Then\marginpar{\color{red} how much is $\psi_L$? Is $\psi_J$ cumulative of $J$ and $J$?}
            \begin{equation}
                \Psi = \psi_J \Delta(s)
            \end{equation}

    \end{itemize}
    }

    \paragraph{Spin-dependent amplitude}\marginpar{\color{red} + Blatt-Weisskopf}
    The spin-dependent amplitude, $\psi_J(\tau)$, describes the angular distribution of the decay and is fully specified by the spin quantum numbers of the initial-state particle, the isobar, and the final-state particles.
    There are, however, a number of formalisms to parametrize $\psi_J(\tau)$.
    

    In the Zemach formalism\index{formalism!Zemach}\index{Zemach formalism}~\cite[\S~V.1]{PhysRev.140.B97}, the form of the decay angular dependence in terms of the final-state particles three-momenta is
    \begin{equation}\label{eq:zemach_formalism}
        \psi_{J}^{(ab)c}(\vec{p}_a,\vec{p}_c) = \frac{J!}{(2J-1)!!}\,P_J(\vec{\hat{p}}_a \cdot \vec{\hat{p}}_c)\,\abs{\vec{p}_a}^J\abs{\vec{p}_c}^J,
    \end{equation}
    where the superscript $(ab)c$ means that $c$ is the spectator particle;
    $P_J$ is the $J$-th Legendre polynomial, which only depends on the \emph{helicity angle}\index{helicity angle}, namely the angle between $a$'s and $c$'s momenta ($\vec{\hat{p}}_a$ and $\vec{\hat{p}}_c$ are unit vectors);
    and the three-momenta $\vec{p}_a$ and $\vec{p}_c$ are evaluated in the rest frame of the $ab$ system.
    \begin{table}
        \centering
        \caption{Expressions of the spin-dependent part of the amplitude $\Psi_J$ in the Zemach formalism for the first three integer values of $J$.
                 The three-momenta $\vec{p}_a$ and $\vec{p}_c$ are to be evaluated in the rest frame of the subsystem $ab$.}
        \label{tab:zemach_formalism}
        
        \begin{tabular}{lc}
            \toprule
            $J$ &$\psi_J^{(ab)c}$\\
            \midrule
            $0$ &$1$ \\
            $1$ &$-2(\vec{p}_a\cdot\vec{p}_c)$ \\
            $2$ &$4(\vec{p}_a\cdot\vec{p}_c)^2 - 4(\abs{\vec{p}_a}\abs{\vec{p}_c})^2\!/3$\\
            \bottomrule
        \end{tabular}
    \end{table}
    The table~\ref{tab:zemach_formalism} shows the explicit expressions of $\psi_J$ for $J\in\set{0,1,2}$ in the Zemach formalism.


    Another possible angular formalism is the helicity formalism\index{helicity formalism}\index{formalism!helicity}~\cite{jacob1959404}, which I will not discuss here.
    Regardless of the particular formalism one chooses, the physical prediction, $\Intensity(\tau)$, must be the same.

    \paragraph{Dynamic shape}
    Unlike the spin-dependent part of the isobar amplitude, there is no way to derive the dynamic shape from first principles.
    In the past, physicists have mostly adopted a heuristic model-dependent description to obtain the form of the dynamic shape.
    This approach is described in the following section.
    A model-independent approach is described in section~\ref{sec:model_independent_isobar_decomposition}.


    \subsection{Model-dependent isobar decomposition}


    In the model-dependent approach to the isobar decomposition, the dynamic shape of the decay amplitude is expanded in terms of the resonances that populate the isobar:
    \begin{equation}\label{eq:isobar_mass_shape_expansion}
        \Delta_I(s) = \sum_{\xi\in I} \alpha_{\xi}\Delta_{\xi}(s),
    \end{equation}
    with $s$ the squared mass of the resonance, \ie~the squared four-momentum of the resonance daughter-particle system: referring to figure~\ref{fig:isobar_three_body_decay}, $s\coloneqq p_{ab}^2 = p_{\xi}^2$.
    Each coefficient $\alpha_{\xi}$ is the complex weight of the corresponding resonance $\xi$.
    Please note that the dynamic shape does not depend on the full phase-space coordinate but only on the invariant mass of the resonance it refers to.


    One of the most common parametrizations for the dynamic shape of a resonance is the relativistic Breit-Wigner distribution,\index{relativistic Breit-Wigner}\index{Breit-Wigner!relativistic}
    \begin{equation}\label{eq:rbw}
        \Delta_{\xi}^{\text{RBW}}(s) \coloneqq \frac{1}{m_{\xi}^2 - s - \iu \sqrt{s}\, \Gamma_\xi(s)},
    \end{equation}
    where the $m_{\xi}$ parameter is the nominal mass of the resonance.
    The mass-dependent width reads
    \begin{equation}
        \Gamma_\xi(s) \coloneqq \Gamma_{\xi} \, \mathcal{F}_{\!J_\xi}\!(q_{ab};R)\, \frac{m_{\xi}}{\sqrt{s}} \bigg(\frac{q_{ab}}{q_{\xi}}\bigg)^{2J_{\xi}+1},
    \end{equation}
    where the $\Gamma_{\xi}$ and $J_{\xi}$ parameters are the width and spin of the resonance;
    $\mathcal{F}$ is the Blatt-Wei\ss{}kopf form factor (see table~\ref{table:blatt_weisskopf});
    $q_{ab}$ is the break-up momentum at the measured mass $\sqrt{s}$;
    and $q_\xi$ is the break-up momentum at the nominal mass of $\xi$.
    \begin{figure}
        \centering
        \input{fig/relativistic_bws_comparison.pgf}
        \caption{Comparison of relativistic Breit-Wigner distributions for $J\in\set{0,1,2}$.}
        \label{fig:comparison_rbws}
    \end{figure}
    Figure~\ref{fig:comparison_rbws} shows a comparison of the relativistic Breit-Wigner distributions for $R = \SI{3}{\per\giga\electronvolt}$, and different values of $J$.


    If the resonance is narrow, namely if all relevant thresholds are far away, the term $\sqrt{s}\,\Gamma_\xi(s)$ in the denominator of equation~\eqref{eq:rbw} can be replaced with a constant quantity, $m_{\xi}\Gamma_{\xi}$,~\cite[\S~47.2.1]{chinese_phisics}.
    With this substitution, one gets the non-relativistic Breit-Wigner distribution,\index{Breit-Wigner}
    \begin{equation}\label{eq:bw}
        \Delta_{\xi}^{\text{BW}}(s) \coloneqq \frac{1}{m_{\xi}^2 - s - \iu m_{\xi} \Gamma_{\xi}},
    \end{equation}
    which is normalized such that $\Delta_{\xi}^{\text{BW}}(m_{\xi}^2) = \iu / m_\xi\Gamma_\xi$.


    Another possible dynamic shape is the pole-mass distribution,\marginpar{motivate the different distributions}
    \begin{equation}\label{eq:pole_mass}
        \Delta_{\xi}^{\text{PM}}(s) \coloneqq \frac{1}{m_{\xi}^2 - s},\quad
        \text{with }m_\xi = \Re (m_\xi) + \iu\Im(m_\xi),
    \end{equation}
    where the resonance-mass parameter, $m_\xi$, is a complex number.
    The distribution~\eqref{eq:pole_mass} has been proposed by \citeauthor{PhysRevD.71.054030} to parametrize the \Psigma{} resonance~\cite{PhysRevD.71.054030}.
    \begin{figure}
        \centering
        \begin{tikzpicture}
    \begin{axis}[
            ylabel={Magnitude $[\si{1/(\giga\electronvolt\per\c^2)^2}]$},
            xlabel={$s$ $[\si{\giga\electronvolt^2\per\c^4}]$},
            legend style={at={(1.1,.9)}}
        ]
        \pgfmathsetmacro{\mPi}{.139}
        \pgfmathsetmacro{\R}{3.}
        
        \pgfmathsetmacro{\m}{.98}
        \pgfmathsetmacro{\T}{.07}

        \addplot [domain={.279:1.73}, samples=100, smooth, densely dashed]
                 gnuplot {1./sqrt((\m * \m - x * x)**2 + (\m * \T)**2)};\addlegendentry{Breit-Wigner}

        \addplot [domain={.279:1.73}, smooth, densely dotted]
                 gnuplot [raw gnuplot] {
                     set samples 200;
                     q2_ab(x) = (.5 * x) ** 2 - \mPi * \mPi;
                     q2_xi    = q2_ab(\m * \m);
                     width(x) = \T * \m * sqrt(q2_ab(x)) / (sqrt(q2_xi * x));
                     BW(x) = 1. / (\m * \m - x * x - {0,1} * width(x * x));
                     plot [.279:1.73] abs(BW(x));
                 };\addlegendentry{Rel.~Breit-Wigner ($J=0$)}

%        \addplot [domain={.279:1.73}, smooth, densely dotted]
%                 gnuplot [raw gnuplot] {
%                     set samples 200;
%                     q2_ab(x) = (.5 * x) ** 2 - \mPi * \mPi;
%                     q2_xi    = q2_ab(\m * \m);
%                     F1(x) = sqrt(2 * x / (1 + x));
%                     width(x) = \T * \m * F1(q2_ab(x) * \R * \R) * (sqrt(q2_ab(x)) ** 3) / (sqrt((q2_xi ** 3) * x));
%                     BW(x) = 1. / (\m * \m - x * x - {0,1} * width(x * x));
%                     plot [.279:1.73] abs(BW(x));
%                 };\addlegendentry{Rel.~Breit-Wigner ($J=1$)}
%
%        \addplot [domain={.279:1.73}, smooth, densely dotted]
%                 gnuplot [raw gnuplot] {
%                     set samples 200;
%                     q2_ab(x) = (.5 * x) ** 2 - \mPi * \mPi;
%                     q2_xi    = q2_ab(\m * \m);
%                     F2(x)    = sqrt(13 * x * x / (x * x + 3 * x + 9));
%                     width(x) = \T * \m * F2(q2_ab(x) * \R * \R) * (sqrt(q2_ab(x)) ** 5) / (sqrt((q2_xi ** 5) * x));
%                     BW(x) = 1. / (\m * \m - x * x - {0,1} * width(x * x));
%                     plot [.279:1.73] abs(BW(x));
%                 };\addlegendentry{Rel.~Breit-Wigner ($J=2$)}

        \addplot [domain={.279:1.73}, samples=100, smooth, loosely dash dot]
                 gnuplot {1./sqrt((\m * \m - \T * \T - x * x)**2 + 4 * (\m * \T)**2)};\addlegendentry{Pole mass}
    \end{axis}
\end{tikzpicture}

        \caption{Comparison between the non-relativistic and relativistic Breit-Wigner, and the pole-mass dynamic shapes.
                 The first two are very close to each other.
        }
       % The parameters I used are $(\SI{.98}{\giga\electronvolt/\c^2}, \SI{.07}{\giga\electronvolt/\c^2})$, corresponding to $(m_\xi,\Gamma_\xi)$ for the Breit-Wigner shape; and $(\Re(m_\xi),\Im(m_\xi))$ for the pole-mass shape.}
        \label{fig:bw_pm_comparison}
    \end{figure}
    Figure~\ref{fig:bw_pm_comparison} shows a comparison between the non-relativistic Breit-Wigner and the pole-mass dynamic shapes.

    
    The \Pfnez{} resonance lies close to both the \Ppiplus{}\Ppiminus{} and the \PKplus{}\PKminus{} decay channels and is usually parametrized by a \citeauthor{FLATTE1976224}\index{Flatte@\citeauthor{FLATTE1976224}} distribution~\cite{FLATTE1976224}, which takes into account the opening of the second channel:
    \begin{equation}\label{eq:flatte}
        \Delta_{\xi}^{\text{F}}(s) \coloneqq \frac{1}{m_{\xi}^2 - s - \iu(g_1^2 \rho_{1\!}(s)  + g_2^2 \rho_2(s))},
    \end{equation}
    $g_1$ and $g_2$ being the coupling constants to the channels;
    and $\rho_{1\!}(s)$ and $\rho_2(s)$ the (complex) phase-space factors,
    \begin{equation}\label{eq:flatte_phase_space}
        \rho_i(s) = 2\frac{q_i(s)}{\sqrt{s}},
    \end{equation}
    with $q_i(s)$ the break-up momentum of the daughter particles of the $i$-th decay channel.
    Please note that the phase-space parametrization \citeauthor{FLATTE1976224} introduced in~\cite{FLATTE1976224} is not the one reported in equation~\eqref{eq:flatte_phase_space}.

    In the model-dependent \ac{pwa}, the isobar decay amplitude~\eqref{eq:isobar_decomposition} finally reads
    \begin{equation}\label{eq:redundant_model_dependent_isobar_decomposition}
        \Psi_I(\tau) =  \sum_{\xi\in I} \gamma_\xi \,\psi_{J_\xi\!}(\tau)\,\mathcal{F\!}_{J_\xi\!}(\tau)\,\Delta_{\xi}(s)\,\A_{d_1}\!(m_{d_1\!}; \tau)\,\A_{d_2}\!(m_{d_2\!}; \tau),\quad
        \text{with }
        \gamma_\xi\coloneqq c_I \alpha_{\xi}.
    \end{equation}
    I have absorbed the coefficients $c_I$ and $\alpha_\xi$ into one coefficient, $\gamma_{\xi}$, that I will call \emph{free amplitude}\index{free amplitude}.
    Please note that the free amplitudes in equation~\eqref{eq:redundant_model_dependent_isobar_decomposition} do not need the isobar label: since each resonance is in a well-defined isobar, a double index, $(I,\xi)$, would be redundant.
    Moreover, for the same reason, the common angular part in equation~\eqref{eq:redundant_model_dependent_isobar_decomposition} can be collected in front of the sum:
    \begin{equation}\label{eq:non_redundant_model_dependent_isobar_decomposition}
        \Psi_I(\tau) =  \psi_{J_I\!}(\tau)\,\mathcal{F\!}_{J_I\!}(\tau)\,\A_{d_1}\!(m_{d_1\!}; \tau)\,\A_{d_2}\!(m_{d_2\!}; \tau)\sum_{\xi\in I} \gamma_\xi \,\Delta_{\xi}(s).
    \end{equation}


    At this point, I would like to stress that in the expansion~\eqref{eq:non_redundant_model_dependent_isobar_decomposition} there are several sources of systematic uncertainties.
    The resonance content of each isobar is not known and highly affects the quality of the description the model provides.
    Moreover, some resonances---like the \Pfnez{}, the \Pfotsz, and the pole \Psigma{}---are not well established~\cite[\S~B]{PhysRevD.76.012001}.


    The dynamic shape of each resonance---along with its parameters---has to be experimentally determined: our knowledge about the dynamic shape comes from previous experiments performed under different conditions.
    Thus, when modelling a decay, it has to be silently assumed that neither the dynamic shape of the resonance, nor its parameters are affected by the particular interaction in the decay.


    Due to the availability of larger and larger experimental data sets, the limitations of the model-dependent isobar formalism are becoming critical.
    In the next section I will present how such limitations can be circumvented by means of the model-independent isobar formalism.



    \subsection{Model-independent isobar decomposition}
\label{sec:model_independent_isobar_decomposition}

    The model-independent approach to the isobar decomposition aims to reduce the analysis' dependence on possibly incomplete or incorrect fit models, as it does not need any assumption on the resonant content of the isobars.
    In this approach, the dynamic shapes of the isobars are directly extracted from the available experimental events, provided the data set is large enough.
    

    A characteristic function\index{characteristic function} on a real interval, $M \coloneqq [m_{\text{low}}, m_{\text{up}})$, is defined as
    \begin{equation}
        \Characteristic_M(m) \coloneqq 
        \begin{cases}
            1 &\text{if }m \in M, \\
            0 &\text{otherwise}.
        \end{cases}
    \end{equation}
    Let $\mathcal{P}(M) \coloneqq \set{m_0, m_1,\dots, m_{N_\text{bins}}}$ be a partition of $M$---\ie~a set of points such that $m_{i} < m_{i+1}$; and whose extremes coincide with those of $M$, namely $(m_0, m_{N_\text{bins}}) = (m_{\text{low}}, m_{\text{up}})$.
    \begin{figure}
        \centering
        \begin{tikzpicture}
    \pgfmathtruncatemacro{\a}{3}
    \pgfmathtruncatemacro{\b}{1}
    \pgfmathtruncatemacro{\c}{2}

    \begin{scope}[color=black!50]
        \coordinate (o) at (-1, 0);
        \draw (o) -- +(\a + 2, 0) coordinate (a);
        \draw [densely dashed] (a) -- +(\b, 0) coordinate (b);
        \draw [->] (b) -- +(\c, 0) coordinate (c);
    \end{scope}

    \begin{scope}
        % vertical label shift
        \pgfmathsetmacro{\s}{-.5}

        \foreach \x in {0,1,...,\a} {
            \node [draw, inner sep=1pt, circle, fill=black] (x\x) at (\x, 0) {};
            \node (sx\x) at ($(x\x) + (0, \s)$)  {$m_\x$};
        }

        \node [draw, inner sep=1pt, circle, fill=black] (last) at (\a + \b + 2, 0) {};
        \node (slast) at ($(last) + (0, \s)$) {$m_{N_\text{bins}}$};

    \end{scope}

    \node () at (x0) [label={above:$m_a + m_b$}] {};
    \node () at (last) [label={above:$m_X - m_c$}] {};


\end{tikzpicture}

        \caption[Partition of the invariant-mass range into $N_\text{bins}$ right-open bins.]%
        {Partition of the invariant-mass range, $M \coloneqq [m_a+m_b,m_X - m_c)$, into $N_\text{bins}$ right-open bins, $B_i = [m_{i-1}, m_i)$.}
        \label{fig:invariant-mass-partition}
    \end{figure}
    Figure~\ref{fig:invariant-mass-partition} shows the partitioning of the interval $[m_a+m_b,m_X-m_c)$, the allowed invariant-mass range for the final-state $ab$ system.\footnote{The allowed mass range of the $ab$ system also includes the value $m_X - m_c$; I will anyway ignore this value as it is a null-measure interval, so leaving it out will not affect the results of the analysis.}
    The partition $\mathcal{P}(M)$ defines $N_\text{bins}$ bins on $M$, $\set{B_i}$, where $B_i$ is the right-open interval $[m_{i-1},m_i)$ for $i\in\set{1,\dots, N_\text{bins}}$.
    A step function\index{step function} on the interval $M$, given the partition $\mathcal{P}(M)$ is a finite linear combination of characteristic functions on the bins:
    \begin{equation}\label{eq:step_dynamic_shape}
        \Delta_{\mathcal{P}(M)}^{\text{SF}}(m) \coloneqq \sum_{i = 1}^{N_\text{bins}} \alpha_i\, \Characteristic_{B_i}\!(m),
    \end{equation}
    $\alpha_i$ being a complex coefficient.


    In the model-independent isobar decomposition, equation~\eqref{eq:step_dynamic_shape} is the approximate dynamic shape of the isobar, and its coefficients are determined from the analysis of the observed events.
    \begin{figure}
        \centering
        \begin{tikzpicture}
    \begin{axis}[
        amplitude_plot,
        xlabel={$s$ $[\si{\giga\electronvolt^2\per\c^4}]$},
    ]
        \pgfmathsetmacro{\m}{.98}
        \pgfmathsetmacro{\T}{.07}
        \addplot+ [guess, domain={.279:1.73}, samples=100, smooth] gnuplot {1./sqrt((\m * \m - x * x)**2 + (\m * \T)**2)};
        \addplot+ [fit, mark=none] table {data/binned_breit-wigner.txt};
    \end{axis}
\end{tikzpicture}

        \caption{Step-function approximation of the magnitude of the non-relativistic Breit-Wigner dynamic shape, equation~\eqref{eq:bw}.}
        \label{fig:step_function_approximation}
    \end{figure}
    Figure~\ref{fig:step_function_approximation} shows the basic idea behind the model-independent isobar decomposition: the piecewise function in equation~\eqref{eq:step_dynamic_shape} will approximate the isobar mass shape.


    With the piecewise dynamic shape, the isobar amplitude reads
    \begin{equation}\label{eq:freed_wave_amplitude}
        \Psi_I(\tau) = \psi_{J_I\!}(\tau)\,\mathcal{F\!}_{J_I\!}(\tau)\,\A_{d_1}\!(m_{d_1\!}; \tau)\,\A_{d_2}\!(m_{d_2\!}; \tau) \sum_{i=1}^{N_\text{bins}} \gamma_{(I,i)}\, \Characteristic_{B_i}\!(s),
        \quad
        \text{with }
        \gamma_{(I,i)} \coloneqq c_I \alpha_i. 
    \end{equation}
    The free amplitudes, $\gamma_{(I,i)}$, now need to have an isobar label too.
    I will call a wave whose decay amplitude is written in the form of equation~\eqref{eq:freed_wave_amplitude} \emph{freed wave}\index{freed wave}.


    At this point it is worth noting that, when fitting the decay model to the data, the piecewise description of the dynamic shape introduces $N_\text{bins}$ complex \acp{dof}---for a total of $2N_{\text{bins}}$ real \acp{dof}.
    This drastically increases the computational complexity of the fit.


    To contain the number of fit \acp{dof}, it is possible to mix the model-dependent and model-independent \ac{pwa} approaches: the isobars with well-known resonances may be described in a fixed way, while the other isobars may be freed.
    The \focus{} collaboration performed a \ac{pwa} of the $\PDplus \to \PKminus\Ppiplus\Ppiplus$ decay employing a fixed description of the P- and D-wave components of the $\PKminus\Ppiplus$ decay amplitude (described as a sum of Breit-Wigner distributions), and a model-independent description of the S-wave component~\cite{Link200914}.


    In chapter~\ref{chap:model_independent_pwa}, I exploit the information about the resonant content of the \ac{mc} data to make the bins finer in the surroundings of the resonance peaks.
    A non-uniform optimized partitioning of the mass range speeds up the fit convergence.
    Unfortunately, the information about a wave's resonant content is not always available.


        \subsubsection{Zero modes}
        Introduction of the isobar-freed fit leads to linear dipendencies among the waves in the fit model.
This can be seen by means of the integral matrix (now $w$ and $v$ are wave indices)
\begin{equation}
    I_{wv} \coloneqq \int_\Omega \psi_w^*(\tau)\,\psi_v(\tau)\ud\tau
\end{equation}
which represents the integrated overlap of the waves $w$ and $v$ over the phase space.
Since $I$ is hermitian, its eigenvalues are real.
So
\begin{equation}
    I^{\text{Tot}} \coloneqq \int_\Omega \abs{\A(\tau)}^2\!\ud \tau = \sum_{v,w \in \Set{\text{waves}}} T_w^* I_{wv} T_v.
\end{equation}


    {\color{red}
    (Quote Fabian's work as unpublished?)
    }


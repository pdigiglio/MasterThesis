\chapter{\texorpdfstring{\acs{yap} --- \acl{yap}}{YAP - Yet Another PWA Toolkit}}
\label{chap:yap}

    \pac{yap} is a free (licensed under the \acs{gpl3}) and open-source \cpp[11] software for calculating partial-wave amplitudes for $n$-body particle decays.
    It is currently maintained by Dr.~Daniel Greenwald and Johannes Rauch at the \textsl{Techniche Universit\"at M\"unchen}.
    The project is hosted on \textsf{GitHub} at \url{https://github.com/yap/YAP}.\marginpar{\centering\qrcode[height=2cm]{https://github.com/yap/YAP}\captionof*{figure}{\pac{yap}'s {\footnotesize URL}.}}


    The source code is designed to be human-readable and to reflect the mathematical formalism describing the particle decays.
    All the physics used is cited and documented in the code with the \package{doxygen} package; and verified through an extensive suite of tests.\footnote{As of writing, the tests cover \SI{90}{\percent} of the code.}


    One of the main goals of \pac{yap} is modularity.
    In fact, it is distributed in the form of a shared library to be linked against the user's application.
    Moreover, \pac{yap} does not depend on any library beyond the \cpp{} \ac{stl}, which makes it easily compilable on most operating systems and portable on many architectures.
    It is worth stressing that \pac{yap} is an amplitude calculator only and it does not come with any \ac{mc} sampler or \ac{gui}.
    Thus it has to be coupled with third-party software that provides such functionalities---if needed.


    A similar project to \pac{yap}, which also implements the \ac{pwa} formalism, is {\footnotesize{\package{ROOTPWA}}}.
    It is based on \ROOT{}, on \package{boost} and requires other external libraries for the user to get started.
    {\footnotesize{\package{ROOTPWA}}} is also developed at the \textsl{Techniche Universit\"at M\"unchen} and is hosted at \url{https://github.com/ROOTPWA-Maintainers/ROOTPWA}.


    In this chapter, I will explain the basic concepts behind \pac{yap}, its goals and provide some usage examples.


    \section{Computational challenges}

    The motivation for developing \pac{yap} comes from challenges that need to be addressed on both the user interface and the computational level.
    I will show examples of the user interface later on; here I will present some of the optimizations \pac{yap} features.
    The performance of the amplitude calculator is critical because the \ac{pwa} requires expensive calculations especially for the following reasons.


    The large experimental data sets, as already mentioned, require the decay models to increase in complexity in order to provide a satisfactory description of the process under exam.
    Moreover, the large number of free parameters during the fit increases the chances for local maxima of the likelihood function.
    These issues affect both the memory footprint and the \acs{cpu} usage of the amplitude calculator.


    To reduce the number of expensive amplitude calculations, \pac{yap} features a smart caching system.
    The wave amplitudes are evaluated once on all the input data and their values are cached.
    Thanks to the \lstinline!Parameter! class, the amplitudes are evaluated again only if their parameters have changed during the run time.
    I will show a short example on how the caching system works for a user-defined dynamic-shape class in the code~\ref{lst:mass_shape}, page~\pageref{lst:mass_shape}.


    Both the input data and the cached amplitude values are stored in instances of the \lstinline!DataPoint!\index{DataPoint@\texttt{DataPoint}} class.
    The data is aggregated into a \lstinline!DataPartition!\index{DataPartition@\texttt{DataPartition}}, a \ac{stl}-like container which can be traversed by means of its own iterator, the \lstinline!DataIterator!\index{DataIterator@\texttt{DataIterator}}.
    The idea behind the partitioning is to define disjoint regions in the input data that allow for thread-safe parallel calculations of the amplitudes without race hazards\index{race hazard}---\ie~when threads try to write to the same piece of memory at the same time; or when one thread reads from a piece of memory that is being updated by another, thus reading an intermediate undefined value.
    So far, \pac{yap} supports \acs{cpu} parallelization via the platform-independent thread-support library introduced in the \cpp[11]{} standard.


    The \lstinline!DataPoint! class is a wrapper around a \lstinline!std::vector! of \lstinline!std::vector!'s: the build system of \pac{yap} offers a switch to decide at compile time whether the underlying type of the \lstinline!DataPoint! should be a \lstinline!float! or a \lstinline!double!.\footnote{The default data type, currently \lstinline!double!, can be changed via the \lstinline[language=bash]|-DYAP_DATA_POINT_TYPE=float| option to the \lstinline!cmake! command.}
    This allows to reduce the memory footprint of the data by approximately \SI{50}{\percent} at the cost of losing some precision in the calculations.
    If the memory footprint is an issue in the analysis, or the expected uncertainty in the result largely exceeds the \lstinline!float! numerical resolution, this is a useful optimization to adopt.


    To further reduce the memory footprint, another optimization \pac{yap} features is to share among the waves all the elements they have in common.
    As an example, let's refer to the equation~\eqref{eq:non_redundant_model_dependent_isobar_decomposition}, which gives the decay amplitude of an isobar.
    The cached values of the angular part and the Blatt-Wei\ss{}kopf factor are shared among all the resonances in the isobar---and, possibly, among all the isobars---with the same value of $J$.
    Moreover, for $J = 1$, the angular part is just a constant and it is not cached, as its value can be absorbed in the free-amplitude coefficient.


    \section{Structure and user interface}

    The source code of \pac{yap} aims to be human readable and to highlight the physics behind it.
    To implement a modern and efficient design, and provide a clear \ac{api}, we mostly adopted the best practices described by~\citeauthor{stl_meyers} in~\cite{stl_meyers,effective_cpp_meyers,effective_modern_cpp_meyers}.
    Besides the documentation of \pac{yap} itself, these are surely good references if one wants to have a look into the code.


    %\pac{yap} is targeted to physicists, to witch it offers a comprehensive and intuitive framework.
    To the final user, \pac{yap} provides a \ac{stl}-like \ac{api} which mimics the \ac{pwa} analysis formalism for particle decays.
    To avoid name collisions, we enclosed \pac{yap}'s classes and free functions in the \lstinline!yap! namespace, which I will always drop in the following examples for simplicity.


    As stated earlier, \pac{yap}'s code mimics the mathematical description of particle decays.
    As an example, each term of the isobar amplitude~\eqref{eq:isobar_amplitude}---spin amplitude, Blatt-Wei\ss{}kopf factor, dynamic amplitude---is modeled in the code by a corresponding class which reflects its properties, in observance of the \emph{single-responsibility principle}.
    The total decay amplitude is then obtained by multiplying these terms; the free amplitudes; and, recursively, the amplitudes of all the sub-decays.
    The symmetrization of indistinguishable particles is managed by the \lstinline!ParticleCombination!\index{ParticleCombination@\texttt{ParticleCombination}} class.


    The abstract \lstinline!Particle!\index{Particle@\texttt{Particle}} class models a particle and contains the information about its name and quantum numbers---namely the electric charge, spin, parity, charge-conjugation parity, isospin, and G parity.
    This class is the base for the final-state and the decaying particles.
    Note that \pac{yap} makes no distinction between decaying particles and resonances.
    The \lstinline!DecayingParticle!\index{DecayingParticle@\texttt{DecayingParticle}} class knows about the particle's dynamic-shape distribution, penetration factor, and decay channels.
    The \lstinline!DecayChannel!\index{DecayChannel@\texttt{DecayChannel}} class, in turn, stores the daughter decaying or final-state particles of the particle it refers to, and its cached spin amplitude.
    The \lstinline!FinalStateParticle!\index{FinalStateParticle@\texttt{FinalStateParticle}} class is a \lstinline!Particle! with a well-defined mass (this is consistent with the assertion that the dynamic shape of a final-state particle is a Dirac delta centered on its nominal mass).
    The information about the particles and resonances---quantum numbers, names, masses and widths---has to be read from an external database.
    So far, \pac{yap} only implements an interface to the \texttt{evt.pdl} table~\cite[p.~13]{evtgen_manual} of the \package{EvtGen} package.


    The decay channels and the data points are handled by the \lstinline!Model!\index{Model@\texttt{Model}} class.
    To keep the data safe from accidental corruption, the only way to access the cache in the data points is through the \lstinline!DataAccessor! class.
    \begin{sidewaysfigure}
        \centering
        \tikzumlset{
    fill class=blue!00,%
    font=\ttfamily%
}
\begin{tikzpicture}
    % Classes
    \umlsimpleclass{DataAccessor}
    \umlsimpleclass[x=-4.5, y=-4]{RecalculableDataAccessor}
    \umlsimpleclass[x=-7.5, y=-8]{BlattWeisskopf}
    \umlsimpleclass[x=-2.1, y=-8]{MassShape}

    \umlsimpleclass[x=4.5, y=-4]{StaticDataAccessor}
    \umlsimpleclass[x=2.1, y=-8]{HelicitySpinAmplitude}
    \umlsimpleclass[x=7.5, y=-8]{ZemachSpinAmplitude}

    % Inheritances
    \umlVHVinherit{RecalculableDataAccessor}{DataAccessor}
    \umlVHVinherit{StaticDataAccessor}{DataAccessor}

    \umlVHVinherit{BlattWeisskopf}{RecalculableDataAccessor}
    \umlVHVinherit{MassShape}{RecalculableDataAccessor}

    \umlVHVinherit{HelicitySpinAmplitude}{StaticDataAccessor}
    \umlVHVinherit{ZemachSpinAmplitude}{StaticDataAccessor}
\end{tikzpicture}

        \caption{A simplified \ac{uml} diagram that shows some of the daughter classes of \lstinline!DataAccessor!.}
        \label{fig:data_accessor_hierarchy}
    \end{sidewaysfigure}
    The figure~\ref{fig:data_accessor_hierarchy} shows a partial \ac{uml} diagram of the \lstinline!DataAccessor! class.
    This diagram shows a further feature of \pac{yap}: the amplitude parts that do not depend on any of the model parameters inherit from \lstinline!StaticDataAccessor!\index{StaticDataAccessor@\texttt{StaticDataAccessor}} and are always evaluated only once on the input data.


    Please note that the \lstinline!Model! class also provides the spin formalism.
    As of now, the Zemach and the helicity spin formalisms are implemented.


    \section{Examples}

        In this section, I will show a couple of code excerpts on how to make a decay model that can be used to generate \ac{mc} data, and how to extend the available dynamic shapes \pac{yap} already provides.
        I chose the following examples both to illustrate the user interface and because they are relevant for the model-independent fit utility I implemented.

        \subsection{\texorpdfstring{Generating a \acs{mc} data set}{Generating a MC data set}}
        

        \begin{table}
            \centering
            \caption{Resonance content of the Dalitz plot in the figure~\ref{fig:partial_cleo_decay}. The properties of the resonances are taken from the analysis of the $\PDplus\to\Ppiplus\Ppiminus\Ppiplus$ decay by the \cleo{} collaboration~\cite{PhysRevD.76.012001}.}
            \label{tab:cleo_model}
            \begin{tabular}{llcc}
                \toprule
                Resonance &Dynamic shape &Free amplitude &Phase [\si{\degree}]\\
                \midrule
                \Prhozero{} &Rel.~Breit-Wigner &\num{1}   &\num{0}   \\
                \Pfnez{}    &Flatté            &\num{1.4} &\num{12}  \\
                \Pfofzz{}   &Rel.~Breit-Wigner &\num{1.1} &\num{-44} \\
                \Psigma{}   &Pole mass         &\num{3.7} &\num{-3}  \\
                \bottomrule
            \end{tabular}
        \end{table}
        In the following code snippet, I will illustrate an usage example of the \ac{api} to create a decay model with the resonant content listed in the table~\ref{tab:cleo_model}.
        This model may be used to perform a fit to experimental data or to generate \ac{mc} data.

        % -- Model-creation example ---------------------
        \lstinputlisting[caption={Usage example of the \ac{api} of \pac{yap}. The following code implements a decay model with the resonant content reported in the table~\ref{tab:cleo_model}.\label{lst:partial_cleo_decay}}]{source/d3pi.cxx}
        % -----------------------------------------------
        \begin{figure}
            \centering
            \begin{tikzpicture}
    \begin{axis} [dalitz_plot]
        \addplot3 [dalitz]
          gnuplot [raw gnuplot] { splot "data/cleo/rho0_f0_f0_1500_f0_500_mcmc.txt" using 1:2:3 };
    \end{axis}
\end{tikzpicture}

            \caption{\ac{mc}-generated Dalitz plot of a $\PDplus \to \Ppiplus\Ppiminus\Ppiplus$ decay with the resonance content summarized in the table~\ref{tab:cleo_model}.}
            \label{fig:partial_cleo_decay}
        \end{figure}
        The initial-state particle must be set after the whole decay structure has been defined with the \lstinline!addInitialState! member function (line~\ref{line:addInitialState}).
        The \lstinline!Model! needs to be locked (line~\ref{line:lock}) before the free amplitudes can be accessed and set.
        Please note that reading the \texttt{evt.pdl} table (line~\ref{line:read_pdl_table}) is not strictly necessary.


        The \pac{yap} repository contains examples on how to couple it with a sampler like \pac{bat}~\cite{Caldwell20092197}.
        The Dalitz plot in the figure~\ref{fig:partial_cleo_decay} shows the \ac{mc} data generated with the model in the listing~\ref{lst:partial_cleo_decay}.


        \subsection{Adding a new dynamic shape}

        As of writing, \pac{yap} implements the relativistic Breit-Wigner~\eqref{eq:rbw}, the non-relativistic Breit-Wigner~\eqref{eq:bw}, the pole-mass~\eqref{eq:pole_mass}, and the Flatté~\eqref{eq:flatte} dynamic shapes.
        The user may need to define a new dynamic shape a new dynamic shape beyond the most common ones already provided to perform an analysis.


    \begin{figure}
        \centering
        \tikzumlset{
    fill class=blue!00,%
    font=\ttfamily%
}
\begin{tikzpicture}
    % Classes
    \umlsimpleclass{MassShape}
    \umlclass[x=-4, y=-4]{MassShapeWithNominalMass}%
                         {Mass\_ \ : NonnegativeRealParameter}%
                         {mass() : NonnegativeRealParameter}
    \umlsimpleclass[x=+4, y=-4]{PoleMass}
    \umlsimpleclass[x=-6, y=-8]{BreitWigner}
    \umlsimpleclass[x=-6, y=-10]{RelativisticBreitWigner}
    \umlsimpleclass[x=-2, y=-8]{Flatte}

    % Inheritances
    \umlVHVinherit{MassShapeWithNominalMass}{MassShape}
    \umlVHVinherit{PoleMass}{MassShape}

    \umlVHVinherit{BreitWigner}{MassShapeWithNominalMass}
    \umlVHVinherit{Flatte}{MassShapeWithNominalMass}

    \umlVHVinherit{RelativisticBreitWigner}{BreitWigner}
\end{tikzpicture}

        \caption{\Ac{uml} diagram for the \lstinline!MassShape! class hierarchy. I give more details for the \lstinline!MassShapeWithNominalMass! class as I would like to point out its dependence on the smart \lstinline!Parameter! class.}
        \label{fig:mass_shape_hierarchy}
    \end{figure}
    As an example, I will show an excerpt of the declaration and definition of the \lstinline!MassBin! dynamic shape I implemented to be used in the model independent fit utility described in the chapter~\ref{chap:model_independent_pwa}.
    Given the \ac{uml} diagram shown in the figure~\ref{fig:mass_shape_hierarchy}, the most meaningful design choice is to inherit the \lstinline!MassBin! class directly from \lstinline!MassShape!.
    
    \lstinputlisting[caption={Excerpt of the definition and the declaration of the \lstinline!MassBin! class.}\label{lst:mass_shape}]{source/mass_shape.h}
    In the code~\ref{lst:mass_shape}, it is important to note that the class parameters need to be added to the \lstinline!Model! via the inherited \lstinline!addParameter! member function (line~\ref{line:addParameter}), otherwise the model intensity will be undefined.
    As I stated before, the \lstinline!calculate! member function will only be triggered if (at least) one of the dynamic-shape parameters has changed.


    \section{Future enhancements}

    At the current state, \pac{yap} is already usable but is not yet complete.
    Some of the future milestones of the project are:
    \begin{itemize}
        \item
            Extend the number of dynamic-shape classes, introducing distributions like the Bowler, the Gounora-Sakurai, and so on;

        \item
            Add an user interface to coherently or incoherently sum an arbitrary function of the phase space to the decay amplitude: this would allow the user to introduce a parametrization for the non-resonant and background components;

        \item
            Add an user interface to multiply an arbitrary function of the phase space to the decay amplitude: this would allow the user to introduce its detector acceptance parametrization;

        \item
            Implement the $K$-matrix formalism.

    \end{itemize}

    From the performance point of view, in the future \pac{yap} may exploit the huge parallel capabilities offered by \acsp{gpu}, by means of the \package{OpenCL} or Nvidia's \pacs{cuda} frameworks.
    A paper by \citeauthor{gpu_pwa_berger} reports that offloading the amplitude calculation to co-processor \acsp{gpu} can speed up the runtime of \ac{pwa} software by more than two orders of magnitude~\cite{gpu_pwa_berger}.


    Another yet missing feature is a piecewise dynamic shape for model-independent \acp{pwa}.
    In the next chapter, I present my fit utility that builds on top of \pac{yap} and implements the model-independent \ac{pwa} formalism for the specific $\PDplus\to\Ppiplus\Ppiminus\Ppiplus$ decay.
    In the future, a generalization of the formalism I implemented in my fit utility will be included in \pac{yap}.

\section{The isobar decomposition}

    In the isobar formalism, a resonance\index{resonance} is a bound state of particles with well-defined quantum numbers.
    The quantum numbers that characterize a resonance are its isospin, its spin, $J$, its parity, $P$, and its charge-conjugation parity, $C$.
    The notation usually adopted to indicate the resonance quantum numbers is $J^{PC}$.


    A set of bound states with the same quantum numbers is called \emph{isobar}\index{isobar}.
    Consequently, the decomposition of a decay amplitude in terms of isobars is called \emph{isobar model}.

    In this formalism, the decay amplitude of an initial-state particle of invariant mass $M$ into a final state whose coordinate in the phase space is $\tau$ can be written as a coherent sum of isobar amplitudes,
    \begin{equation}\label{eq:isobar_decomposition}
        \A(M,\tau) = \sum_{I\in\set{(J, P, C)}} c_I \Psi_I(M, \tau),
    \end{equation}
    being $c_I$ a complex parameter quantifying the magnitude and (relative) phase of the isobar amplitude labeled by $I$.
    In the context of the isobar decomposition, the partial waves correspond to the isobars.


    In heavy-meson decays, a known initial-state particle decays to a known final state.
    As the initial-state particle has a well-defined invariant mass, and fixes the $C$ and $P$ quantum numbers of the decay, the isobar decomposition can be simplified as follows:
    \begin{equation}\label{eq:heavy_meson_isobar_decomposition}
        \A(\tau) = \sum_{J} c_J \Psi_J(\tau),
    \end{equation}
    where I dropped the dependence on the initial-particle mass; and parity, and charge-conjugation parity quantum numbers.
    The next sections will be on how to find a mathematical expression for the isobar amplitude $\Psi_J(\tau)$.
    {\color{red} which angular momentum/spin quantum numbers are allowed?}





    \subsection{Model-dependent isobar decomposition}

        \begin{figure}
            \centering
            \begin{tikzpicture}[
        particle/.style = {circle},
        P/.style        = {particle, ball color=black!20},
        a/.style        = {particle, ball color=blue!20},
        b/.style        = {particle, ball color=green!20},
        c/.style        = {particle, ball color=red!20},
    ]

    % Parent particle
    \node[P] (P) at (0,0) {$X$};
    \draw [<->, color=black!50] ($(P) + (-25:1.5)$) arc (-25:25:1.5) node [midway, label={right:$L$}] {};
    %\draw (330:1) arc (30:1);

    \pgfmathsetmacro{\xDist}{6}
    \pgfmathsetmacro{\yDist}{3}

    \node[a] (a) at ($(P) + (\xDist, \yDist)$) {$a$};
    \node[b] (b) at ($(P) + (\xDist, 0)$)      {$b$};
    \node[c] (c) at ($(P) + (\xDist,-\yDist)$) {$c$};
    % Resonance
    \node[ ] (r) at ($(P) + (.5*\xDist, .5*\yDist)$) {$\xi$};
    \draw [<->, color=black!50] ($(r) + (-25:1.5)$) arc (-25:25:1.5) node [midway, label={right:$J_{\xi}$}] {};

    % Segments
    \draw[->] (P) -- (r);
    \draw[->] (r) -- (a);
    \draw[->] (r) -- (b);
    \draw[->] (P) -- (c);
\end{tikzpicture}

            \caption{$P\to abc$ decay through a resonance $\xi$ in the $ab$ channel.}
            \label{fig:isobar_three_body_decay}
        \end{figure}
        To reduce the complexity of the decay model, it is usually assumed that the initial-state particle decays to the final-state particles via subsequent two-body decays, as schematically depicted in figure~\ref{fig:isobar_three_body_decay}. {\color{red} Am I assuming that?}
        For example, with this assumption, the $\PDplus \to \Ppiplus\Ppiminus\Ppiplus$ will proceed as:
        \begin{equation*}
            \PDplus\to\xi\Ppiplus,\qquad
            \xi\to\Ppiplus\Ppiminus,
        \end{equation*}
        being $\xi$ and unknown resonance like the \Pfnez{} in the S wave (the isobar with $J=0$), the \Prhozero{} in the P wave (the isobar with $J=1$), the \Pfii{} in the D wave (the isobar with $J=2$), and so on.
        This assumption has been verified in non-leptonic three-body \PD{} and \PB{} particle decays~\cite[\S~13.2]{Bevan:2014iga}.
        
        
        Each isobar amplitude $\Psi_J(\tau)$ can be factorized into a spin-dependent and a dynamic part.

        \paragraph{Spin-dependent amplitude}
        The spin-dependent amplitude, $\psi_J(\tau)$, describes the angular distribution of the decay and is fully specified by the spin quantum numbers of the initial-state particle, the isobar, and the final-state particles.
        There are, however, a number of formalisms to parametrize $\psi_J(\tau)$; examples are the helicity formalism, the Zemach formalism.
        {\color{red} Maybe show how some of these functions look like and talk about the Blatt-Wei\ss{}kopf factors.}


        \paragraph{Dynamic shape}
        The form of the dynamic shape\index{dynamic shape} of the isobar, $\Delta_I(s)$, is not dictated by first principles.
        The dynamic shape of an isobar can be expanded in terms of the mass shapes of the resonances that populate it:
        \begin{equation}\label{eq:isobar_mass_shape_expansion}
            \Delta_I(s) = \sum_{\xi} \alpha_{\xi}\Delta_{\xi}(s),
        \end{equation}
        being $s$ the squared invariant mass of each resonance.
        One of the most common forms for the dynamic shape of a resonance is the Breit-Wigner function\index{Breit-Wigner dynamic shape}
        \begin{equation}
            \Delta_{\xi}^{\text{BW}}(s) \coloneqq \frac{\sqrt{s}\,\Gamma_{\xi}(s)}{m_{\xi}^2 - s - \iu \sqrt{s}\, \Gamma_{\xi}(s)}.
        \end{equation}
        For narrow resonances, the term $\sqrt{s}\,\Gamma_{\xi}(s)$ can be replaced with the constant quantity $m_{\xi}\Gamma_{\xi}$~\cite[\S~47.2.1]{chinese_phisics}.\marginpar{{\color{red} What is $\Gamma_{\xi}$?}}
        In what follows I will use the following form of the Breit-Wigner
        \begin{equation}
            \Delta_{\xi}^{\text{BW}}(s) \coloneqq \frac{1}{m_{\xi}^2 - s - \iu m_{\xi} \Gamma_{\xi}}.
        \end{equation}
        The normalization is chosen to match the one used in the \lstinline!BreitWigner! class in \pacs{yap}.


        {\color{red} Other parametrizations are the Flatté and the pole-mass.}

        \begin{equation}
            \Delta_{\xi}^{\text{F}}(s) \coloneqq \frac{g_1}{m_{\xi}^2 - s - \iu(\rho_1 g_1^2 + \rho_2 g_2^2)}.
        \end{equation}
        \begin{equation}
            \Delta_{\xi}^{\text{PM}}(s) \coloneqq \frac{1}{m_{\xi}^2 - s},
        \end{equation}
        Where the mass of the resonance $\xi$ is complex.

        The decay amplitude~\eqref{eq:isobar_decomposition} is then 
        \begin{equation}
            \A(\tau) = \sum_{i\in\set{(J, P, C)}} \psi_i(\tau) \sum_{\xi} \gamma_{(i,\xi)} \Delta_{\xi}(s),\quad
            \text{being }
            \gamma_{(i,\xi)}\coloneqq c_i \alpha_{\xi}.
        \end{equation}
        Where I have absorbed the coefficients $c_i$ and $\alpha_\xi$ into one coefficient $\gamma_{(i,\xi)}$, that I will call \emph{free amplitude}\index{free amplitude}.
    The decay amplitude of the initial state into the final state thus includes all the possible contributions coming from the resonance $\xi$.


    Besides, contributions from the possible associations of the final-state particles (\ie~which particles form the resonance and which is the \emph{spectator}\index{spectator particle}, or \emph{bachelor}\index{bachelor particle}).
    This is the Bose symmetrization; a sum over the particle symmetrizations is implicit when not explicitly stated.


    \ac{pwa} is the tool we use to disentangle the various contributions to the decay amplitude.
    \ac{pwa} parametrizes the various $J^{PC}$-isobar contributions.


    Three-body heavy-meson decays have been analyzed via Dalitz plots.
    The isobars are decomposed in terms of the contributing resonances and the quality of the Dalitz-plot description is maximized by the appropriate choice of the intermediate-resonance set.
    An example is $ \PDplus\to\xi\Ppiplus, \xi\to\Ppiplus\Ppiminus$, with many possible resonances $\xi$.


    The result of the analyses is strongly dependent on the quality of the model used; namely:
    \begin{itemize}
        \item Assumptions on the resonant content;
        \item Invariant-mass dependence of the amplitude;
        \item Parameters of the resonances (taken from other experiments).
    \end{itemize}
    All these systematics effect result in problems that are increasingly present in the analyses of the data sets, which are becoming larger and larger.

    In \ac{pwa} the observed events {\color{red} of what} are distributed according to the decay intensity, $\Intensity(\tau)$ being $\tau$ the coordinate in the phase space of the decay.
    {\color{red} This will be the pair $(m^2,m^2)$ in the Dalitz-plot analyses.}

    {\color{red}
    It's difficult to find a heuristic model for the S wave, since it is the most populated.
    For this, we implement a model-independent description in \ac{yap}.
    }

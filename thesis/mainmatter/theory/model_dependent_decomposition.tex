\subsection{Model-dependent isobar decomposition}


    In the model-dependent approach to the isobar decomposition, the dynamic shape of the decay amplitude is expanded in terms of the resonances that populate the isobar:
    \begin{equation}\label{eq:isobar_mass_shape_expansion}
        \Delta_I(s) = \sum_{\xi\in I} \alpha_{\xi}\Delta_{\xi}(s),
    \end{equation}
    with $s$ the squared mass of the resonance, \ie~the squared four-momentum of the resonance daughter-particle system: referring to figure~\ref{fig:isobar_three_body_decay}, $s\coloneqq p_{ab}^2 = p_{\xi}^2$.
    Each coefficient $\alpha_{\xi}$ is the complex weight of the corresponding resonance $\xi$.
    Please note that the dynamic shape does not depend on the full phase-space coordinate but only on the invariant mass of the resonance it refers to.


    One of the most common parametrizations for the dynamic shape of a resonance is the relativistic Breit-Wigner distribution,\index{relativistic Breit-Wigner}\index{Breit-Wigner!relativistic}
    \begin{equation}\label{eq:rbw}
        \Delta_{\xi}^{\text{RBW}}(s) \coloneqq \frac{1}{m_{\xi}^2 - s - \iu \sqrt{s}\, \Gamma_\xi(s)},
    \end{equation}
    where the $m_{\xi}$ parameter is the nominal mass of the resonance.
    The mass-dependent width reads
    \begin{equation}
        \Gamma_\xi(s) \coloneqq \Gamma_{\xi} \, \mathcal{F}_{\!J_\xi}\!(q_{ab};R)\, \frac{m_{\xi}}{\sqrt{s}} \bigg(\frac{q_{ab}}{q_{\xi}}\bigg)^{2J_{\xi}+1},
    \end{equation}
    where the $\Gamma_{\xi}$ and $J_{\xi}$ parameters are the width and spin of the resonance;
    $\mathcal{F}$ is the Blatt-Wei\ss{}kopf form factor (see table~\ref{table:blatt_weisskopf});
    $q_{ab}$ is the break-up momentum at the measured mass $\sqrt{s}$;
    and $q_\xi$ is the break-up momentum at the nominal mass of $\xi$.
    \begin{figure}
        \centering
        \begin{tikzpicture}
    \begin{axis}[
            ylabel={Magnitude $[\si{1/(\giga\electronvolt\per\c^2)^2}]$},
            xlabel={$s$ $[\si{\giga\electronvolt^2\per\c^4}]$},
        ]
        \pgfmathsetmacro{\mPi}{.139}
        \pgfmathsetmacro{\R}{3.}
        
        \pgfmathsetmacro{\m}{.98}
        \pgfmathsetmacro{\T}{.07}

        %\addplot [domain={.279:1.73}, samples=100, smooth]
        %         gnuplot {1./sqrt((\m * \m - x * x)**2 + (\m * \T)**2)};\addlegendentry{Breit-Wigner}

        \addplot [domain={.279:1.73}, smooth, densely dotted]
                 gnuplot [raw gnuplot] {
                     set samples 200;
                     q2_ab(x) = (.5 * x) ** 2 - \mPi * \mPi;
                     q2_xi    = q2_ab(\m * \m);
                     width(x) = \T * \m * sqrt(q2_ab(x)) / (sqrt(q2_xi * x));
                     BW(x) = 1. / (\m * \m - x * x - {0,1} * x * width(x * x));
                     plot [.279:1.73] abs(BW(x));
                 };\addlegendentry{$J=0$}

        \addplot [domain={.279:1.73}, smooth, densely dashed]
                 gnuplot [raw gnuplot] {
                     set samples 200;
                     q2_ab(x) = (.5 * x) ** 2 - \mPi * \mPi;
                     q2_xi    = q2_ab(\m * \m);
                     F1(x) = sqrt(2 * x / (1 + x));
                     width(x) = \T * \m * F1(q2_ab(x) * \R * \R) * (sqrt(q2_ab(x)) ** 3) / (sqrt((q2_xi ** 3) * x));
                     BW(x) = 1. / (\m * \m - x * x - {0,1} * x * width(x * x));
                     plot [.279:1.73] abs(BW(x));
                 };\addlegendentry{$J=1$}

        \addplot [domain={.279:1.73}, smooth, dash dot]
                 gnuplot [raw gnuplot] {
                     set samples 200;
                     q2_ab(x) = (.5 * x) ** 2 - \mPi * \mPi;
                     q2_xi    = q2_ab(\m * \m);
                     F2(x)    = sqrt(13 * x * x / (x * x + 3 * x + 9));
                     width(x) = \T * \m * F2(q2_ab(x) * \R * \R) * (sqrt(q2_ab(x)) ** 5) / (sqrt((q2_xi ** 5) * x));
                     BW(x) = 1. / (\m * \m - x * x - {0,1} * x * width(x * x));
                     plot [.279:1.73] abs(BW(x));
                 };\addlegendentry{$J=2$}
    \end{axis}
\end{tikzpicture}

        \caption{Comparison of relativistic Breit-Wigner distributions for $J\in\set{0,1,2}$.}
        \label{fig:comparison_rbws}
    \end{figure}
    Figure~\ref{fig:comparison_rbws} shows a comparison of relativistic Breit-Wigner distributions~\eqref{eq:rbw} with $R = \SI{3}{\per\giga\electronvolt}$ for different values of $J$.


    If the resonance is narrow, namely if all relevant thresholds are far away from the nominal mass of the $\xi$ resonance, the term $\sqrt{s}\,\Gamma_\xi(s)$ in the denominator of equation~\eqref{eq:rbw} can be replaced with a constant quantity, $m_{\xi}\Gamma_{\xi}$,~\cite[\S~47.2.1]{chinese_phisics}.
    With this substitution, one gets the non-relativistic Breit-Wigner distribution,\index{Breit-Wigner}
    \begin{equation}\label{eq:bw}
        \Delta_{\xi}^{\text{BW}}(s) \coloneqq \frac{1}{m_{\xi}^2 - s - \iu m_{\xi} \Gamma_{\xi}},
    \end{equation}
    which is normalized such that $\Delta_{\xi}^{\text{BW}}(m_{\xi}^2) = \iu / m_\xi\Gamma_\xi$.


    Another possible dynamic shape is the pole-mass distribution,\marginpar{motivate the different distributions}
    \begin{equation}\label{eq:pole_mass}
        \Delta_{\xi}^{\text{PM}}(s) \coloneqq \frac{1}{m_{\xi}^2 - s},\quad
        \text{with }m_\xi = \Re (m_\xi) + \iu\Im(m_\xi),
    \end{equation}
    where the resonance-mass parameter, $m_\xi$, is a complex number.
    The distribution~\eqref{eq:pole_mass} has been proposed by \citeauthor{PhysRevD.71.054030} to parametrize the \Psigma{} resonance~\cite{PhysRevD.71.054030}.
    \begin{figure}
        \centering
        \begin{tikzpicture}
    \begin{axis}[
            ylabel={Magnitude $[\si{1/(\giga\electronvolt\per\c^2)^2}]$},
            xlabel={$s$ $[\si{\giga\electronvolt^2\per\c^4}]$},
        ]
        \pgfmathsetmacro{\m}{.98}
        \pgfmathsetmacro{\T}{.07}
        \addplot [domain={.279:1.73}, samples=100, smooth] gnuplot {1./sqrt((\m * \m - x * x)**2 + (\m * \T)**2)};\addlegendentry{Breit-Wigner}

        \addplot [domain={.279:1.73}, samples=100, smooth, densely dashed]
                 gnuplot {1./sqrt((\m * \m - \T * \T - x * x)**2 + 4 * (\m * \T)**2)};\addlegendentry{Pole mass}
    \end{axis}
\end{tikzpicture}

        \caption{Comparison between the non-relativistic and relativistic Breit-Wigner, and the pole-mass dynamic shapes.
                 The first two are very close to each other.
        }
       % The parameters I used are $(\SI{.98}{\giga\electronvolt/\c^2}, \SI{.07}{\giga\electronvolt/\c^2})$, corresponding to $(m_\xi,\Gamma_\xi)$ for the Breit-Wigner shape; and $(\Re(m_\xi),\Im(m_\xi))$ for the pole-mass shape.}
        \label{fig:bw_pm_comparison}
    \end{figure}
    Figure~\ref{fig:bw_pm_comparison} shows a comparison between the non-relativistic Breit-Wigner and the pole-mass dynamic shapes.

    
    The \Pfnez{} resonance lies close to both the \Ppiplus{}\Ppiminus{} and the \PKplus{}\PKminus{} decay channels and is usually parametrized by a \citeauthor{FLATTE1976224}\index{Flatte@\citeauthor{FLATTE1976224}} distribution~\cite{FLATTE1976224}, which takes into account the opening of the second channel:
    \begin{equation}\label{eq:flatte}
        \Delta_{\xi}^{\text{F}}(s) \coloneqq \frac{1}{m_{\xi}^2 - s - \iu(g_1^2 \rho_{1\!}(s)  + g_2^2 \rho_2(s))},
    \end{equation}
    $g_1$ and $g_2$ being the coupling constants to the channels;
    and $\rho_{1\!}(s)$ and $\rho_2(s)$ the (complex) phase-space factors,
    \begin{equation}\label{eq:flatte_phase_space}
        \rho_i(s) = 2\frac{q_i(s)}{\sqrt{s}},
    \end{equation}
    with $q_i(s)$ the break-up momentum of the daughter particles of the $i$-th decay channel.
    Please note that the phase-space parametrization \citeauthor{FLATTE1976224} introduced in~\cite{FLATTE1976224} is not the one reported in equation~\eqref{eq:flatte_phase_space}.

    In the model-dependent \ac{pwa}, the isobar decay amplitude~\eqref{eq:isobar_decomposition} finally reads
    \begin{equation}\label{eq:redundant_model_dependent_isobar_decomposition}
        \Psi_I(\tau) =  \sum_{\xi\in I} \gamma_\xi \,\psi_{J_\xi\!}(\tau)\,\mathcal{F\!}_{J_\xi\!}(\tau)\,\Delta_{\xi}(s)\,\A_{d_1}\!(m_{d_1\!}; \tau)\,\A_{d_2}\!(m_{d_2\!}; \tau),\quad
        \text{with }
        \gamma_\xi\coloneqq c_I \alpha_{\xi}.
    \end{equation}
    I have absorbed the coefficients $c_I$ and $\alpha_\xi$ into one coefficient, $\gamma_{\xi}$, that I will call \emph{free amplitude}\index{free amplitude}.
    Please note that the free amplitudes in equation~\eqref{eq:redundant_model_dependent_isobar_decomposition} do not need the isobar label: since each resonance is in a well-defined isobar, a double index, $(I,\xi)$, would be redundant.
    Moreover, for the same reason, the common angular part in equation~\eqref{eq:redundant_model_dependent_isobar_decomposition} can be collected in front of the sum:
    \begin{equation}\label{eq:non_redundant_model_dependent_isobar_decomposition}
        \Psi_I(\tau) =  \psi_{J_I\!}(\tau)\,\mathcal{F\!}_{J_I\!}(\tau)\,\A_{d_1}\!(m_{d_1\!}; \tau)\,\A_{d_2}\!(m_{d_2\!}; \tau)\sum_{\xi\in I} \gamma_\xi \,\Delta_{\xi}(s).
    \end{equation}


    At this point, I would like to stress that in the expansion~\eqref{eq:non_redundant_model_dependent_isobar_decomposition} there are several sources of systematic uncertainties.
    The resonance content of each isobar is not known and highly affects the quality of the description the model provides.
    Moreover, some resonances---like the \Pfnez{}, the \Pfotsz, and the pole \Psigma{}---are not well established~\cite[\S~B]{PhysRevD.76.012001}.


    The dynamic shape of each resonance---along with its parameters---has to be experimentally determined: our knowledge about the dynamic shape comes from previous experiments performed under different conditions.
    Thus, when modelling a decay, it has to be silently assumed that neither the dynamic shape of the resonance, nor its parameters are affected by the particular interaction in the decay.


    Due to the availability of larger and larger experimental data sets, the aforementioned limitations of the model-dependent isobar formalism are becoming critical.
    In the next section, I will present how such limitations can be circumvented by means of the model-independent isobar formalism.



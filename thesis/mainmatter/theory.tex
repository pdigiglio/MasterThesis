\chapter{Theory introduction}

    \Ac{pwa} is a standard technique in scattering theory.
    It allows to write down the the scattering probability amplitude in terms of Legendre polynomials, the eigenvectors of the angular momentum operator, each representing a partial\index{partial wave} wave~\cite[\S~11.2]{griffiths_intro_qm}.


    In this chapter, I will show how we apply the \ac{pwa} to heavy-meson decays to extract the various contributions to the decay amplitude.
    I will focus on the isobar decomposition of a spinless parent particle to three pseudo-scalar daughter particles in the final state.
    I will restrict to this case as I want to apply the formalism to the $\PDplus\to\Ppiplus\Ppiminus\Ppiplus$ decay.

    \section{The isobar decomposition}

    In the isobar formalism, a resonance\index{resonance} is a bound state of particles with well-defined quantum numbers.
    The quantum numbers that characterize a resonance are its isospin, its spin, $J$, its parity, $P$, and its charge-conjugation parity, $C$.
    The notation usually adopted to indicate the resonance quantum numbers is $J^{PC}$.


    A set of bound states with the same quantum numbers is called \emph{isobar}\index{isobar}.
    Consequently, the decomposition of a decay amplitude in terms of isobars is called \emph{isobar model}.

    In this formalism, the decay amplitude of an initial-state particle of invariant mass $M$ into a final state whose coordinate in the phase space is $\tau$ can be written as a coherent sum of isobar amplitudes,
    \begin{equation}\label{eq:isobar_decomposition}
        \A(M,\tau) = \sum_{I\in\set{(J, P, C)}} c_I \Psi_I(M, \tau),
    \end{equation}
    being $c_I$ a complex parameter quantifying the magnitude and (relative) phase of the isobar amplitude labeled by $I$.
    In the context of the isobar decomposition, the partial waves correspond to the isobars.


    In heavy-meson decays, a known initial-state particle decays to a known final state.
    As the initial-state particle has a well-defined invariant mass, and fixes the $C$ and $P$ quantum numbers of the decay, the isobar decomposition can be simplified as follows:
    \begin{equation}\label{eq:heavy_meson_isobar_decomposition}
        \A(\tau) = \sum_{J} c_J \Psi_J(\tau),
    \end{equation}
    where I dropped the dependence on the initial-particle mass; and parity, and charge-conjugation parity quantum numbers.
    The next sections will be on how to find a mathematical expression for the isobar amplitude $\Psi_J(\tau)$.
    {\color{red} which angular momentum/spin quantum numbers are allowed?}





    \subsection{Model-dependent isobar decomposition}

        \begin{figure}
            \centering
            \begin{tikzpicture}[
        particle/.style = {circle},
        P/.style        = {particle, ball color=black!20},
        a/.style        = {particle, ball color=blue!20},
        b/.style        = {particle, ball color=green!20},
        c/.style        = {particle, ball color=red!20},
    ]

    % Parent particle
    \node[P] (P) at (0,0) {$X$};
    \draw [<->, color=black!50] ($(P) + (-25:1.5)$) arc (-25:25:1.5) node [midway, label={right:$L$}] {};
    %\draw (330:1) arc (30:1);

    \pgfmathsetmacro{\xDist}{6}
    \pgfmathsetmacro{\yDist}{3}

    \node[a] (a) at ($(P) + (\xDist, \yDist)$) {$a$};
    \node[b] (b) at ($(P) + (\xDist, 0)$)      {$b$};
    \node[c] (c) at ($(P) + (\xDist,-\yDist)$) {$c$};
    % Resonance
    \node[ ] (r) at ($(P) + (.5*\xDist, .5*\yDist)$) {$\xi$};
    \draw [<->, color=black!50] ($(r) + (-25:1.5)$) arc (-25:25:1.5) node [midway, label={right:$J_{\xi}$}] {};

    % Segments
    \draw[->] (P) -- (r);
    \draw[->] (r) -- (a);
    \draw[->] (r) -- (b);
    \draw[->] (P) -- (c);
\end{tikzpicture}

            \caption{$P\to abc$ decay through a resonance $\xi$ in the $ab$ channel.}
            \label{fig:isobar_three_body_decay}
        \end{figure}
        To reduce the complexity of the decay model, it is usually assumed that the initial-state particle decays to the final-state particles via subsequent two-body decays, as schematically depicted in figure~\ref{fig:isobar_three_body_decay}. {\color{red} Am I assuming that?}
        For example, with this assumption, the $\PDplus \to \Ppiplus\Ppiminus\Ppiplus$ will proceed as:
        \begin{equation*}
            \PDplus\to\xi\Ppiplus,\qquad
            \xi\to\Ppiplus\Ppiminus,
        \end{equation*}
        being $\xi$ and unknown resonance like the \Pfnez{} in the S wave (the isobar with $J=0$), the \Prhozero{} in the P wave (the isobar with $J=1$), the \Pfii{} in the D wave (the isobar with $J=2$), and so on.
        This assumption has been verified in non-leptonic three-body \PD{} and \PB{} particle decays~\cite[\S~13.2]{Bevan:2014iga}.
        
        
        Each isobar amplitude $\Psi_J(\tau)$ can be factorized into a spin-dependent and a dynamic part.

        \paragraph{Spin-dependent amplitude}
        The spin-dependent amplitude, $\psi_J(\tau)$, describes the angular distribution of the decay and is fully specified by the spin quantum numbers of the initial-state particle, the isobar, and the final-state particles.
        There are, however, a number of formalisms to parametrize $\psi_J(\tau)$; examples are the helicity formalism, the Zemach formalism.
        {\color{red} Maybe show how some of these functions look like and talk about the Blatt-Wei\ss{}kopf factors.}


        \paragraph{Dynamic shape}
        The form of the dynamic shape\index{dynamic shape} of the isobar, $\Delta_I(s)$, is not dictated by first principles.
        The dynamic shape of an isobar can be expanded in terms of the mass shapes of the resonances that populate it:
        \begin{equation}\label{eq:isobar_mass_shape_expansion}
            \Delta_I(s) = \sum_{\xi} \alpha_{\xi}\Delta_{\xi}(s),
        \end{equation}
        being $s$ the squared invariant mass of each resonance.
        One of the most common forms for the dynamic shape of a resonance is the Breit-Wigner function\index{Breit-Wigner dynamic shape}
        \begin{equation}
            \Delta_{\xi}^{\text{BW}}(s) \coloneqq \frac{\sqrt{s}\,\Gamma_{\xi}(s)}{m_{\xi}^2 - s - \iu \sqrt{s}\, \Gamma_{\xi}(s)}.
        \end{equation}
        For narrow resonances, the term $\sqrt{s}\,\Gamma_{\xi}(s)$ can be replaced with the constant quantity $m_{\xi}\Gamma_{\xi}$~\cite[\S~47.2.1]{chinese_phisics}.\marginpar{{\color{red} What is $\Gamma_{\xi}$?}}
        In what follows I will use the following form of the Breit-Wigner
        \begin{equation}
            \Delta_{\xi}^{\text{BW}}(s) \coloneqq \frac{1}{m_{\xi}^2 - s - \iu m_{\xi} \Gamma_{\xi}}.
        \end{equation}
        The normalization is chosen to match the one used in the \lstinline!BreitWigner! class in \pacs{yap}.


        {\color{red} Other parametrizations are the Flatté and the pole-mass.}

        \begin{equation}
            \Delta_{\xi}^{\text{F}}(s) \coloneqq \frac{g_1}{m_{\xi}^2 - s - \iu(\rho_1 g_1^2 + \rho_2 g_2^2)}.
        \end{equation}
        \begin{equation}
            \Delta_{\xi}^{\text{PM}}(s) \coloneqq \frac{1}{m_{\xi}^2 - s},
        \end{equation}
        Where the mass of the resonance $\xi$ is complex.

        The decay amplitude~\eqref{eq:isobar_decomposition} is then 
        \begin{equation}
            \A(\tau) = \sum_{i\in\set{(J, P, C)}} \psi_i(\tau) \sum_{\xi} \gamma_{(i,\xi)} \Delta_{\xi}(s),\quad
            \text{being }
            \gamma_{(i,\xi)}\coloneqq c_i \alpha_{\xi}.
        \end{equation}
        Where I have absorbed the coefficients $c_i$ and $\alpha_\xi$ into one coefficient $\gamma_{(i,\xi)}$, that I will call \emph{free amplitude}\index{free amplitude}.
    The decay amplitude of the initial state into the final state thus includes all the possible contributions coming from the resonance $\xi$.


    Besides, contributions from the possible associations of the final-state particles (\ie~which particles form the resonance and which is the \emph{spectator}\index{spectator particle}, or \emph{bachelor}\index{bachelor particle}).
    This is the Bose symmetrization; a sum over the particle symmetrizations is implicit when not explicitly stated.


    \ac{pwa} is the tool we use to disentangle the various contributions to the decay amplitude.
    \ac{pwa} parametrizes the various $J^{PC}$-isobar contributions.


    Three-body heavy-meson decays have been analyzed via Dalitz plots.
    The isobars are decomposed in terms of the contributing resonances and the quality of the Dalitz-plot description is maximized by the appropriate choice of the intermediate-resonance set.
    An example is $ \PDplus\to\xi\Ppiplus, \xi\to\Ppiplus\Ppiminus$, with many possible resonances $\xi$.


    The result of the analyses is strongly dependent on the quality of the model used; namely:
    \begin{itemize}
        \item Assumptions on the resonant content;
        \item Invariant-mass dependence of the amplitude;
        \item Parameters of the resonances (taken from other experiments).
    \end{itemize}
    All these systematics effect result in problems that are increasingly present in the analyses of the data sets, which are becoming larger and larger.

    In \ac{pwa} the observed events {\color{red} of what} are distributed according to the decay intensity, $\Intensity(\tau)$ being $\tau$ the coordinate in the phase space of the decay.
    {\color{red} This will be the pair $(m^2,m^2)$ in the Dalitz-plot analyses.}

    {\color{red}
    It's difficult to find a heuristic model for the S wave, since it is the most populated.
    For this, we implement a model-independent description in \ac{yap}.
    }


%        \subsection{Isobar model}
%        In the isobar model only two-body decays are allowed.
The resonance, in turn, will decay into the particles $a$ and $b$.


This way, the~\eqref{eq:pwa_ang_mom} will be further decomposed into a coherent sum of the amplitudes that populate each wave:
\begin{equation}
    \A_L = \sum_r a_L^r \A_L^r,
\end{equation}
being $r$ and index for the alloewd resonances.
Resonances have to fulfill conservation laws at each decay vertex, just like particles.
Resonances are intermediary states that cannot, by their nature, be observed.





%            \subsubsection{How to model the amplitude of a resonance}
%            Motivate the decomposition in $F_P F_r T_r W_r$.

    \section{Dalitz-plot analysis}
    \section{Dalitz-plot analysis}

    In a two-body decay, the energy-momentum conservation fully determines the energy of the final-state particles in the center-of-mass frame.
    In a three-body decay, after requiring energy and momentum conservation, there are still five remaining \acp{dof}.
    If the initial- and final-state particles are spinless, up to an arbitrary rotation of the decay plane, two \acp{dof} remain.
    Therefore, for the decay under consideration, the phase-space coordinate, $\tau$, is a two-element vector.

   
   There are several possible choices for the observable pair to parametrize $\tau$.
    It is thus convenient to choose a couple of observables such that the phase-space measure is constant within the kinematically-allowed region they span.
    Two such observables are the squared invariant masses of two pairs of final-state particles, i.e.~$m_{ab}^2 = p_{ab}^2$ and $m_{bc}^2 = p_{bc}^2$, with $a$, $b$, and $c$ the labels for the final-state particles (see figure~\ref{fig:isobar_three_body_decay}).


    With this parametrization of the phase-space coordinate, the differential decay probability is~\cite[\S~13.1.1]{Bevan:2014iga}
    \begin{equation}
        \ud\Gamma = \frac{1}{(2\pi)^3}\frac{1}{32 m_X^3}\abs{\A(m_{ab}^2, m_{bc}^2)}^2\!
        \ud m_{ab}^2 \!\ud m_{bc}^2.
    \end{equation}
    As the phase-space measure is a constant function throughout the whole phase space, any non-uniformity in the distribution of the events is to be ascribed to the decay amplitude only.
    \begin{figure}
        \centering
        \begin{tikzpicture}
    \begin{axis} [
        width = .7\textwidth,
        view={0}{90},%
        xlabel={$m_{bc}^2$},
        ylabel={$m_{ab}^2$},
        ticks=none,
        grid=none
    ]
        \addplot3 [dalitz]
          gnuplot [raw gnuplot] { splot "data/resonance_example/s_wave_resonance_mcmc.txt" using 1:2:3 };
    \end{axis}
\end{tikzpicture}

        \caption{S-wave resonance in the $ab$ channel.}
        \label{fig:s_wave_resonance_example}
    \end{figure}
    \begin{figure}
        \centering
        \begin{tikzpicture}
    \begin{axis} [
        width = .7\textwidth,
        view={0}{90},%
        xlabel={$m_{bc}^2$},
        ylabel={$m_{ab}^2$},
        ticks=none,
        grid=none
    ]
        \addplot3 [dalitz]
          gnuplot [raw gnuplot] { splot "data/resonance_example/p_wave_resonance_mcmc.txt" using 1:2:3 };
    \end{axis}
\end{tikzpicture}

        \caption{P-wave resonance in the $ab$ channel.}
        \label{fig:p_wave_resonance_example}
    \end{figure}
    \begin{figure}
        \centering
        \begin{tikzpicture}
    \begin{axis} [
        width = .7\textwidth,
        view={0}{90},%
        xlabel={$m_{bc}^2$},
        ylabel={$m_{ab}^2$},
        ticks=none,
        grid=none
    ]
        \addplot3 [dalitz]
          gnuplot [raw gnuplot] { splot "data/resonance_example/d_wave_resonance_mcmc.txt" using 1:2:3 };
    \end{axis}
\end{tikzpicture}

        \caption{D-wave resonance in the $ab$ channel.}
        \label{fig:d_wave_resonance_example}
    \end{figure}
    The resonances appear as bands within the allowed region, as shown in Dalitz plots~\ref{fig:s_wave_resonance_example}, \ref{fig:p_wave_resonance_example}, and \ref{fig:d_wave_resonance_example}.


    \begin{figure}
        \centering
        \begin{tikzpicture}
%    \pgfmathsetmacro{\mTwoA}{(.13957)^2} % pi+ mass squared [GeV/c^2]
%    \pgfmathsetmacro{\mTwoB}{(.13957)^2} % pi- mass squared [GeV/c^2d
%    \pgfmathsetmacro{\mTwoC}{(.13957)^2} % pi+ mass squared [GeV/c^2d
%    \pgfmathsetmacro{\mTwo}{(1.86963)^2} % D+  mass squared [GeV/c^2d      

    \pgfmathsetmacro{\mTwoA}{(.17)^2}
    \pgfmathsetmacro{\mTwoB}{(.17)^2}
    \pgfmathsetmacro{\mTwoC}{(.17)^2}
    \pgfmathsetmacro{\mTwo}{(1.)^2}

    \pgfmathsetmacro{\lBound}{(sqrt(\mTwoA) + sqrt(\mTwoB))^2}
    \pgfmathsetmacro{\uBound}{(sqrt(\mTwo)  - sqrt(\mTwoC))^2}
    \pgfmathsetmacro{\lyBound}{(sqrt(\mTwoC) + sqrt(\mTwoB))^2}
    \pgfmathsetmacro{\uyBound}{(sqrt(\mTwo)  - sqrt(\mTwoA))^2}

    \begin{axis}[
        ylabel={$m_{ab}^2$},
        xlabel={$m_{bc}^2$},
        ytick={\lBound,\uBound},
        yticklabels={$(m_a + m_b)^2$, $(M-m_c)^2$},
        xtick={\lyBound,\uyBound},
        xticklabels={$(m_b + m_c)^2$, $(M-m_a)^2$},
        yticklabel style={sloped like y axis},
        grid = major,
        declare function = { E_b(\t) = ((\t -\mTwoB + \mTwoA)/(2*sqrt(\t))); },
        declare function = { P_b(\t) = sqrt(E_b(\t)^2 -\mTwoB); },
        declare function = { E_c(\t) = ((\mTwo -\t -\mTwoC)/(2*sqrt(\t))); },
        declare function = { P_c(\t) = sqrt(E_c(\t)^2 -\mTwoC); },
        enlargelimits=.12,
    ]

        \addplot[
            name path = A,
            black,
            opacity=0,
            domain = {\lBound:\uBound},
            samples=250,
        ] {(E_b(x) + E_c(x))^2 - (P_b(x) - P_c(x))^2};

        \addplot[
            name path = B,
            black,
            opacity=0,
            domain = {\lBound:\uBound},
            samples=250,
        ] {(E_b(x) + E_c(x))^2 - (P_b(x) + P_c(x))^2};

        \addplot[blue!50, opacity=.3] fill between[of=A and B];

    \end{axis}
\end{tikzpicture}

        \caption{Kinematically-allowed region in the phase space of a three-body decay. The corners correspond to the configuration in which one of the daughter particles is produced at rest (in the frame of the parent particle).}
        \label{fig:dalitz_kinematically_allowed}
    \end{figure}
    The boundaries of the kinematically-allowed phase-space region, shown in figure~\ref{fig:dalitz_kinematically_allowed}, can be expressed in terms of a constraint on $m_{bc}^2$ as follows:
    \begin{equation}
        \begin{aligned}
            (m_{bc}^2)_{\text{max}} &= (E_b + E_c)^2 - (\abs{\vec{p}_b} - \abs{\vec{p}_c})^2,\\
            (m_{bc}^2)_{\text{min}} &= (E_b + E_c)^2 - (\abs{\vec{p}_b} + \abs{\vec{p}_c})^2,
        \end{aligned}
    \end{equation}
    where
    \begin{equation}
        E_b = \frac{m_{ab}^2 - m_a^2 + m_b^2}{2m_{ab}},\text{ and }
        E_c = \frac{m_X^2 - m_{ab}^2 - m_c^2}{2 m_{ab}}
    \end{equation}
    are the energies of the $b$ and $c$ final-state particles in the rest frame of the $ab$ system, and
    \begin{equation}
        \abs{\vec{p}_b} = \sqrt{E_b^2 - m_b^2},\text{ and }
        \abs{\vec{p}_c} = \sqrt{E_c^2 - m_c^2}
    \end{equation}
    the corresponding 3-momenta magnitudes.
    The shape of the phase space, in figure~\ref{fig:dalitz_kinematically_allowed}, depends on the values of the final-state-particle masses:
    if $m_a = m_b = m_c = 0$, it will stretch to a triangle; whereas, if $m_a + m_b + m_c = m_X$, it will shrink to a point---thus making the decay impossible.

    \begin{figure}
        \centering

        \subfloat[]%
                 [P wave.]%
                 {\begin{tikzpicture}
    \begin{axis} [dalitz_plot]

        \addplot3 [dalitz]
          gnuplot [raw gnuplot] { splot "data/angular_part/uniform_p_wave_mcmc.txt" using 1:2:3 };
    \end{axis}
\end{tikzpicture}
}

        \subfloat[]%
                 [D wave.]%
                 {\begin{tikzpicture}
    \begin{axis} [dalitz_plot]

        \addplot3 [dalitz]
          gnuplot [raw gnuplot] { splot "data/angular_part/uniform_d_wave_mcmc.txt" using 1:2:3 };
    \end{axis}
\end{tikzpicture}
}

        \caption{Angular part of the decay amplitude (spin part in the Zemach formalism times Blatt-Wei\ss{}kopf factor) for the P and D waves.}
        \label{fig:dalitz_angular_parts}
    \end{figure}



    \section{Isobar formalism}
    Experiments have shown that nonleptonic three-body decays proceed through intermediate two-body resonant decays.
Thus, the amplitude is modeled as a coherent sum of two-body decays plus a non-resonant contribution:
\begin{equation}\label{eq:general_decay_amplitude}
    \A(m_{ab}^2, m_{bc}^2) = \sum_r a_r \eu^{\iu\phi_r}\A_r(m_{ab}^2,m_{bc}^2)
    + a_{\textup{NR}} \eu^{\iu \phi_{\textup{NR}}} \A_{\textup{NR}}(m_{ab}^2,m_{bc}^2).
\end{equation}
The parameters $a_r$ and $\phi_r$ are the magnitude and phase of the amplitude for the resonance component $r$.
The same interpretation holds for $a_\text{NR}$ and $\phi_\text{NR}$.
These parameters are modeled by the \lstinline!FreeAmplitude! class in \pac{yap}.


One way to parametrize the resonant components of the decay amplitude~\eqref{eq:general_decay_amplitude} is through the isobar model (or isobar formalism).
In this form, the function $\A_r$ is decomposed in the following product:
\begin{equation}
    \A_r = F_P\,F_r\, T_r\, W_r.
\end{equation}
Here, the product of the dynamical function, $T_r$, and the angular distribution, $W_r$, is the propagator of the resonance $r$.
The functions $F_P$ and $F_r$ are the transition form factors of the parent particle and the resonance.

The dynamical function is usually described by a relativistic Breit-Wigner with a mass-dependent width:
\begin{equation}
    T_r = \frac{1}{m_r^2 - m_{ab}^2 - \iu m_r \Gamma_{ab}}
\end{equation}

The angular distribution is described either by using Zemach tensors of by using the helicity formalism.


The form factors $F_P$ and $F_r$ usually use the Blatt-Weisskopf parametrization for the decay vertex.


The $K$-matrix is an alternative approach to the isobar formalism for the amplitude calculation.


Model independent \gls{pwa} and binned analysis are examples of model independent Dalitz-plot analysis.


    \section{Freed-isobar partial-wave analysis}
    The accuracy on the parameters of resonances strongly depends on the quality of the decay model that has been adopted to perform the fit.
In particular, the isobar formalism has some limits that must be overcome if the decay description is to be improved.
The knowledge of isobars comes from past experiments that had to determine mass and width of intermediate resonances.
This means that the resonance parameters (and, sometimes, shapes as in the case of ?) are only known with limited precision, thus introducing systematic effects in the decay modelling.
Thus it is useful to reduce the analysis dependence on the isobar model and extract the intermediate resonance shapes directly from the experiment itself.


In the freed-isobar \ac{pwa}, a set of binned function is used to describe the unknown shape of an isobar amplitude.
This way, there is no need to use any knowledge on the isobar from past experiment.
The kinematically-allowed phase-space mass range for the final state $ab$, which is $[m_a + m_b, M - m_c]$, is partitioned into a $N_\text{bins}$ number of bins.
\begin{figure}
    \centering
    \begin{tikzpicture}
        \pgfmathtruncatemacro{\a}{3}
        \pgfmathtruncatemacro{\b}{1}
        \pgfmathtruncatemacro{\c}{2}

        \begin{scope}[color=black!50]
            \coordinate (o) at (-1, 0);
            \draw (o) -- +(\a + 2, 0) coordinate (a);
            \draw [densely dashed] (a) -- +(\b, 0) coordinate (b);
            \draw [->] (b) -- +(\c, 0) coordinate (c);
        \end{scope}

        \begin{scope}
            % vertical label shift
            \pgfmathsetmacro{\s}{-.5}

            \foreach \x in {0,1,...,\a} {
                \node [draw, inner sep=1pt, circle, fill=black] (x\x) at (\x, 0) {};
                \node (sx\x) at ($(x\x) + (0, \s)$)  {$m_\x$};
            }

            \node [draw, inner sep=1pt, circle, fill=black] (last) at (\a + \b + 2, 0) {};
            \node (slast) at ($(last) + (0, \s)$) {$m_{N_\text{bins}}$};

        \end{scope}

        \node () at (x0) [label={above:$m_a + m_b$}] {};
        \node () at (last) [label={above:$M - m_c$}] {};


    \end{tikzpicture}
    \caption{Invariant-mass partition into $N_\text{bins}$ bins.}
    \label{fig:invariant-mass-partition}
\end{figure}
Referring to the labeling scheme showed in the figure~\ref{fig:invariant-mass-partition}, let $B_i \coloneqq [m_{i-1}, m_i)$ be the $i$-th right-open mass range, where $i$ ranges from $1$ to $N_\text{bins}$.
The bin functions are defined as follows:
\begin{equation}
    \Delta_i(m_X; m_a, m_b) =
    \begin{cases}
        1 &\text{if}\ m_X \in B_i, \\
        0 &\text{otherwise}.
    \end{cases}
\end{equation}
The dynamical function of the isobar can now be constructed as a linear combination of the bin functions:
\begin{equation}
    \Delta(m_X; m_a, m_b) = \sum_{i=1}^{N_\text{bins}} a_i \Delta_i(m_X;m_a, m_b),
\end{equation}
which is a step function depending on the complex parameters $\Set{a_1, \dots, a_{N_\text{bins}}}$.
The unknown parameters $\Set{ a_i }$ will be determined from a fit of the decay model to the experimental data.


At this point it is worth noting that this step-like description of the dynamic function of an isobar introduces $N_\text{bins}$ fit parameters, which drastically increases the complexity of fitting the model to the data.
This problem can be partially solved considering that:
\begin{itemize}
    \item 
        This formalism allows mixing fixed and freed resonance dynamic functions in a model, so that the number of free fit parameters can be somehow contained;
    \item
        The partition bins do not have to be uniform. By means of a non-uniform mass-range partition, making narrower bins in the mass interval where a resonance is expected, the number of necessary fit parameters can be optimized.
\end{itemize}

{\color{red}
(Quote Fabian's work as unpublished?)
}


        \subsection{Zero modes}
        Introduction of the isobar-freed fit leads to linear dipendencies among the waves in the fit model.
This can be seen by means of the integral matrix (now $w$ and $v$ are wave indices)
\begin{equation}
    I_{wv} \coloneqq \int_\Omega \psi_w^*(\tau)\,\psi_v(\tau)\ud\tau
\end{equation}
which represents the integrated overlap of the waves $w$ and $v$ over the phase space.
Since $I$ is hermitian, its eigenvalues are real.
So
\begin{equation}
    I^{\text{Tot}} \coloneqq \int_\Omega \abs{\A(\tau)}^2\!\ud \tau = \sum_{v,w \in \Set{\text{waves}}} T_w^* I_{wv} T_v.
\end{equation}


    \section{Model-independent descriptions}

    \subsection{Motivation}

    \paragraph{Why should I want to use one?}

    I'm not required to guess what's in the resonance thus introducing biases by an incorrect or incomplete model.
    The drawback is that the fit parameters increase a lot but this is not an issue due to the magnitude of the current data samples.


    \paragraph{Who guarantees me that you're telling the truth and the description works?}

    The fit result obtained via a model-independent description have been compared to the isobar ones.
    And they're compatible, you skeptical moron.


    Let's quote some previous collaborations tha already used a model-independent description to analyze their decays, such as~\cite{PhysRevD.73.032004,Link200914}.


    The approach is flexible because it allows a mixed formalism.
    The {\small FOCUS} collaboration, for instance, used it to describe the S-wave component of a $\PKminus\Ppiplus$ system from a $\PDplus \to \PKminus\Ppiplus\Ppiplus$ decay while still using a model dependent description for P and D waves~\cite{Link200914}.



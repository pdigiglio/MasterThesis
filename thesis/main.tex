\documentclass[
	10pt,
	twoside,
	openright,
%	final
]{scrbook}
\usepackage[utf8]{inputenc}
\usepackage[italian,english]{babel}

% TODO Setup typography (fonts, mparhack, microtype...)
%\input{config/typographyConfig}
\renewcommand{\sfdefault}{iwona}

% Define some useful colors
\usepackage[dvipsnames]{xcolor}

% Define some colors
\definecolor{webgreen}{rgb}{0,.5,0}
\definecolor{webbrown}{rgb}{.6,0,0}
\definecolor{Maroon}{cmyk}{0, 0.87, 0.68, 0.32}
\definecolor{RoyalBlue}{cmyk}{1, 0.50, 0, 0}

% Setup plots
\usepackage{pgfplots}
\pgfplotsset{compat=newest}
\pgfplotsset{
    grid = major,
    major grid style={densely dotted},
    enlargelimits=.05,
    width = .75\textwidth
}
\usepgfplotslibrary{
    fillbetween,
    colormaps,
    units
}

\usepgfplotslibrary{external}
\tikzexternalize%[prefix=TikZ_pictures/]

\usepackage{tikz}
\usepackage{tikz-uml} % For UML diagrams

\usetikzlibrary{
    calc,
    shapes
}
\tikzset{>=stealth,}
\pgfplotsset{
    phase_plot/.style={%
        yticklabel style={sloped like y axis},%
        xlabel = {Mass of the final-state $\Ppiplus\Ppiminus$ system $[\si{\giga\electronvolt/\square\c}]$},
        %x unit = {\si{\giga\electronvolt/\square\c}},
        ylabel = {Dynamic-shape phase $[\si{deg}]$},
        %y unit = {\si{deg}}
    },
    amplitude_plot/.style={%
        yticklabel style={sloped like y axis},%
        xlabel = {Mass of the final-state $\Ppiplus\Ppiminus$ system $[\si{\giga\electronvolt/\square\c}]$},
        %x unit = {\si{\giga\electronvolt/\square\c}},
        ylabel = {Dynamic-shape magnitude $[\si{1\per(\giga\electronvolt/\square\c)^2}]$},
        %y unit = {\si{1\per(\giga\electronvolt/\square\c)^{-2}}}, % misaligns the plots
    },
    fit/.style={%
        mark options={%
            mark size=.75,%
            color=Maroon,%
            mark=*%
        },%
        ybar interval,%
        color=ForestGreen,%
        fill=LimeGreen,%
        fill opacity=.3,%
        % -- error bars --------
        error bars/y dir=both,%
        error bars/y explicit,%
        error bars/error bar style={color=Maroon},%
    },%
    guess/.style={%
        line width=.6,
        color=blue,
        mark=none,
        %mark options={%
        %    mark size=.75,
        %    mark=square*%
        %},%
    },%
    dalitz_plot/.style={%
        yticklabel pos=right,
        view={0}{90},%
        xlabel = {$m_{\Ppiplus\Ppiminus}^2$ $[\si{\giga\electronvolt^2\!/\c^4}]$},%
        ylabel = {$m_{\Ppiplus\Ppiminus}^2$ $[\si{\giga\electronvolt^2\!/\c^4}]$},%
        %x unit = {\si{\giga\electronvolt^2\!/\c^4}},%
        %y unit = {\si{\giga\electronvolt^2\!/\c^4}},%
        %colormap/viridis high res,%
        %colormap/winter,%
        colorbar left=true,%
        colorbar left/.append style={%
            yticklabel style={sloped like x axis},%
        },%
        %axis equal,%
    },%
    dalitz/.style={%
        scatter,%
        scatter src=z,%
        only marks,%
        mark options={%
            mark=square*,%
            mark size=1.1,%
        }%
    }%
}


% Setup units
\usepackage{siunitx}

\sisetup{
%	range-phrase=$-$,
	separate-uncertainty,
%	input-decimal-marker={.},
	output-decimal-marker={.},
	exponent-product = \cdot,
}

% Additional units

\usepackage{eurosym}
\DeclareSIUnit{\EUR}{\text{\euro{}}}
\DeclareSIUnit{\c}{\text{c}}

% Setup math
\usepackage{amsmath}

% For \ltrans
\usepackage{leftidx}

% Load constants
% Some useful math constants

% Euler number
\newcommand{\eu}{\ensuremath{\mathrm{e}}}
% Imaginary unit
\newcommand{\iu}{\ensuremath{\mathrm{i}}}

% Load delimiters
% Delimiters

% Among other tools, define DeclarePairedDelimiter
\usepackage{mathtools}

% Absolute value |-|
\DeclarePairedDelimiter{\abs}{\lvert}{\rvert}
% Norm ||-||
\DeclarePairedDelimiter{\norm}{\lVert}{\rVert}

% For \Set, \bra, \ket and so on
\usepackage{braket}

% Custom parentheses
\DeclarePairedDelimiter{\roundB}{(}{)}
\DeclarePairedDelimiter{\squareB}{[}{]}
\DeclarePairedDelimiter{\curlyB}{\{}{\}}

% Mean <.>
\DeclarePairedDelimiter{\mean}{\langle}{\rangle}

% Load numbersets
% Numbersets

% To enable \mathbb
% Also enables \Cap, \Box, \Diamond and so on
\usepackage{amssymb} % loads amsfonts

% Numbersets ------------------------
\newcommand{\numberset}[1]{\ensuremath{\mathbb{#1}}}
\newcommand{\N}{\numberset{N}}
\newcommand{\Z}{\numberset{Z}}
\newcommand{\Q}{\numberset{Q}}
\newcommand{\R}{\numberset{R}}
\newcommand{\I}{\numberset{I}}
\newcommand{\C}{\numberset{C}}
% Euclidean space
\newcommand{\E}{\numberset{E}}
% -----------------------------------

% Empty set
\newcommand{\0}{\ensuremath{\varnothing}}
% Cartesian product
\newcommand{\x}{\ensuremath{\times}}

% Load operators
% Operators

% Differential operators ------------
\newcommand{\de}{\mathop{}\!\textup{$\partial$}}
\newcommand{\uD}{\mathop{}\!\textup{$\Delta$}}
\newcommand{\ud}{\mathop{}\!\textup{d}}
\newcommand{\V}{\mathop{}\!\nabla}
\newcommand{\VV}{\ensuremath{\V^2}}
% D'Alembert operator
\newcommand{\dal}{\ensuremath{\mathop{}\!\Box}}
% -----------------------------------

% Algebra ---------------------------
\DeclareMathOperator{\tr}{tr}
\DeclareMathOperator{\diag}{diag}
% -----------------------------------

% Statistics ------------------------
\DeclareMathOperator{\erf}{erf}
\DeclareMathOperator{\cov}{cov}
% -----------------------------------


% Vectors
\usepackage{bm}
\renewcommand{\vec}[1]{\ensuremath{\mathbf{{#1}}}}

\newcommand{\A}{\ensuremath{\mathcal{A}}}
\newcommand{\package}[1]{\textsf{#1}}

% Setup index
% COnfiguration file for the index

\usepackage{makeidx}
\usepackage{multicol}

% enable generation of index via '\printindex' command in the document environment
\makeindex

%%
\let\orgtheindex\theindex
\let\orgendtheindex\endtheindex
\def\theindex{%
	\def\twocolumn{\begin{multicols}{2}}%
	\def\onecolumn{}%
	\clearpage
	\orgtheindex
}
\def\endtheindex{%
	\end{multicols}%
	\orgendtheindex
}
%%%%%%%%%%%%%%%%%%%%%%%%%%%

% Setup bibliography
% Setup bibliography
\usepackage[
	backend=biber,
]{biblatex}
\usepackage{csquotes}
\addbibresource{backmatter/bibliography.bib}


% Define abstract environment for *book class
% Define the abstract environment in a book (or scrbook) documentclass
%\usepackage{fancyhdr}
%\newcommand{\fncyblank}{\fancyhf{}}
\newenvironment{abstract}%
{\cleardoublepage\thispagestyle{empty}\null\vfill\begin{center}%
\bfseries\abstractname\end{center}}%
{\vfill\null}

% Define acknowledgements environment
% XXX Load _after_ abstractConfig.tex!!
% Define the acknowledgements environment in a book (or scrbook) documentclass

% These are already loaded/defined in abstractConfig.tex
%\usepackage{fancyhdr}
%\newcommand{\fncyblank}{\fancyhf{}}

\newenvironment{acknowledgements}%
{\cleardoublepage\fncyblank\null\vfill\begin{center}%
\bfseries{Acknowledgements}\end{center}}%
{\vfill\null}


% Setup hyperref
% Setup hyperref
\usepackage{hyperref}
\hypersetup{
    %draft,
    final,
    %colorlinks=false,
    colorlinks=true,
    linktocpage=true,
    pdfstartpage=3,
    pdfstartview=FitV,
    breaklinks=true,
    pdfpagemode=UseNone,
    pageanchor=true,
    pdfpagemode=UseOutlines,%
    plainpages=false,
    bookmarksnumbered,
    bookmarksopen=true,
    bookmarksopenlevel=1,%
    hypertexnames=true,
    pdfhighlight=/O,
    urlcolor=webbrown,
    linkcolor=RoyalBlue,
    citecolor=webgreen,
%   pagecolor=RoyalBlue,%
}

% Setup acronyms
% XXX Load it _after_ hyperref, babel, polyglossia, inputenc and fontenc
% For acronyms (and glossaries)
\usepackage[
	makeindex,% use this to make gloxary
	acronym,%
	nomain,% suppress the main (default) glossary
	hyperfirst=false, % no hyperlink at first use of acronym
	toc,% add voice in toc
]{glossaries} % XXX Load it _after_ hyperref, babel, polyglossia, inputenc and fontenc

\usepackage{relsize} % defines \textsmaller{} used by the following style
\setacronymstyle{long-sm-short}

\makenoidxglossaries


% Define acronyms
\newacronym{cuda}{CUDA}{Compute Unified Device Architecture}


% Add this package to draw molecules (needed for the caffeine!)
\usepackage{chemfig}
\usepackage{mhchem}
\mhchemoptions{
    version=4,
    layout=stacked,   % stack superscript and subscript
    arrows=pgf-filled % set arrow (other options 'font', 'pgf')
}
\usepackage[
%    notitalic, % particle names are upright
%    maybess,  % allow sans serif if surrounding is sans serif
]{hepnames}
\usepackage{pgfplots}
\usepgfplotslibrary{fillbetween}
\usetikzlibrary{calc}
\tikzset{>=stealth,}
\usepackage{lipsum}
\usepackage{qrcode}

\title{}
\subtitle{}
\author{
  \input{AUTHORS}
}
\date{}

\begin{document}

%----------------------------------------------------------------------------------------
% FRONTMATTER
%
\frontmatter
\pdfbookmark[0]{Title}{chap:title}
	\maketitle
	\thispagestyle{empty}

	\cleardoublepage
\pdfbookmark[0]{Dedication}{chap:dedication}
	\thispagestyle{empty}
    \null\vspace{\stretch {1}}
        \begin{flushright}
                This is the Dedication.
        \end{flushright}
\vspace{\stretch{2}}\null


	\cleardoublepage
\pdfbookmark[0]{\abstractname}{chap:abstract}
	\begin{abstract}
    Partial-wave analysis is currently the standard analysis technique in the study of hadronic heavy-meson decays.
    In this context, to find an explicit form of the decay amplitude, most analyses exploit the isobar model, which assumes that the decay of the parent particle proceeds through subsequent two-body decays involving well-defined intermediate states.
    The success of the isobar model in providing a good description of the decay depends on the assumptions about these intermediate states.


    Due to the availability of increasingly large experimental data sets, the systematic uncertainty introduced by partial or incorrect knowledge of the intermediate states dominate the statistical uncertainty.
    A possible extension of the current analysis techniques is the model-independent approach to partial-wave analysis:
    It exploits the increase of the experimental data samples to get rid of unjustified assumptions of the isobar model, as the properties of the isobars are extracted from the data itself.


    Because of the large experimental data sets and the large number of fit parameters, partial-wave analyses involve expensive calculations.
    This motivates the development of \pacs{yap}, a novel toolkit for partial-wave analysis.


    Here, after introducing the partial-wave-analysis formalism, I describe the main features of \pacs{yap} and present my \pacs{yap}-based implementation of a model-independent partial-wave-analysis fit utility.
    I also show the test fits I performed on several \acs{mc} data sets of $\PDplus\to\Ppiplus\Ppiminus\Ppiplus$ decays with increasing number of waves.
    The \acs{mc} data generated according to the fitted parameters correctly reproduce the fit source data.

\end{abstract}


	\cleardoublepage
\pdfbookmark[0]{Acknowledgements}{chap:acknowledgements}
	%\chapter*{Aknowledgements}
%\phantomsection
%\addcontentsline{toc}{chapter}{Aknowledgements}


\begin{acknowledgements}
\dots{}

Last but not least I'd like to thank you, Caffeine, I would not have made it without you!

% Draw a Caffeine molecule
\setcrambond{2pt}{}{}
\setatomsep{2em}
\chemname{%
\chemfig{%
{\color{black!70}H_3C}-[:72]{\color{RoyalBlue}N}%
    *5(-%
	*6(-(={\color{Maroon}O})-{\color{RoyalBlue}N}(-{\color{black!70}CH_3})-(={\color{Maroon}O})-{\color{RoyalBlue}N}(-{\color{black!70}CH_3})-=)%
    --{\color{RoyalBlue}N}=-)}%
}{}

\end{acknowledgements}


	\cleardoublepage
\pdfbookmark[0]{\contentsname}{chap:contents}
	\tableofcontents

%----------------------------------------------------------------------------------------
% MAINMATTER
%
\mainmatter
\chapter{chap first}
\lipsum
%\pgfplotstabletypeset{data/dummyTableRaw.tab}
%\pgfplotstabletypeset[sci zerofill]{
%a b
%5000 1.234e5
%6000 1.631e5
%7000 2.1013e5
%9000 1000000
%}

\section{Test for a decay with \texttt{mchem}}

Now we prove that $\forall z \in \C$
\begin{equation*}
    \abs{z} = \sqrt{z_1^2 + z_2^2}
    \quad
    \norm{z} = \abs{z_1} + \abs{z_2}
\end{equation*}

\ce{D^0_{s} -> \pi^0$\cramped{\pi^0}$}

This is a mathrm delta: $\mathrm{\delta}$ and $\delta$ or $\mathrm{\Delta}$ and $\Delta$.

\begin{equation}
    \int_{\Omega} f(x)\ud x\ \mathrm{d}x
\end{equation}

{\sffamily this is some math $\ud x$.}

\begin{equation}
    \eu^{\iu\pi} + 1 = 0
\end{equation}

\begin{figure}
    \centering
\begin{tikzpicture}[
        particle/.style = {circle},
        M/.style        = {particle, ball color=black!20},
        a/.style        = {particle, ball color=blue!20},
        b/.style        = {particle, ball color=green!20},
        c/.style        = {particle, ball color=red!20},
    ]

    \node[M] (M) at (0,0) {$M$};

    % Draw the particles in the rest frame
    \node[a] (a) at ($(M) + (1,1.5)$) {$a$};
    \node[b] (b) at ($(M) + (2.5,-.9)$) {$b$};
    \node[c] (c) at ($3*(M) - (a) - (b)$) {$c$};

    % Connect the particles
    \draw[->] (M) -- node[midway,above,sloped] {$p_a$} (a);
    \draw[->] (M) -- node[midway,above,sloped] {$p_b$} (b);
    \draw[->] (M) -- node[midway,above,sloped] {$p_c$} (c);

\end{tikzpicture}

    \caption{Three-body decay in the rest frame of $M$, the parent particle.}
\end{figure}

\section{The fit fraction issue}

\begin{equation*}
    f_i =  \int_{\Omega}  % \abs{}% \alpha_i t_i} %\ud m
\end{equation*}


\begin{figure}
	\centering
	\qrcode[]{https://github.com/pdigiglio/MasterThesis}
\caption{The link to my repository on GitHub: \url{https://github.com/pdigiglio/MasterThesis}.}
\end{figure}


Let's put some index\index{index}
ANd now something which is related to the index\index{index!related}

%And the ultimate answer is \SI{34.89}{\angstrom}

\input{data/dummyTable.tab}



I wanna thanks the coffeine
%% Draw a Caffeine molecule
\setcrambond{2pt}{}{}
\setatomsep{2em}
\chemname{%
\chemfig{%
{\color{black!70}H_3C}-[:72]{\color{RoyalBlue}N}%
    *5(-%
	*6(-(={\color{Maroon}O})-{\color{RoyalBlue}N}(-{\color{black!70}CH_3})-(={\color{Maroon}O})-{\color{RoyalBlue}N}(-{\color{black!70}CH_3})-=)%
    --{\color{RoyalBlue}N}=-)}%
}{}


\chapter{Dalitz-plot analysis}

\begin{equation}
    \ud\Gamma = \frac{1}{(2\pi)^3}\frac{1}{32M^3}\abs{\A}\ud m_{ab}^2 \!\ud m_{bc}^2
\end{equation}
So any deviation from the uniform distribution of the decay count plot against these two masses comes from the dynamical amplitude \A{}.

\begin{figure}
    \centering
%\begin{tikzpicture}
%    \pgfmathsetmacro{\mTwoA}{(.13957)^2} % pi+ mass squared [GeV/c^2]
%    \pgfmathsetmacro{\mTwoB}{(.13957)^2} % pi- mass squared [GeV/c^2d
%    \pgfmathsetmacro{\mTwoC}{(.13957)^2} % pi+ mass squared [GeV/c^2d
%    \pgfmathsetmacro{\mTwo}{(1.86963)^2} % D+  mass squared [GeV/c^2d      

    \pgfmathsetmacro{\mTwoA}{(.17)^2}
    \pgfmathsetmacro{\mTwoB}{(.17)^2}
    \pgfmathsetmacro{\mTwoC}{(.17)^2}
    \pgfmathsetmacro{\mTwo}{(1.)^2}

    \pgfmathsetmacro{\lBound}{(sqrt(\mTwoA) + sqrt(\mTwoB))^2}
    \pgfmathsetmacro{\uBound}{(sqrt(\mTwo)  - sqrt(\mTwoC))^2}
    \pgfmathsetmacro{\lyBound}{(sqrt(\mTwoC) + sqrt(\mTwoB))^2}
    \pgfmathsetmacro{\uyBound}{(sqrt(\mTwo)  - sqrt(\mTwoA))^2}

    \begin{axis}[
        ylabel={$m_{ab}^2$},
        xlabel={$m_{bc}^2$},
        ytick={\lBound,\uBound},
        yticklabels={$(m_a + m_b)^2$, $(M-m_c)^2$},
        xtick={\lyBound,\uyBound},
        xticklabels={$(m_b + m_c)^2$, $(M-m_a)^2$},
        yticklabel style={sloped like y axis},
        grid = major,
        declare function = { E_b(\t) = ((\t -\mTwoB + \mTwoA)/(2*sqrt(\t))); },
        declare function = { P_b(\t) = sqrt(E_b(\t)^2 -\mTwoB); },
        declare function = { E_c(\t) = ((\mTwo -\t -\mTwoC)/(2*sqrt(\t))); },
        declare function = { P_c(\t) = sqrt(E_c(\t)^2 -\mTwoC); },
        enlargelimits=.12,
    ]

        \addplot[
            name path = A,
            black,
            opacity=0,
            domain = {\lBound:\uBound},
            samples=250,
        ] {(E_b(x) + E_c(x))^2 - (P_b(x) - P_c(x))^2};

        \addplot[
            name path = B,
            black,
            opacity=0,
            domain = {\lBound:\uBound},
            samples=250,
        ] {(E_b(x) + E_c(x))^2 - (P_b(x) + P_c(x))^2};

        \addplot[blue!50, opacity=.3] fill between[of=A and B];

    \end{axis}
\end{tikzpicture}

    \caption{Kinematically allowed region in the phase space of a three-body decay.}
\end{figure}


Now the problem is how to model the amplitude.
Past experiments have shown that nonleptonic three-body decays proceed through intermediate two-body resonant decays.
Thus, the amplitude is modeled as a coherent sum of two-body decays plus a non-resonant constribution:
\begin{equation}
    \A(m_{ab}^2, m_{bc}^2) = \sum_r a_r \eu^{\iu\phi_r}\A_r(m_{ab}^2,m_{bc}^2)
    + a_{\textup{NR}} \eu^{\iu \phi_{\textup{NR}}} \A_{\textup{NR}}(m_{ab}^2,m_{bc}^2)
\end{equation}

\section{Isobar formalism}

In the isobar formalism the ampliture that describes the decay through a resonance $r$ is modeled as
\begin{equation}
    \A_r = F_P\,F_r\, T_r\, W_r.
\end{equation}

The dynamical function is usually described by a relativistic Breit-Wigner with a mass-dependent width:
\begin{equation}
    T_r = \frac{1}{m_r^2 - m_{ab}^2 - \iu m_r \Gamma_{ab}}
\end{equation}

The angular distribution is described either by using Zemach tensors of by using the helicity formalism.


The form factors $F_P$ and $F_r$ usually use the Blatt-Weisskopf parametrization for the decay vertex.


The $K$-matrix is an alternative approach to the isobar formalism for the amplitude calculation.


Model independent \gls{pwa} and binned analysis are examples of model independent Dalitz-plot analysis.



Implementing model independent analysis tools in \gls{yap} could be a topic!

\section{Resonances}

The $S$-matrix is the unitary operator that connects asymptotic incoming and outgoing states.


\section{Isobar-freed \gls{pwa}}

The invariant-mass range is divided in a set of bin with an index $i \in \Set{ \text{bins} }$ and the binned functions are given by
\begin{equation}
    \Delta_i(m_X; m_a, m_b) = ( m_X \in \text{bin}_i) : 1\  ?\  0.
\end{equation}
So that the shape function of the isobar reads
\begin{equation}
    \Delta(m_X; m_a, m_b) = \sum_{i\in \Set{\text{bins}}} A_i \Delta_i(m_X;m_a, m_b).
\end{equation}

    \subsection{Zero modes}

Introduction of the isobar-freed fit leads to linear dipendencies among the waves in the fit model.
This can be seen by means of the integral matrix (now $w$ and $v$ are wave indices)
\begin{equation}
    I_{wv} \coloneqq \int_\Omega \psi_w^*(\tau)\,\psi_v(\tau)\ud\tau
\end{equation}
which represents the integrated overlap of the waves $w$ and $v$ over the phase space.
Since $I$ is hermitian, its eigenvalues are real.
So
\begin{equation}
    I^{\text{Tot}} \coloneqq \int_\Omega \abs{\A(\tau)}^2\!\ud \tau = \sum_{v,w \in \Set{\text{waves}}} T_w^* I_{wv} T_v.
\end{equation}

\section{Model-independent descriptions}

\subsection{Motivation}

Why should I want to use one?

I'm not required to guess what's in the resonance thus introducing biases by an incorrect or incomplete model.
The drawback is that the fit parameters increase a lot but this is not an issue due to the magnitude of the current data samples.


Who guarantees me that you're telling the truth and the description works?

The fit result obtained via a model-independent description have been compared to the isobar ones.
And they're compatible, you skeptical moron.


Let's quote some previous collaborations tha already used a model-independent description to analyze their decays, such as~\cite{PhysRevD.73.032004,Link200914}.
Now I'll write some text here to see if \texttt{hepnames} breaks the line $\PDplus\cramped{\PDplus} \to \PKminus\Ppiplus\Ppiplus$ $\cramped{\PDplus \to \PKminus\Ppiplus\Ppiplus}$.
If the author is really a clever person, he should have taken this into account, right?


The approach is flexible because it allows a mixed formalism.
The FOCUS collaboration, for instance, used it to describe the S-wave component of a $\PKminus\Ppiplus$ system from a $\PDplus \to \PKminus\Ppiplus\Ppiplus$ decay while still using a model dependent description for P and D waves~\cite{Link200914}.

\section{\acs{yap}}

We used~\cite{stl_meyers,effective_cpp_meyers},

\chapter{Outline}
    \section{Partial-wave analysis}
    The amplitude is decomposed in partial waves, that usually correspond to eigenvectors of an observable.
    In a three-body decay, for instance, one can write
    \begin{equation}\label{eq:pwa_ang_mom}
        \A = \sum_{L=0}^\infty a_L \A_L,
    \end{equation}
    where $L$ is an index for the angular momentum between the $ab$ system and the bachelor particle $c$.

    \subsection{Isobar model}
    In the isobar model only two-body decays are allowed.
    This means that the parent particle produces the bachelor particle $c$ and a resonance $r$.
    The resonance, in turn, will decay into the particles $a$ and $b$.


    This way, the~\eqref{eq:pwa_ang_mom} will be further decomposed into a coherent sum of the amplitudes that populate each wave:
    \begin{equation}
        \A_L = \sum_r a_L^r \A_L^r,
    \end{equation}
    being $r$ and index for the alloewd resonances.
    Resonances have to fulfill conservation laws at each decay vertex, just like particles.
    Resonances are intermediary states that cannot, by their nature, be observed.



    It's difficult to find a heuristic model for the S wave, since it is the most populated.
    For this, we implement a model-independent description in \ac{yap}.

        \subsubsection{How to model the amplitude of a resonance}
        Motivate the decomposition in $F_P F_r T_r W_r$.



    \subsection{Model-independent description}

    The model-independent description is used when the wave comprises of poorly understood resonances.
    It has already been employed to describe the S wave of the $\PKminus\Ppiplus$ system in the $\PDplus \to \PKminus\Ppiplus\Ppiplus$ decay by the Fermilab E791~\cite{PhysRevD.73.032004} and FOCUS~\cite{Link200914} collaborations.

%----------------------------------------------------------------------------------------
% BACKMATTER
%
\backmatter
	% Acronyms
	\printnoidxglossaries

	% Index
	\cleardoublepage
	\phantomsection
	\addcontentsline{toc}{chapter}{\indexname}
	\printindex

	% Bibliography
	\cleardoublepage
	\phantomsection
	\addcontentsline{toc}{chapter}{\bibname}
	\nocite{*}
	\printbibliography

\end{document}

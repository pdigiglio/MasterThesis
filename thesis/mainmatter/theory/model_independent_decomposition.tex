\subsection{Model-independent isobar decomposition}
\label{sec:model_independent_isobar_decomposition}

    The model-independent approach to the isobar decomposition aims to reduce the analysis' dependence on possibly incomplete or incorrect fit models, as it does not need any assumption on the resonant content of the isobars.
    In this approach, the dynamic shapes of the isobars are directly extracted from the available experimental events, provided the data set is large enough.
    

    A characteristic function\index{characteristic function} on a real interval, $M \coloneqq [m_{\text{low}}, m_{\text{up}})$, is defined as
    \begin{equation}
        \Characteristic_M(m) \coloneqq 
        \begin{cases}
            1 &\text{if }m \in M, \\
            0 &\text{otherwise}.
        \end{cases}
    \end{equation}
    Let $\mathcal{P}(M) \coloneqq \set{m_0, m_1,\dots, m_{N_\text{bins}}}$ be a partition of $M$---\ie~a set of points such that $m_{i} < m_{i+1}$; and whose extremes coincide with those of $M$, namely $(m_0, m_{N_\text{bins}}) = (m_{\text{low}}, m_{\text{up}})$.
    \begin{figure}
        \centering
        \begin{tikzpicture}
    \pgfmathtruncatemacro{\a}{3}
    \pgfmathtruncatemacro{\b}{1}
    \pgfmathtruncatemacro{\c}{2}

    \begin{scope}[color=black!50]
        \coordinate (o) at (-1, 0);
        \draw (o) -- +(\a + 2, 0) coordinate (a);
        \draw [densely dashed] (a) -- +(\b, 0) coordinate (b);
        \draw [->] (b) -- +(\c, 0) coordinate (c);
    \end{scope}

    \begin{scope}
        % vertical label shift
        \pgfmathsetmacro{\s}{-.5}

        \foreach \x in {0,1,...,\a} {
            \node [draw, inner sep=1pt, circle, fill=black] (x\x) at (\x, 0) {};
            \node (sx\x) at ($(x\x) + (0, \s)$)  {$m_\x$};
        }

        \node [draw, inner sep=1pt, circle, fill=black] (last) at (\a + \b + 2, 0) {};
        \node (slast) at ($(last) + (0, \s)$) {$m_{N_\text{bins}}$};

    \end{scope}

    \node () at (x0) [label={above:$m_a + m_b$}] {};
    \node () at (last) [label={above:$m_X - m_c$}] {};


\end{tikzpicture}

        \caption[Partition of the invariant-mass range into $N_\text{bins}$ right-open bins.]%
        {Partition of the invariant-mass range, $M \coloneqq [m_a+m_b,m_X - m_c)$, into $N_\text{bins}$ right-open bins, $B_i = [m_{i-1}, m_i)$.}
        \label{fig:invariant-mass-partition}
    \end{figure}
    Figure~\ref{fig:invariant-mass-partition} shows the partitioning of the interval $[m_a+m_b,m_X-m_c)$, the allowed invariant-mass range for the final-state $ab$ system.\footnote{The allowed mass range of the $ab$ system also includes the value $m_X - m_c$; I will anyway ignore this value as it is a null-measure interval, so leaving it out will not affect the results of the analysis.}
    The partition $\mathcal{P}(M)$ defines $N_\text{bins}$ bins on $M$, $\set{B_i}$, where $B_i$ is the right-open interval $[m_{i-1},m_i)$ for $i\in\set{1,\dots, N_\text{bins}}$.
    A step function\index{step function} on the interval $M$, given the partition $\mathcal{P}(M)$ is a finite linear combination of characteristic functions on the bins:
    \begin{equation}\label{eq:step_dynamic_shape}
        \Delta_{\mathcal{P}(M)}^{\text{SF}}(m) \coloneqq \sum_{i = 1}^{N_\text{bins}} \alpha_i\, \Characteristic_{B_i}\!(m),
    \end{equation}
    $\alpha_i$ being a complex coefficient.


    In the model-independent isobar decomposition, equation~\eqref{eq:step_dynamic_shape} is the approximate dynamic shape of the isobar, and its coefficients are determined from the analysis of the observed events.
    \begin{figure}
        \centering
        \begin{tikzpicture}
    \begin{axis}[
        amplitude_plot,
        width = .8\textwidth,
        xlabel={$s$ $[\si{\giga\electronvolt^2\per\c^4}]$},
    ]
        \pgfmathsetmacro{\m}{.98}
        \pgfmathsetmacro{\T}{.07}
        \addplot+ [guess, domain={.279:1.73}, samples=100, smooth] gnuplot {1./sqrt((\m * \m - x * x)**2 + (\m * \T)**2)};
        \addplot+ [fit, mark=none] table {data/binned_breit-wigner.txt};
    \end{axis}
\end{tikzpicture}

        \caption{Step-function approximation of the magnitude of the non-relativistic Breit-Wigner dynamic shape, equation~\eqref{eq:bw}.}
        \label{fig:step_function_approximation}
    \end{figure}
    Figure~\ref{fig:step_function_approximation} shows the basic idea behind the model-independent isobar decomposition: the piecewise function in equation~\eqref{eq:step_dynamic_shape} will approximate the isobar mass shape.


    With the piecewise dynamic shape, the isobar amplitude reads
    \begin{equation}\label{eq:freed_wave_amplitude}
        \Psi_I(\tau) = \psi_{J_I\!}(\tau)\,\mathcal{F\!}_{J_I\!}(\tau)\,\A_{d_1}\!(m_{d_1\!}; \tau)\,\A_{d_2}\!(m_{d_2\!}; \tau) \sum_{i=1}^{N_\text{bins}} \gamma_{(I,i)}\, \Characteristic_{B_i}\!(s),
        \quad
        \text{with }
        \gamma_{(I,i)} \coloneqq c_I \alpha_i. 
    \end{equation}
    The free amplitudes, $\gamma_{(I,i)}$, now need to have an isobar label too.
    I will call a wave whose decay amplitude is written in the form of equation~\eqref{eq:freed_wave_amplitude} \emph{freed wave}\index{freed wave}.


    At this point it is worth noting that, when fitting the decay model to the data, the piecewise description of the dynamic shape introduces $N_\text{bins}$ complex \acp{dof}---for a total of $2N_{\text{bins}}$ real \acp{dof}.
    This drastically increases the computational complexity of the fit.


    To contain the number of fit \acp{dof}, it is possible to mix the model-dependent and model-independent \ac{pwa} approaches: the isobars with well-known resonances may be described in a fixed way, while the other isobars may be freed.
    The \focus{} collaboration performed a \ac{pwa} of the $\PDplus \to \PKminus\Ppiplus\Ppiplus$ decay employing a fixed description of the P- and D-wave components of the $\PKminus\Ppiplus$ decay amplitude (described as a sum of Breit-Wigner distributions), and a model-independent description of the S-wave component~\cite{Link200914}.


    In chapter~\ref{chap:model_independent_pwa}, I exploit the information about the resonant content of the \ac{mc} data to make the bins finer in the surroundings of the resonance peaks.
    A non-uniform optimized partitioning of the mass range speeds up the fit convergence.
    Unfortunately, the information about a wave's resonant content is not always available.


        \subsubsection{Zero modes}
        Introduction of the isobar-freed fit leads to linear dipendencies among the waves in the fit model.
This can be seen by means of the integral matrix (now $w$ and $v$ are wave indices)
\begin{equation}
    I_{wv} \coloneqq \int_\Omega \psi_w^*(\tau)\,\psi_v(\tau)\ud\tau
\end{equation}
which represents the integrated overlap of the waves $w$ and $v$ over the phase space.
Since $I$ is hermitian, its eigenvalues are real.
So
\begin{equation}
    I^{\text{Tot}} \coloneqq \int_\Omega \abs{\A(\tau)}^2\!\ud \tau = \sum_{v,w \in \Set{\text{waves}}} T_w^* I_{wv} T_v.
\end{equation}


\section{Dalitz-plot analysis}

    In a two-body decay, the energy-momentum conservation fully determines the energy of the final-state particles in the center-of-mass frame.
    In a three-body decay, after requiring energy and momentum conservation, there are still five remaining \acp{dof}.
    If the initial- and final-state particles are spinless, up to an arbitrary rotation of the decay plane, two \acp{dof} remain.
    Therefore, for the decay under consideration, the phase-space coordinate, $\tau$, is a two-element vector.

   
   There are several possible choices for the observable pair to parametrize $\tau$.
    It is thus convenient to choose a couple of observables such that the phase-space measure is constant within the kinematically-allowed region they span.
    Two such observables are the squared invariant masses of two pairs of final-state particles, i.e.~$m_{ab}^2 = p_{ab}^2$ and $m_{bc}^2 = p_{bc}^2$, with $a$, $b$, and $c$ the labels for the final-state particles (see figure~\ref{fig:isobar_three_body_decay}).


    With this parametrization of the phase-space coordinate, the differential decay probability is~\cite[\S~13.1.1]{Bevan:2014iga}
    \begin{equation}
        \ud\Gamma = \frac{1}{(2\pi)^3}\frac{1}{32 m_X^3}\abs{\A(m_{ab}^2, m_{bc}^2)}^2\!
        \ud m_{ab}^2 \!\ud m_{bc}^2.
    \end{equation}
    As the phase-space measure is a constant function throughout the whole phase space, any non-uniformity in the distribution of the events is to be ascribed to the decay amplitude only.
    \begin{figure}
        \centering
        \begin{tikzpicture}
    \begin{axis} [
        width = .7\textwidth,
        view={0}{90},%
        xlabel={$m_{bc}^2$},
        ylabel={$m_{ab}^2$},
        ticks=none,
        grid=none
    ]
        \addplot3 [dalitz]
          gnuplot [raw gnuplot] { splot "data/resonance_example/s_wave_resonance_mcmc.txt" using 1:2:3 };
    \end{axis}
\end{tikzpicture}

        \caption{S-wave resonance in the $ab$ channel.}
        \label{fig:s_wave_resonance_example}
    \end{figure}
    \begin{figure}
        \centering
        \begin{tikzpicture}
    \begin{axis} [
        width = .7\textwidth,
        view={0}{90},%
        xlabel={$m_{bc}^2$},
        ylabel={$m_{ab}^2$},
        ticks=none,
        grid=none
    ]
        \addplot3 [dalitz]
          gnuplot [raw gnuplot] { splot "data/resonance_example/p_wave_resonance_mcmc.txt" using 1:2:3 };
    \end{axis}
\end{tikzpicture}

        \caption{P-wave resonance in the $ab$ channel.}
        \label{fig:p_wave_resonance_example}
    \end{figure}
    \begin{figure}
        \centering
        \begin{tikzpicture}
    \begin{axis} [
        width = .7\textwidth,
        view={0}{90},%
        xlabel={$m_{bc}^2$},
        ylabel={$m_{ab}^2$},
        ticks=none,
        grid=none
    ]
        \addplot3 [dalitz]
          gnuplot [raw gnuplot] { splot "data/resonance_example/d_wave_resonance_mcmc.txt" using 1:2:3 };
    \end{axis}
\end{tikzpicture}

        \caption{D-wave resonance in the $ab$ channel.}
        \label{fig:d_wave_resonance_example}
    \end{figure}
    The resonances appear as bands within the allowed region, as shown in Dalitz plots~\ref{fig:s_wave_resonance_example}, \ref{fig:p_wave_resonance_example}, and \ref{fig:d_wave_resonance_example}.


    \begin{figure}
        \centering
        \begin{tikzpicture}
%    \pgfmathsetmacro{\mTwoA}{(.13957)^2} % pi+ mass squared [GeV/c^2]
%    \pgfmathsetmacro{\mTwoB}{(.13957)^2} % pi- mass squared [GeV/c^2d
%    \pgfmathsetmacro{\mTwoC}{(.13957)^2} % pi+ mass squared [GeV/c^2d
%    \pgfmathsetmacro{\mTwo}{(1.86963)^2} % D+  mass squared [GeV/c^2d      

    \pgfmathsetmacro{\mTwoA}{(.17)^2}
    \pgfmathsetmacro{\mTwoB}{(.17)^2}
    \pgfmathsetmacro{\mTwoC}{(.17)^2}
    \pgfmathsetmacro{\mTwo}{(1.)^2}

    \pgfmathsetmacro{\lBound}{(sqrt(\mTwoA) + sqrt(\mTwoB))^2}
    \pgfmathsetmacro{\uBound}{(sqrt(\mTwo)  - sqrt(\mTwoC))^2}
    \pgfmathsetmacro{\lyBound}{(sqrt(\mTwoC) + sqrt(\mTwoB))^2}
    \pgfmathsetmacro{\uyBound}{(sqrt(\mTwo)  - sqrt(\mTwoA))^2}

    \begin{axis}[
        ylabel={$m_{ab}^2$},
        xlabel={$m_{bc}^2$},
        ytick={\lBound,\uBound},
        yticklabels={$(m_a + m_b)^2$, $(M-m_c)^2$},
        xtick={\lyBound,\uyBound},
        xticklabels={$(m_b + m_c)^2$, $(M-m_a)^2$},
        yticklabel style={sloped like y axis},
        grid = major,
        declare function = { E_b(\t) = ((\t -\mTwoB + \mTwoA)/(2*sqrt(\t))); },
        declare function = { P_b(\t) = sqrt(E_b(\t)^2 -\mTwoB); },
        declare function = { E_c(\t) = ((\mTwo -\t -\mTwoC)/(2*sqrt(\t))); },
        declare function = { P_c(\t) = sqrt(E_c(\t)^2 -\mTwoC); },
        enlargelimits=.12,
    ]

        \addplot[
            name path = A,
            black,
            opacity=0,
            domain = {\lBound:\uBound},
            samples=250,
        ] {(E_b(x) + E_c(x))^2 - (P_b(x) - P_c(x))^2};

        \addplot[
            name path = B,
            black,
            opacity=0,
            domain = {\lBound:\uBound},
            samples=250,
        ] {(E_b(x) + E_c(x))^2 - (P_b(x) + P_c(x))^2};

        \addplot[blue!50, opacity=.3] fill between[of=A and B];

    \end{axis}
\end{tikzpicture}

        \caption{Kinematically-allowed region in the phase space of a three-body decay. The corners correspond to the configuration in which one of the daughter particles is produced at rest (in the frame of the parent particle).}
        \label{fig:dalitz_kinematically_allowed}
    \end{figure}
    The boundaries of the kinematically-allowed phase-space region, shown in figure~\ref{fig:dalitz_kinematically_allowed}, can be expressed in terms of a constraint on $m_{bc}^2$ as follows:
    \begin{equation}
        \begin{aligned}
            (m_{bc}^2)_{\text{max}} &= (E_b + E_c)^2 - (\abs{\vec{p}_b} - \abs{\vec{p}_c})^2,\\
            (m_{bc}^2)_{\text{min}} &= (E_b + E_c)^2 - (\abs{\vec{p}_b} + \abs{\vec{p}_c})^2,
        \end{aligned}
    \end{equation}
    where
    \begin{equation}
        E_b = \frac{m_{ab}^2 - m_a^2 + m_b^2}{2m_{ab}},\text{ and }
        E_c = \frac{m_X^2 - m_{ab}^2 - m_c^2}{2 m_{ab}}
    \end{equation}
    are the energies of the $b$ and $c$ final-state particles in the rest frame of the $ab$ system, and
    \begin{equation}
        \abs{\vec{p}_b} = \sqrt{E_b^2 - m_b^2},\text{ and }
        \abs{\vec{p}_c} = \sqrt{E_c^2 - m_c^2}
    \end{equation}
    the corresponding 3-momenta magnitudes.
    The shape of the phase space, in figure~\ref{fig:dalitz_kinematically_allowed}, depends on the values of the final-state-particle masses:
    if $m_a = m_b = m_c = 0$, it will stretch to a triangle; whereas, if $m_a + m_b + m_c = m_X$, it will shrink to a point---thus making the decay impossible.

    \begin{figure}
        \centering

        \subfloat[]%
                 [P wave.]%
                 {\begin{tikzpicture}
    \begin{axis} [dalitz_plot]

        \addplot3 [dalitz]
          gnuplot [raw gnuplot] { splot "data/angular_part/uniform_p_wave_mcmc.txt" using 1:2:3 };
    \end{axis}
\end{tikzpicture}
}

        \subfloat[]%
                 [D wave.]%
                 {\begin{tikzpicture}
    \begin{axis} [dalitz_plot]

        \addplot3 [dalitz]
          gnuplot [raw gnuplot] { splot "data/angular_part/uniform_d_wave_mcmc.txt" using 1:2:3 };
    \end{axis}
\end{tikzpicture}
}

        \caption{Angular part of the decay amplitude (spin part in the Zemach formalism times Blatt-Weisskopf factor) for the P and D waves.}
        \label{fig:dalitz_angular_parts}
    \end{figure}

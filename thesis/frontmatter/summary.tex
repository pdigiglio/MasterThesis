\begin{abstract}
    \Acf{pwa} is currently the standard analysis technique in the study of hadronic heavy-meson decays into multi-body final states.
    Such decays lead to the discovery of new mesons, and are a great source of information for precise measurements of charge-conjugation and parity violation in experiments like Belle and \acs{lhcb}.


    To find an explicit form of the decay amplitude, most analyses exploit the isobar model, which assumes that the decay of the parent particle proceeds through subsequent two-body decays involving well-defined intermediate states.
    Isobars can contain one or several intermediate states.
    The success of the isobar model in providing a good description of the decay depends on the assumptions about the intermediate states.


    Because of the availability of increasingly large experimental data sets, the systematic uncertainty introduced by partial or incorrect knowledge of the intermediate states dominate the statistical uncertainty.
    This motivates extensions of the analysis techniques beyond the isobar model.
    One of such possible extensions is the model-independent approach to \acs{pwa}:
    It exploits the increase of the experimental data samples to get rid of the unjustified assumptions of the isobar model.
    In this approach, the properties of an isobar are obtained by replacing the only part of its decay amplitude that cannot be derived from first principles, the dynamic shape, by a step-like function whose properties have to be extracted by a fit to the data themselves.
    We call such an isobar, for which we drop the assumptions about the intermediate states, ``freed isobar''.



    \Acsp{pwa} require expensive calculations, due to the large experimental data sets and the large number of parameters involved in the fits.
    This motivates the development of \pacl{yap}~(\pacs{yap}), a novel toolkit for \acs{pwa}.
    With respect to similar toolkits, \pacs{yap} aims at being computationally efficient, through its smart-caching feature and built-in parallelization support;
    physically correct, being extensively tested;
    physicist friendly, with an interface that currently handles three- and four-particle final states, and various \acs{pwa} formalisms;
    and human readable, with a structure that highlights the physics of the decays.
    So far, \pacs{yap} does not handle the model-independent \acs{pwa} formalism.


    Here, after introducing the \acs{pwa} formalism, I describe the main features of \pacs{yap} and present my \pacs{yap}-based implementation of a model-independent \acs{pwa} fit utility.
    I show the test fits I performed on several \acs{mc} data sets of a show-case decay channel, $\PDplus\to\Ppiplus\Ppiminus\Ppiplus$, with increasing number of intermediate states and waves.
    Some ambiguities, the zero modes, may arise in the model-independent fits, which do not allow the reconstruction of the wave content, when more than one isobar is freed.
    In this case, the knowledge of the zero modes and a two-step fit are needed to extract the wave content from the data.
    The fit utility can already precisely extract the wave content from the data, when one isobar only is freed or when there is no zero mode among the freed isobars.
    The \acs{mc} data generated according to the fitted parameters always correctly reproduce the fit source data.


    This study shows that \pacs{yap} is suited for model-independent \acsp{pwa}.
    A generalization of the fit utility developed in this thesis, which also handles zero-mode ambiguities, will be included in \pacs{yap}.
\end{abstract}

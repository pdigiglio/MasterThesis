\chapter{Introduction}


    \paragraph{Some physical motivation}

    Heavy-meson decays are particularly useful to look for charge-conjugation an parity violation.


    Due to the availability of larger and larger data sets, the systematic uncertainties due to the \ac{pwa}---so far, the standard analysis technique for light- and heavy-meson decays---are dominating the statistical uncertainties.
    One of the greatest source of bias in the \ac{pwa} is the incorrect assumptions about the resonant content of the waves, \ie~wrong assumptions on the decay model.
    This motivates the development of a novel analysis approach, the \ac{mipwa}, which exploits the increased size of the data to get rid of unjustified model assumptions.
    

    Il mio studio si inserisce all'interno di un progetto più generale del dipartimento di fisica, che sta cercando di formalizzare il \ac{mipwa} (zero waves).


    \Ac{pwa} is a common tools to analyze data.


    Also, through the Dalitz-plot analysis one can:
    \begin{itemize}
        \item Searches for new states;
        \item Measuging resonance properties;
        \item $CP$ violation;
        \item B and D mixing;
        \item Resolving ambiguities.
    \end{itemize}

    A toolkit for \ac{pwa} is \pac{yap}.

    On top of \pac{yap} and \pac{bat} I have built a utility that implements a model-independent fit for analysis of $\PDplus\to\Ppiplus\Ppiminus\Ppiplus$ decays.

    \paragraph{Why model-independent analysis}

    Due to the availability of larger and larger data sets, the limiting factor in the data analysis is the quality of the model.


    Model-independent analysis can be useful to solve ambiguity like:

    ovelapping states

    not well-known resonances.

    \paragraph{goals of the thesis}

    I have contributed to the development of \pac{yap}, a novel toolkit for \ac{pwa}, briefly described in the chapter~\ref{chap:yap}.
    I have developed a fitting routine based on \pac{yap} that implements a model-independent fit.
    I have tested the routine on a variety of \ac{mc} data with increasing complexity (see the chapter~\ref{chap:model_independent_pwa}).





    The aim of this thesis is the description of \pac{yap}, a novel \ac{pwa} toolkit that I have contributed to develop.
    I also show the results of some model-independent fits performed using a self-developed piece of software relying on the external libraries \pac{yap} and \pac{bat}.



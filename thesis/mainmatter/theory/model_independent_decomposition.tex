\subsection{Model-independent isobar decomposition}
\label{sec:model_independent_isobar_decomposition}

    The model-independent approach to the isobar decomposition aims to reduce the analysis dependence on possibly incomplete fit models, as no assumption on the resonant content of the isobars is needed.
    The shapes of the resonances are directly extracted from the available experimental events, provided the data set is large enough.
    

    A characteristic function\index{characteristic function} on an real interval $R \coloneqq [a,b)$ is defined as
    \begin{equation}
        \Characteristic_R(m) \coloneqq 
        \begin{cases}
            1 &\text{if }m \in R, \\
            0 &\text{otherwise}.
        \end{cases}
    \end{equation}
    Let $\mathcal{P}(R) \coloneqq \set{m_0, m_1,\dots, m_{N_\text{bins}}}$ be a partition of $R$, \ie~a set of points such that $m_{i} < m_{i+1}$; and whose extremes coincide with those of $R$, namely $(m_0, m_{N_\text{bins}}) = (a, b)$.
    \begin{figure}
        \centering
        \begin{tikzpicture}
    \pgfmathtruncatemacro{\a}{3}
    \pgfmathtruncatemacro{\b}{1}
    \pgfmathtruncatemacro{\c}{2}

    \begin{scope}[color=black!50]
        \coordinate (o) at (-1, 0);
        \draw (o) -- +(\a + 2, 0) coordinate (a);
        \draw [densely dashed] (a) -- +(\b, 0) coordinate (b);
        \draw [->] (b) -- +(\c, 0) coordinate (c);
    \end{scope}

    \begin{scope}
        % vertical label shift
        \pgfmathsetmacro{\s}{-.5}

        \foreach \x in {0,1,...,\a} {
            \node [draw, inner sep=1pt, circle, fill=black] (x\x) at (\x, 0) {};
            \node (sx\x) at ($(x\x) + (0, \s)$)  {$m_\x$};
        }

        \node [draw, inner sep=1pt, circle, fill=black] (last) at (\a + \b + 2, 0) {};
        \node (slast) at ($(last) + (0, \s)$) {$m_{N_\text{bins}}$};

    \end{scope}

    \node () at (x0) [label={above:$m_a + m_b$}] {};
    \node () at (last) [label={above:$m_X - m_c$}] {};


\end{tikzpicture}

        \caption{Partition of the invariant-mass range, $R \coloneqq [m_a+m_b,m_X - m_c)$, into $N_\text{bins}$ right-open bins, $R_i = [m_i, m_{i+1})$.}
        \label{fig:invariant-mass-partition}
    \end{figure}
    The figure~\ref{fig:invariant-mass-partition} shows the partitioning of the interval $[m_a+m_b,m_X-m_c)$, the allowed invariant-mass range for the final state decay of the $ab$ system.\footnote{The $ab$ allowed mass range also includes the value $m_X - m_c$; I will anyway ignore this value as it is a null-measure interval, so leaving it out will not affect the results.}
    The partitioning $\mathcal{P}(R)$ defines $N_\text{bins}$ bins on $R$, $\set{B_i}$, where $B_i$ is the right-open interval $[m_{i-1},m_i)$ for $i\in\set{1,\dots, N_\text{bins}}$.
    A step function\index{step function} on the interval $R$, given the partition $\mathcal{P}(R)$ is a linear combination of characteristic functions on the bins:
    \begin{equation}\label{eq:step_dynamic_shape}
        \Delta_{\mathcal{P}(R)}^{\text{SF}}(m) \coloneqq \sum_{i = 1}^{N_\text{bins}} \alpha_i\, \Characteristic_{B_i}\!(m),
    \end{equation}
    being $a_i$'s complex parameters.


    In the model-independent isobar decomposition, the equation~\eqref{eq:step_dynamic_shape} is used as the dynamic shape of the isobar.
    The parameters are to be determined from an analysis of the observed events.
    \begin{figure}
        \centering
            \begin{tikzpicture}
                \begin{axis}[
                    amplitude_plot,
                    xlabel={$s$ $[\si{\giga\electronvolt^2\per\c^4}]$},
                ]
                    \pgfmathsetmacro{\m}{.98}
                    \pgfmathsetmacro{\T}{.07}
                    \addplot+ [guess, domain={.279:1.73}, samples=100, smooth] gnuplot {1./sqrt((\m * \m - x * x)**2 + (\m * \T)**2)};
                    \addplot+ [fit, mark=none] table {data/binned_breit-wigner.txt};
                \end{axis}
            \end{tikzpicture}
        \caption{Step-function approximation of the magnitude of the non-relativistic Breit-Wigner dynamic shape~\eqref{eq:bw}.}
        \label{fig:step_function_approximation}
    \end{figure}
    The figure~\ref{fig:step_function_approximation} shows the basic idea of the model-independent isobar decomposition: the step function~\eqref{eq:step_dynamic_shape} will approximate the isobar mass shape.


    The isobar amplitude now reads:
    \begin{equation}
        \Psi_I(\tau) = \psi_{J_I}\!(\tau) \sum_{i=1}^{N_\text{bins}} \gamma_{(I,i)}\, \Characteristic_{B_i}\!(s),
        \quad
        \text{being }
        \gamma_{(I,i)} \coloneqq c_I \alpha_i. 
    \end{equation}
    The free amplitudes, $\gamma_{(I,i)}$, now need to have an isobar label.
    This is called \emph{freed wave}\index{freed wave}.


    A straightforward drawback of the model-independent approach is the drastical increase of the fit complexity.
    Each bin introduces a complex fit parameter, for a total of $2N_{\text{bins}}$ real fit parameters.


    The model-independent formalism can easily be used together with the model-dependent one, to make the fits faster.

    At this point it is worth noting that this step-like description of the dynamic function of an isobar introduces $N_\text{bins}$ fit parameters, which drastically increases the complexity of fitting the model to the data.
    This problem can be partially solved considering that:
    \begin{itemize}
        \item 
            This formalism allows mixing fixed and freed resonance dynamic functions in a model, so that the number of free fit parameters can be somehow contained;
        \item
            The partition bins do not have to be uniform. By means of a non-uniform mass-range partition, making narrower bins in the mass interval where a resonance is expected, the number of necessary fit parameters can be optimized.
    \end{itemize}

    {\color{red}
    (Quote Fabian's work as unpublished?)
    }

    {\color{red}
    I will construct a step function\index{step function} on the kinematically-allowed phase-space mass range for the final state $ab$ by a linear combination of characteristic functions.
    The final-state particles $a$ and $b$ can decay in states whose mass, $m_{ab}$, lies in the interval $[m_a + m_b, m_X - m_c]$.
    Let $\set{B_i}$ be a set of $N_\text{bins}$ disjoint intervals of the final-particle mass range, such that their union covers it:
    \begin{equation}
        \set{m_0,m_1,\dots,m_N},\ 
        m_i < m_{i+1}
    \end{equation}

    Referring to the labeling scheme showed in the figure~\ref{fig:invariant-mass-partition}, let $B_i \coloneqq [m_{i-1}, m_i)$ be the $i$-th right-open mass range, where $i$ ranges from $1$ to $N_\text{bins}$.

    A finite linear combination of characteristic functions will result in a step function\index{step function}:
    \begin{equation}
        \Delta
    \end{equation}

    The unknown dynamic shape of the isobar can be modele

    In the freed-isobar \ac{pwa}, a set of binned function is used to describe the unknown shape of an isobar amplitude.
    This way, there is no need to use any knowledge on the isobar from past experiment.
    The bin functions are defined as follows:
    \begin{equation}
        \Delta_i(m_X; m_a, m_b) =
        \begin{cases}
            1 &\text{if}\ m_X \in B_i, \\
            0 &\text{otherwise}.
        \end{cases}
    \end{equation}
    The dynamical function of the isobar can now be constructed as a linear combination of the bin functions:
    \begin{equation}
        \Delta(m_X; m_a, m_b) = \sum_{i=1}^{N_\text{bins}} a_i \Delta_i(m_X;m_a, m_b),
    \end{equation}
    which is a step function depending on the complex parameters $\Set{a_1, \dots, a_{N_\text{bins}}}$.
    The unknown parameters $\Set{ a_i }$ will be determined from a fit of the decay model to the experimental data.
    }


    At this point it is worth noting that this step-like description of the dynamic function of an isobar introduces $N_\text{bins}$ fit parameters, which drastically increases the complexity of fitting the model to the data.
    This problem can be partially solved considering that:
    \begin{itemize}
        \item 
            This formalism allows mixing fixed and freed resonance dynamic functions in a model, so that the number of free fit parameters can be somehow contained;
        \item
            The partition bins do not have to be uniform. By means of a non-uniform mass-range partition, making narrower bins in the mass interval where a resonance is expected, the number of necessary fit parameters can be optimized.
    \end{itemize}

    {\color{red}
    (Quote Fabian's work as unpublished?)
    }


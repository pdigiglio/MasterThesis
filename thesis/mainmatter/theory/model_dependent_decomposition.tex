\subsection{Model-dependent isobar decomposition}



    In the model-dependent approach to the isobar decomposition, the dynamic shape of the decay amplitude is expanded in terms of the resonances that populate the isobar:
    \begin{equation}\label{eq:isobar_mass_shape_expansion}
        \Delta_I(s) = \sum_{\xi\in I} \alpha_{\xi}\Delta_{\xi}(s),
    \end{equation}
    being $s$ the squared mass of the resonance, \ie~the squared four-momentum of the resonance daughter-particle system; referring to the figure~\ref{fig:isobar_three_body_decay}, $s\coloneqq p_{ab}^2 = p_{\xi}^2$.
    Each parameter $\alpha_{\xi}$ is the complex weight of the corresponding resonance $\xi$.


    \begin{table}
        \centering
        \caption{Expressions of the Blatt-Wei\ss{}kopf penetration factors for the first three integer values of $J$. To simplify the notation, I define $z\coloneqq R^2 q_{ab}^2$.}
        \label{table:blatt_weisskopf}
        \begin{tabular}{lc}
            \toprule
            $J$ &$\mathcal{F}_{\!J}$\\
            \midrule
            $0$ &$1$ \\
            $1$ &$\bigg(\displaystyle\frac{2z}{z + 1}\bigg)^{1/2}$ \\
            $2$ &$\bigg(\displaystyle\frac{13 z^2 }{z^2 + 3z + 9}\bigg)^{1/2}$ \\
            \bottomrule
        \end{tabular}
    \end{table}
    One of the most common forms for the dynamic shape of a resonance is the relativistic Breit-Wigner function\index{relativistic Breit-Wigner}\index{Breit-Wigner!relativistic}
    \begin{equation}\label{eq:rbw}
        \Delta_{\xi}^{\text{RBW}}(s) \coloneqq \frac{1}{m_{\xi}^2 - s - \iu \sqrt{s}\, \Gamma_\xi(s)},
    \end{equation}
    being $m_{\xi}$ the nominal mass of the resonance.
    The mass-dependent width reads
    \begin{equation}
        \Gamma_\xi(s) \coloneqq \Gamma_{\xi} \, \mathcal{F}_{\!J_\xi}\!(q_{ab};R)\, \frac{m_{\xi}}{\sqrt{s}} \bigg(\frac{q_{ab}}{q_{\xi}}\bigg)^{2J_{\xi}+1},
    \end{equation}
    where $\Gamma_{\xi}$ and $J_{\xi}$ are the width and spin of the resonance;
    $\mathcal{F}$ is the Blatt-Wei\ss{}kopf form factor (see the table~\ref{table:blatt_weisskopf}), which depends on $R$, the \emph{radial size}\index{radial size};
    $q_{ab}$ is the measured break-up momentum;
    and $q_\xi$ is the break-up momentum at the nominal mass of $\xi$.


    If the resonance is narrow and all the relevant thresholds are far away, the term $\sqrt{s}\,\Gamma_\xi(s)$ in the denominator of the equation~\eqref{eq:rbw} can be replaced with the constant quantity $m_{\xi}\Gamma_{\xi}$~\cite[\S~47.2.1]{chinese_phisics}.
    With this substitution, one gets the non-relativistic Breit-Wigner\index{Breit-Wigner}
    \begin{equation}\label{eq:bw}
        \Delta_{\xi}^{\text{BW}}(s) \coloneqq \frac{1}{m_{\xi}^2 - s - \iu m_{\xi} \Gamma_{\xi}},
    \end{equation}
    which normalized such that $\Delta_{\xi}^{\text{BW}}(m_{\xi}^2) = \iu / m_\xi\Gamma_\xi$.


    Another possible dynamic shape is the pole-mass distribution
    \begin{equation}
        \Delta_{\xi}^{\text{PM}}(s) \coloneqq \frac{1}{m_{\xi}^2 - s},
    \end{equation}
    in which the nominal mass of the resonance is a complex number $m_\xi = \Re (m_\xi) + \iu\Im(m_\xi)$.
    \citeauthor{PhysRevD.71.054030}, for example, proposes the pole-mass shape as a parametrization for the \Psigma{} resonance~\cite{PhysRevD.71.054030}.
    \begin{figure}
        \centering
        \begin{tikzpicture}
    \begin{axis}[
            ylabel={Magnitude $[\si{1/(\giga\electronvolt\per\c^2)^2}]$},
            xlabel={$s$ $[\si{\giga\electronvolt^2\per\c^4}]$},
            legend style={at={(1.1,.9)}}
        ]
        \pgfmathsetmacro{\mPi}{.139}
        \pgfmathsetmacro{\R}{3.}
        
        \pgfmathsetmacro{\m}{.98}
        \pgfmathsetmacro{\T}{.07}

        \addplot [domain={.279:1.73}, samples=100, smooth, densely dashed]
                 gnuplot {1./sqrt((\m * \m - x * x)**2 + (\m * \T)**2)};\addlegendentry{Breit-Wigner}

        \addplot [domain={.279:1.73}, smooth, densely dotted]
                 gnuplot [raw gnuplot] {
                     set samples 200;
                     q2_ab(x) = (.5 * x) ** 2 - \mPi * \mPi;
                     q2_xi    = q2_ab(\m * \m);
                     width(x) = \T * \m * sqrt(q2_ab(x)) / (sqrt(q2_xi * x));
                     BW(x) = 1. / (\m * \m - x * x - {0,1} * width(x * x));
                     plot [.279:1.73] abs(BW(x));
                 };\addlegendentry{Rel.~Breit-Wigner ($J=0$)}

%        \addplot [domain={.279:1.73}, smooth, densely dotted]
%                 gnuplot [raw gnuplot] {
%                     set samples 200;
%                     q2_ab(x) = (.5 * x) ** 2 - \mPi * \mPi;
%                     q2_xi    = q2_ab(\m * \m);
%                     F1(x) = sqrt(2 * x / (1 + x));
%                     width(x) = \T * \m * F1(q2_ab(x) * \R * \R) * (sqrt(q2_ab(x)) ** 3) / (sqrt((q2_xi ** 3) * x));
%                     BW(x) = 1. / (\m * \m - x * x - {0,1} * width(x * x));
%                     plot [.279:1.73] abs(BW(x));
%                 };\addlegendentry{Rel.~Breit-Wigner ($J=1$)}
%
%        \addplot [domain={.279:1.73}, smooth, densely dotted]
%                 gnuplot [raw gnuplot] {
%                     set samples 200;
%                     q2_ab(x) = (.5 * x) ** 2 - \mPi * \mPi;
%                     q2_xi    = q2_ab(\m * \m);
%                     F2(x)    = sqrt(13 * x * x / (x * x + 3 * x + 9));
%                     width(x) = \T * \m * F2(q2_ab(x) * \R * \R) * (sqrt(q2_ab(x)) ** 5) / (sqrt((q2_xi ** 5) * x));
%                     BW(x) = 1. / (\m * \m - x * x - {0,1} * width(x * x));
%                     plot [.279:1.73] abs(BW(x));
%                 };\addlegendentry{Rel.~Breit-Wigner ($J=2$)}

        \addplot [domain={.279:1.73}, samples=100, smooth, loosely dash dot]
                 gnuplot {1./sqrt((\m * \m - \T * \T - x * x)**2 + 4 * (\m * \T)**2)};\addlegendentry{Pole mass}
    \end{axis}
\end{tikzpicture}

        \caption{Comparison between the non-relativistic Breit-Wigner and the pole-mass dynamic shapes. The parameters I used are $(\SI{.98}{\giga\electronvolt/\c^2}, \SI{.07}{\giga\electronvolt/\c^2})$, corresponding to $(m_\xi,\Gamma_\xi)$ for the Breit-Wigner shape; and $(\Re(m_\xi),\Im(m_\xi))$ for the pole-mass shape.}

        \label{fig:bw_pm_comparison}
    \end{figure}
    The figure~\ref{fig:bw_pm_comparison} shows a comparison between the non-relativistic Breit-Wigner and the pole-mass dynamic shapes.

    {\color{red} Other parametrizations are the Flatté and the pole-mass.\marginpar{should I mention this?}

    \begin{equation}
        \Delta_{\xi}^{\text{F}}(s) \coloneqq \frac{g_1}{m_{\xi}^2 - s - \iu(\rho_1 g_1^2 + \rho_2 g_2^2)}.
    \end{equation}
    Where the mass of the resonance $\xi$ is complex.
    }

    In the model-dependent isobar decomposition, the decay amplitude~\eqref{eq:isobar_decomposition} finally reads\marginpar{more efficient for caching, but wrong with my conventions}
    \begin{equation}\label{eq:redundant_model_dependent_isobar_decomposition}
        \Psi_I(\tau) =  \sum_{\xi\in I} \gamma_\xi \,\psi_{J_\xi}\!(\tau)\,\Delta_{\xi}(s),\quad
        \text{being }
        \gamma_\xi\coloneqq c_I \alpha_{\xi}.
    \end{equation}
%    \begin{equation}\label{eq:redundant_model_dependent_isobar_decomposition}
%        \A(\tau) = \sum_{I\in\set{(J, P, C)}} \psi_I(\tau) \sum_{\xi\in I} \gamma_\xi \Delta_{\xi}(s),\quad
%        \text{being }
%        \gamma_\xi\coloneqq c_I \alpha_{\xi}.
%    \end{equation}
    I have absorbed the coefficients $c_I$ and $\alpha_\xi$ into one coefficient $\gamma_{\xi}$, that I will call \emph{free amplitude}\index{free amplitude}.
    Please note that the free amplitude $\gamma_\xi$ in the equation~\eqref{eq:redundant_model_dependent_isobar_decomposition} does not need the isobar label: since each resonance is in a well-defined isobar, a double index like $\gamma_{I,\xi}$ would be redundant.
    For the same reason, the common angular part in the equation~\eqref{eq:redundant_model_dependent_isobar_decomposition} can be factored out of the sum:
    \begin{equation}\label{eq:non_redundant_model_dependent_isobar_decomposition}
        \Psi_I(\tau) =  \psi_{J_I}\!(\tau)\sum_{\xi\in I} \gamma_\xi \,\Delta_{\xi}(s).
    \end{equation}
    {\color{red} TODO point out that one is more suited for smart caching.}




    At this point, I would like to stress that in the expansion~\eqref{eq:non_redundant_model_dependent_isobar_decomposition} there are several sources of systematic uncertainties.
    The assumption about the resonance content of each isobar is not known and highly affects the quality of the model description. {\color{red} cite \cite{PhysRevD.76.012001}}\marginpar{some resonances are not yet well-established}
    The dynamic shape of each resonance---along with its parameters---has to be experimentally determined; so it has to be silently assumed that neither the dynamic shape of the resonance, nor its parameters are affected by the particular interaction in the decay.


    The limitations of the model-dependent isobar formalism are becoming critical due to the availability of larger and larger data sets.
    In the next section I will present how such limitations can be circumvented by means of the model-independent isobar formalism.


In the model-independent \ac{pwa}, the waves are no longer linearly independent.
In fact, it may exist a set of non-vanishing functions, $\set{f_I}$, such that:
\begin{equation}\label{eq:zero_mode}
    z(m_X,\tau) = \sum_{\sigma} \sum_{I} \Psi_I(m_X,\tau)\,f_I(m_\sigma) = 0,
\end{equation}
$\sigma$ being the symmetrization index of the particles forming the isobar, and $I$ being the isobar index~\cite{green_krinn_paul}.
The set $\set{f_I}$ is called \emph{zero mode}\index{zero mode}.


In particular, equation~\eqref{eq:zero_mode} implies that
\begin{equation}
    \abs{\A(m_x,\tau)}^2 = \abs{\A(m_X, \tau) + z(m_X,\tau)}^2,
\end{equation}
namely, there may be a continuous ambiguity in the wave content when more than one wave is freed in the fit.
In my model-independent fits of $\PDplus\to\Ppiplus\Ppiminus\Ppiplus$ data sets, I observed that there is no zero mode when the S and D wave are simultaneously freed (see plots in figures~\ref{fig:f0_f2:f0_fit}, \ref{fig:f0_f2:f2_fit}, and~\ref{fig:f0_f2_dalitz}); 
whereas a zero mode appears when the S and P waves are freed (see plots in figures~\ref{fig:f0_f0_1500_sigma_rho0:s_wave}, \ref{fig:f0_f0_1500_sigma_rho0:p_wave}, and~\ref{fig:f0_f0_1500_sigma_rho0_dalitz}).

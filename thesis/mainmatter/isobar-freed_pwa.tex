The accuracy on the parameters of resonances strongly depends on the quality of the decay model that has been adopted to perform the fit.
In particular, the isobar formalism has some limits that must be overcome if the decay description is to be improved.
The knowledge of isobars comes from past experiments that had to determine mass and width of intermediate resonances.
This means that the resonance parameters (and, sometimes, shapes as in the case of ?) are only known with limited precision, thus introducing systematic effects in the decay modelling.
Thus it is useful to reduce the analysis dependence on the isobar model and extract the intermediate resonance shapes directly from the experiment itself.


In the freed-isobar \ac{pwa}, a set of binned function is used to describe the unknown shape of an isobar amplitude.
This way, there is no need to use any knowledge on the isobar from past experiment.
The kinematically-allowed phase-space mass range for the final state $ab$, which is $[m_a + m_b, M - m_c]$, is partitioned into a $N_\text{bins}$ number of bins.
\begin{figure}
    \centering
    \begin{tikzpicture}
        \pgfmathtruncatemacro{\a}{3}
        \pgfmathtruncatemacro{\b}{1}
        \pgfmathtruncatemacro{\c}{2}

        \begin{scope}[color=black!50]
            \coordinate (o) at (-1, 0);
            \draw (o) -- +(\a + 2, 0) coordinate (a);
            \draw [densely dashed] (a) -- +(\b, 0) coordinate (b);
            \draw [->] (b) -- +(\c, 0) coordinate (c);
        \end{scope}

        \begin{scope}
            % vertical label shift
            \pgfmathsetmacro{\s}{-.5}

            \foreach \x in {0,1,...,\a} {
                \node [draw, inner sep=1pt, circle, fill=black] (x\x) at (\x, 0) {};
                \node (sx\x) at ($(x\x) + (0, \s)$)  {$m_\x$};
            }

            \node [draw, inner sep=1pt, circle, fill=black] (last) at (\a + \b + 2, 0) {};
            \node (slast) at ($(last) + (0, \s)$) {$m_{N_\text{bins}}$};

        \end{scope}

        \node () at (x0) [label={above:$m_a + m_b$}] {};
        \node () at (last) [label={above:$M - m_c$}] {};


    \end{tikzpicture}
    \caption{Invariant-mass partition into $N_\text{bins}$ bins.}
    \label{fig:invariant-mass-partition}
\end{figure}
Referring to the labeling scheme showed in the figure~\ref{fig:invariant-mass-partition}, let $B_i \coloneqq [m_{i-1}, m_i)$ be the $i$-th right-open mass range, where $i$ ranges from $1$ to $N_\text{bins}$.
The bin functions are defined as follows:
\begin{equation}
    \Delta_i(m_X; m_a, m_b) =
    \begin{cases}
        1 &\text{if}\ m_X \in B_i, \\
        0 &\text{otherwise}.
    \end{cases}
\end{equation}
The dynamical function of the isobar can now be constructed as a linear combination of the bin functions:
\begin{equation}
    \Delta(m_X; m_a, m_b) = \sum_{i=1}^{N_\text{bins}} a_i \Delta_i(m_X;m_a, m_b),
\end{equation}
which is a step function depending on the complex parameters $\Set{a_1, \dots, a_{N_\text{bins}}}$.
The unknown parameters $\Set{ a_i }$ will be determined from a fit of the decay model to the experimental data.


At this point it is worth noting that this step-like description of the dynamic function of an isobar introduces $N_\text{bins}$ fit parameters, which drastically increases the complexity of fitting the model to the data.
This problem can be partially solved considering that:
\begin{itemize}
    \item 
        This formalism allows mixing fixed and freed resonance dynamic functions in a model, so that the number of free fit parameters can be somehow contained;
    \item
        The partition bins do not have to be uniform. By means of a non-uniform mass-range partition, making narrower bins in the mass interval where a resonance is expected, the number of necessary fit parameters can be optimized.
\end{itemize}

{\color{red}
(Quote Fabian's work as unpublished?)
}

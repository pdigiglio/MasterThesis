\subsection{Single-wave Dalitz plots}

    \begin{figure}
        \centering

        \subfloat[][]{\begin{tikzpicture}
    \begin{axis}[phase_plot]
        \addplot+ [guess] table [x index={0}, y index={5}] {data/f2/par_guess.txt};
        \addplot+ [fit] table [x index={0}, y index={5}, y error index={6}]
                  {data/f2/par_fit.txt};

    \end{axis}
\end{tikzpicture}
}

        \subfloat[][]{\begin{tikzpicture}
    \begin{axis}[amplitude_plot]
        \addplot+ [guess] table [x index={0}, y index={1}] {data/f0/par_guess.txt};
        \addplot+ [fit] table [x index={0}, y index={1}, y error index={2}]
                  {data/f0/par_fit.txt};
    \end{axis}
\end{tikzpicture}

}

        \caption{Freed-wave fit of the S wave of a $\PDplus \to \Ppiplus\Ppiminus\Ppiplus$ decay through the \Pfnez{} and \Pfofzz{} resonances.~\Star}
        \label{fig:f0_f2_fit}
    \end{figure}
    \begin{figure}
        \centering
        \subfloat[][Source data: S-wave decay through the \Pfnez{} and \Pfofzz{} resonances.\label{fig:f0_f0_1500_dalitz_data}]{\begin{tikzpicture}
    \begin{axis} [dalitz_plot]

        \addplot3 [dalitz]
          gnuplot [raw gnuplot] { splot "data/f2/f2_mcmc.txt" using 1:2:3 };
    \end{axis}
\end{tikzpicture}
}

        \subfloat[][Result of the model-independent fit to the plot shown above.\label{fig:f0_f0_1500_dalitz_fit}]{\begin{tikzpicture}
    \begin{axis} [dalitz_plot]
        \addplot3 [dalitz]
          gnuplot [raw gnuplot]
                  { splot "data/f0_f0_1500_sigma_rho0/f0_f0_1500_sigma_rho0_fit_result_mcmc.txt" using 1:2:3 };
    \end{axis}
\end{tikzpicture}
}

        \caption{\ac{mc}-generated Dalitz plots of a $\PDplus \to \Ppiplus\Ppiminus\Ppiplus$ decay.~\Star}
        \label{fig:f0_f2_dalitz}
    \end{figure}
    Figure~\ref{fig:f0_f2_fit} shows the fit performed on the $\PDplus\to\Ppiplus\Ppiminus\Ppiplus$ decay proceeding through the $\Pfnez{}$ and $\Pfofzz{}$ resonances in the S wave.
    Both resonances are modeled as non-relativistic Breit-Wigner with zero relative phase.


    When there is more than one resonance in the same wave, the dynamic-shape phase is no longer guaranteed to be monotonically increasing.
    To perform the fit, I set the phase-motion range to $[-\pi/2, \pi/2]$.
    Also in this case, the fit well reproduces the data resonance content.
    Figure~\ref{fig:f0_f2_dalitz} shows the comparison between the Dalitz plots of the source data and the data generated with the fitted parameters.


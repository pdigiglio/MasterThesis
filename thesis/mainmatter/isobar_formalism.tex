Experiments have shown that nonleptonic three-body decays proceed through intermediate two-body resonant decays.
Thus, the amplitude is modeled as a coherent sum of two-body decays plus a non-resonant contribution:\marginpar{\color{red}Mention the NR term, also mention it's problematic (quote) and that I'll drop it}
\begin{equation}\label{eq:general_decay_amplitude}
    \A(m_{ab}^2, m_{bc}^2) = \sum_r a_r \eu^{\iu\phi_r}\A_r(m_{ab}^2,m_{bc}^2)
    + a_{\textup{NR}} \eu^{\iu \phi_{\textup{NR}}} \A_{\textup{NR}}(m_{ab}^2,m_{bc}^2).
\end{equation}
The parameters $a_r$ and $\phi_r$ are the magnitude and phase of the amplitude for the resonance component $r$.
The same interpretation holds for $a_\text{NR}$ and $\phi_\text{NR}$.
These parameters are modeled by the \lstinline!FreeAmplitude! class in \pac{yap}.


One way to parametrize the resonant components of the decay amplitude~\eqref{eq:general_decay_amplitude} is through the isobar model (or isobar formalism).
In this form, the function $\A_r$ is decomposed in the following product:\marginpar{\color{red} Show how my decay amplitude looks like}
\begin{equation}
    \A_r = F_P\,F_r\, T_r\, W_r.
\end{equation}
Here, the product of the dynamical function, $T_r$, and the angular distribution, $W_r$, is the propagator of the resonance $r$.

The form factors $F_P$ and $F_r$ usually use the Blatt-Weisskopf parametrization for the decay vertex.

The $K$-matrix is an alternative approach to the isobar formalism for the amplitude calculation.

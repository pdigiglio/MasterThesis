\chapter{Introduction}


    \paragraph{Some physical motivation}
        Examples of usage:
        \begin{itemize}
            \item Searches for new states;
            \item Measuging resonance properties;
            \item $CP$ violation;
            \item B and D mixing;
            \item Resolving ambiguities.
        \end{itemize}

    \paragraph{Why model-independent analysis}

    Due to the availability of larger and larger data sets, the limiting factor in the data analysis is the quality of the model.


    Model-independent analysis can be useful to solve ambiguity like:

    ovelapping states

    not well-known resonances.

    \paragraph{goals of the thesis}

    I have contributed to the development of \pac{yap}, a novel toolkit for \ac{pwa}, briefly described in the chapter~\ref{chap:yap}.
    I have developed a fitting routine based on \pac{yap} that implements a model-independent fit.
    I have tested the routine on a variety of \ac{mc} data with increasing complexity (see the chapter~\ref{chap:model_independent_pwa}).





    The aim of this thesis is the description of \pac{yap}, a novel \ac{pwa} toolkit that I have contributed to develop.
    I also show the results of some model-independent fits performed using a self-developed piece of software relying on the external libraries \pac{yap} and \pac{bat}.



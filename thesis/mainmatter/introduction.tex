\chapter{Introduction}
\label{chap:introduction}

    Hadronic heavy-meson decays into multi-body final states are a great source of information for studies about charge-conjugation and parity violation, mostly performed by means of a \ac{pwa}.


    So far, the model-dependent approach to the \ac{pwa} has been the standard analysis technique in these studies for light- and heavy-meson decays.
    The success of this technique in providing a good description of the decay strongly depends on the quality of the decay model, \ie~on the correct assumptions about the resonant and non-resonant content of the waves and their parametrization.


    Increasingly large experimental data sets are collected at both collider and fixed-target experiments.
    On the one hand, this led to the discovery of new bound states and gave physicists the possibility to test very refined models.
    On the other hand, the model-dependent approach to \ac{pwa} is now proving limited as the systematic uncertainties introduced by incomplete or incorrect modelling of the decays dominate the statistical uncertainties.
    %the systematic uncertainties introduced by the model-dependent \ac{pwa} are now dominating the statistical uncertainties.
    An extension of the \ac{pwa} technique is thus needed to achieve a higher precision in the phenomenological decay description.


    The approach I propose and test here is the \ac{mipwa}, which exploits the increased size of the available experimental data to get rid of unjustified model assumptions.
    While not being a widespread analysis technique, the \ac{mipwa} has already been successfully used in the past---for example by the \focus{} collaboration to parametrize the S-wave component of a $\PDplus\to\PKminus\Ppiplus\Ppiplus$ decay~\cite{Link200914}.


    In this thesis, I will discuss my implementation of a \ac{mipwa}-fit utility.
    I will show the difficulties that come with a model-independent decay description---most notably, the large increase in the number of fit parameters and the appearance of fit ambiguities---and the results of several tests of my utility on \ac{mc} data.
    The aim of such a study is to provide a general utility for \ac{mipwa} fits to be used on real data collected at the \textsl{Belle} and \textsl{Belle II} experiments.


    My fit utility is based on \pac{yap}, a novel \cpp{} toolkit for \ac{pwa} developed at the \textsl{Technische Universit\"at M\"unchen} by Dr.~Greenwald and Mr.~Rauch, to which I have also contributed.
    I will briefly describe \pac{yap} in chapter~\ref{chap:yap}, after a short introduction to the \ac{pwa} formalism in heavy-meson decays in chapter~\ref{chap:theory}, and finally discuss my fit utility in chapter~\ref{chap:model_independent_pwa}.

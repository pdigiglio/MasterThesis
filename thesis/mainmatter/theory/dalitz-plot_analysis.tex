\section{Dalitz-plot analysis}
    In a two-body decay, the energy-momentum conservation fully determines the energy of the final-state particles in the center-of-mass frame.
    In a three-body decay, after requiring energy and momentum conservation, there are still five remaining \acp{dof} in the frame of the final state (i.e.~the rest frame of the parent particle).
    If the initial and final state particles have spin zero, up to an arbitrary rotation, two \acp{dof} remain.


    The amplitude of the three-body decay can be expressed with respect to several parameter pairs.
    It is thus convenient to choose a couple of parameters such that the phase-space term is constant within the kinematically-allowed region they span.
    Two such variables are the square invariant masses of the two pairs of final-state particles, i.e.~$m_{ab}^2 = p_{ab}^2$ and $m_{bc}^2 = p_{bc}^2$, being $a$, $b$, and $c$ the labels for the final-state particles.
    The differential decay probability for a parent particle with mass $M$ into three final-state particles is
    \begin{equation}
        \ud\Gamma = \frac{1}{(2\pi)^3}\frac{1}{32M^3}\abs{\A(m_{ab}^2, m_{bc}^2)}^2
        \ud m_{ab}^2 \!\ud m_{bc}^2.
    \end{equation}
    As the phase-space measure is a constant function, any non-uniformity in the distribution of $m_{ab}^2$ and $m_{bc}^2$ is due to the decay amplitude $\A$.

    \begin{figure}
        \centering
        \begin{tikzpicture}
%    \pgfmathsetmacro{\mTwoA}{(.13957)^2} % pi+ mass squared [GeV/c^2]
%    \pgfmathsetmacro{\mTwoB}{(.13957)^2} % pi- mass squared [GeV/c^2d
%    \pgfmathsetmacro{\mTwoC}{(.13957)^2} % pi+ mass squared [GeV/c^2d
%    \pgfmathsetmacro{\mTwo}{(1.86963)^2} % D+  mass squared [GeV/c^2d      

    \pgfmathsetmacro{\mTwoA}{(.17)^2}
    \pgfmathsetmacro{\mTwoB}{(.17)^2}
    \pgfmathsetmacro{\mTwoC}{(.17)^2}
    \pgfmathsetmacro{\mTwo}{(1.)^2}

    \pgfmathsetmacro{\lBound}{(sqrt(\mTwoA) + sqrt(\mTwoB))^2}
    \pgfmathsetmacro{\uBound}{(sqrt(\mTwo)  - sqrt(\mTwoC))^2}
    \pgfmathsetmacro{\lyBound}{(sqrt(\mTwoC) + sqrt(\mTwoB))^2}
    \pgfmathsetmacro{\uyBound}{(sqrt(\mTwo)  - sqrt(\mTwoA))^2}

    \begin{axis}[
        ylabel={$m_{ab}^2$},
        xlabel={$m_{bc}^2$},
        ytick={\lBound,\uBound},
        yticklabels={$(m_a + m_b)^2$, $(M-m_c)^2$},
        xtick={\lyBound,\uyBound},
        xticklabels={$(m_b + m_c)^2$, $(M-m_a)^2$},
        yticklabel style={sloped like y axis},
        grid = major,
        declare function = { E_b(\t) = ((\t -\mTwoB + \mTwoA)/(2*sqrt(\t))); },
        declare function = { P_b(\t) = sqrt(E_b(\t)^2 -\mTwoB); },
        declare function = { E_c(\t) = ((\mTwo -\t -\mTwoC)/(2*sqrt(\t))); },
        declare function = { P_c(\t) = sqrt(E_c(\t)^2 -\mTwoC); },
        enlargelimits=.12,
    ]

        \addplot[
            name path = A,
            black,
            opacity=0,
            domain = {\lBound:\uBound},
            samples=250,
        ] {(E_b(x) + E_c(x))^2 - (P_b(x) - P_c(x))^2};

        \addplot[
            name path = B,
            black,
            opacity=0,
            domain = {\lBound:\uBound},
            samples=250,
        ] {(E_b(x) + E_c(x))^2 - (P_b(x) + P_c(x))^2};

        \addplot[blue!50, opacity=.3] fill between[of=A and B];

    \end{axis}
\end{tikzpicture}

        \caption{Kinematically-allowed region in the phase space of a three-body decay. Resoances will appear as bands within the allowed region.}
        \label{fig:dalitz_kinematically_allowed}
    \end{figure}
    The kinematically-allowed phase-space region, shown in figure~\ref{fig:dalitz_kinematically_allowed}, can be expressed in terms of a constrain on $m_{bc}^2$ as follows:
    \begin{equation}
        \begin{aligned}
            (m_{bc}^2)_{\text{max}} &= (E_b^* + E_c^*)^2 - (p_b^* - p_c^*)^2,\\
            (m_{bc}^2)_{\text{min}} &= (E_b^* + E_c^*)^2 - (p_b^* + p_c^*)^2,
        \end{aligned}
    \end{equation}
    where
    \begin{equation}
        E_b^* = \frac{m_{ab}^2 - m_a^2 + m_b^2}{2m_{ab}},\quad% \text{ and }
        E_c^* = \frac{M^2 - m_{ab}^2 - m_c^2}{2 m_{ab}}
    \end{equation}
    are the energies of the final-state particles $b$ and $c$ in the rest frame of the system $ab$ and
    \begin{equation}
        p_b^* = \sqrt{{E_b^*}^2 - m_b^2},\quad% \text{ and }
        p_c^* = \sqrt{{E_c^*}^2 - m_c^2}
    \end{equation}
    the corresponding 3-momenta magnitudes.

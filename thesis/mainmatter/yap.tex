\chapter{YAP}
\label{chap:yap}

    \pac{yap} is a free (licensed under the \acs{gpl3}) and open-source \cpp[11] software for calculating partial-wave amplitudes for $n$-body particle decays.
    The source code is designed to be human-readable and to reflect the mathematical formalism describing the particle decays.
    All the physics used is cited and documented in the code with the \package{doxygen} package, and verified through an extensive suite of tests.\footnote{As of writing, the tests cover \SI{90}{\percent} of the code.}

    One of the main goals of \pac{yap} is modularity.
    In fact, it is distributed in the form of a shared library to be linked against the user's application.
    Moreover, \pac{yap} does not depend on any library beyond the \cpp{} \ac{stl}, which makes it easily compilable on most operating systems and portable on many architectures.
    It is worth stressing that  \pac{yap} is an amplitude calculator only and it does not come with any \ac{mc} sampler or \acs{gui}.
    Thus it has to be coupled with third-party software that provides such functionalities---if needed.


    \pac{yap} is currently maintained by Dr.~Daniel Greenwald and Johannes Rauch at the \textsl{Techniche Universit\"at M\"unchen}.
    The project is hosted on \textsf{GitHub} at \url{https://github.com/yap/YAP}.\marginpar{\centering\qrcode[height=2cm]{https://github.com/yap/YAP}\captionof*{figure}{\pac{yap}'s {\footnotesize URL}.}}
 

%    \pac{yap} is designed to handle particle decays into final states with three or more particles through an intuitive \ac{api}.
%    Also, the library is designed to handle the parallelization of the amplitude calculation through its own threading system.
%    In the future, parts of the calculations may be offloaded to \acsp{gpu} using \package{OpenCL} or Nvidia's \pacs{cuda}.
%
%
%    We wrote code that is compliant to the last supported \cpp{} standards (currently, \cpp[14]) and is inspired to the \ac{stl}.
%    For instance, we provide iterators, inserters and algorithms.
%    For the development of \pac{yap} we used~\cite{stl_meyers,effective_cpp_meyers,effective_modern_cpp_meyers}.

    \section{Structure}

    As stated earlier, \pac{yap}'s code aims to mimic the mathematical description of particle decays.
    Its design parallels the recursive formula that gives a particle-decay probability amplitude.
    Given a decaying particle $X$, its decay probability amplitude is the sum over the decay amplitudes of all the channels $X\to D$, being $D\coloneqq \Set{d_1, d_2,\dots,d_n}$ a possible decay product:
    \begin{equation}\label{eq:yap_decay_amplitude}
        \A_X = \sum_{X\to D} a_{X\to D}\cdot F_{X\to D}\cdot \Omega_{X\to D}\cdot T(m_X)\prod_{d\in D} \A_d.
    \end{equation}
    The terms in which the amplitude~\eqref{eq:yap_decay_amplitude} is factorized are to be interpreted as follows:
    \begin{description}
        \item[$a_{X\to D}$] is the free amplitude\index{free amplitude};
        \item[$F_{X\to D}$] is the Blatt-Wei\ss{}kopf angular-momentum penetration factor\index{form factor}\index{Blatt-Wei\ss{}kopf form factor};
        \item[$\Omega_{X\to D}$] is the spin amplitude\index{spin amplitude};
        \item[$T(m_X)$] is the dynamic shape\index{dynamic shape} of the particle $X$;
        \item[$\A_d$] is the amplitude of the daughter $d$ and, again, has the same form of the equation~\eqref{eq:yap_decay_amplitude}.
            If $d$ is a final-state particle, $\A_d = 1$. 
    \end{description}

    \pac{yap} provides (among the others) the following classes.
    \paragraph{\lstinline!Model!} provides the spin formalism, manages all the decay channels and the data points.
    \paragraph{\lstinline!Particle!} is the base class for the final-state particles, decaying particles and resonances.
    \paragraph{\lstinline!DecayChannel!} holds daughters and spin amplitudes.
    \paragraph{\lstinline!ParticleFactory!} is a helper class to create the particles.
        It provides an intuitive back-end to the \package{EvtGen} particle information contained in the \texttt{evt.pdl} table described in~\cite[p.~13]{evtgen_manual}.

%    Moreover, the library code is strucitured to mimic the mathematics of \ac{pwa}.
%    {\color{red}{what's the meaning of that?}}
%
%    \pac{yap} is targeted to physicists.
%    Its code is human readable and is written to mimic the matematical structure of \ac{pwa}.
%    \pac{yap}'s \ac{api} allows the users to focus on the physics only.
%
%    Moreover, \pac{yap} can be coupled to external tools such as \pac{bat} for event generation and fitting, and \MINUIT{} for fitting.
%
%    The structure of \pac{yap} reflects the mathematical construction of a \ac{pwa}.
%    The amplitude of an $n$-body decay $X\to D$, being $D = \Set{d_1, d_2,\dots,d_n}$ a possible decay product, is given by the recursive formula
%    \begin{equation}
%        \A = \sum_{\text{decays}} \prod_{d \in D} \A_d
%    \end{equation}

    \section{Challenges}

        The \ac{pwa} poses several challenges that \pac{yap} aims to project has to face both from the computational and programming point of views.
        The large data and wave sets make the analyses computationally expensive.
        Moreover, the large number of free parameters during the fit increases the chances for local maxima of the likelihood function.


        \pac{yap} features a smart-caching\index{smart caching} feature for fast amplitude calculation.
        The wave amplitudes know when their parameters have changed and are not recalculated untill then.
        As an example, the \lstinline!BreitWigner! dynamic shape internally stores the mass and width of the resonance they it is modeling as smart parameters.
        Thus they will be evaluated again on all the \lstinline!DataPoint!'s only if any of the parameters changes.




        \pac{yap} supports the parallelization of the calculations on the \acs{cpu} through the thread-support library introduced with the \cpp{11} standard.\footnote{For more information on this library, please refer to~\cite[chap.~?]{effective_modern_cpp_meyers}.}
        The \lstinline!DataPartition! class is a partitioning of the data set and allows for thread-safe parallel calculations.
        We also investigated whether it was possible to offload part of the calculation to \acsp{gpu} using \package{OpenCL} or Nvidia's \pacs{cuda}.
        However, the state of the code did not allow to easily implement it.
        At the current state this could be easier as the data is already structured in a thread-frendly way.


        Incomplete decay modeling can result in fit failure.
        To prevent this, I implemented a \pac{yap}-based utility for \ac{mipwa}. Please refer to chapter~\ref{chap:model_independent_pwa} for more information.

        \section{Example}
            \subsection{Generate a \ac{mc} data set}

            \pac{yap} is an amplitude calculator only, not a sampler.
            However, examples on how to interface it with \pac{bat}---which implements a \ac{mcmc} method---are provided.
                \begin{figure}
                    \centering
                    \begin{tikzpicture}
                        \begin{axis} [dalitz_plot]
                            \addplot3 [dalitz]
                              gnuplot [raw gnuplot] { splot "data/cleo/rho0_f0_f0_1500_f0_500_mcmc.txt" using 1:2:3 };
                        \end{axis}
                    \end{tikzpicture}

                    \caption{\ac{mc}-generated Dalitz plot of a $\PDplus \to \Ppiplus\Ppiminus\Ppiplus$ decay with the following resonances:
                             \Prhozero{} with a relativistic Breit-Wigner dynamic shape,
                             \Pfnez{} with a Flatté dynamic shape,
                             \Pfofzz{} with a relativistic Breit-Wigner dynamic shape, and \Psigma{} with a pole-mass dynamic shape.
                             The resonance dynamic shapes, parameters, and free amplitudes are those the \cleo~collaboration reported in~\cite{PhysRevD.76.012001}.}
                    \label{fig:partial_cleo_decay}
                \end{figure}
            % -- Model-creation example ---------------------
            \clearpage
            \lstinputlisting[caption={Excerpt of the function used to create the Dalitz plot~\ref{fig:partial_cleo_decay}.\label{lst:partial_cleo_decay}}]{source/d3pi.cxx}
            % -----------------------------------------------
            The listing~\ref{lst:partial_cleo_decay} shows an excerpt of the model used to generate the Dalitz plot shown in figure~\ref{fig:partial_cleo_decay}.
            {\color{red} Why am I trying to emulate \cleo{} studies?}

        \subsection{Add a new dynamic shape}

            To perform the \ac{mipwa} fit I had to extend \pac{yap}'s functionalities by implementing a mass-bin dynamic shape.
            In \pac{yap} any new dynamic shape should inherit from the pure-virtual \lstinline!MassShape! class.
            \lstinputlisting[caption={Creation of a new dynamic shape.{\color{red} TODO add the correct source}}]{source/d3pi.cxx}
        


%        \subsection{Performance}
%
%        \pac{yap} is still under developement so its \ac{api} may change.
%        Nevertheless, the following structure should remain valid in the future.
%
%        Parameters in \pac{yap} are shared among calculating objects which then know if a parameter has changed and if new calculations need to be done.
%        {\color{red} examples}
%
%        Data is cached in data points, \lstinline!DataPoint! classes, and organized in sets, which are then partitioned for parallel calculations.


    \section{Future enhancements}
    Our library is usable but still under heavy developement and far from complete.
    Some feature to be implemented are:
    \begin{itemize}
        \item Add the $K$-matrix formalism;
        \item bla bla bla
    \end{itemize}

    
    
    I took over the task of implementing a model-independent approach to \ac{pwa} through picewise mass shapes.





\subsection{The fit \acp{dof}}

    To perform the fit, I developed a back-end to \lstinline!BCModel!, the \pac{bat} class that sets the fit parameters up and returns the logarithm of the model likelihood.
    As \pac{bat} can only handle real fit \acp{dof}, I introduce an invertible map between the complex free amplitudes---the $\gamma_{(I,i)}$ coefficients of the decay model in equation~\eqref{eq:freed_wave_amplitude}---and the fit parameters. % $\Phi\colon \C\to\R\x\R$.


    To reconstruct the isobar dynamic shape, we are interested in amplitude and phase of the free amplitudes, $\gamma_i$, namely a choice of the fit parameters may be (please note that the isobar label, $I$, is now fixed):
    \begin{equation}
        (\abs{\gamma_i}, \arg\gamma_i).
    \end{equation}
    However, \pac{bat} cannot properly sample the phase space of a periodic variable.
    So, instead, I parametrize a free amplitude with its amplitude and \emph{phase motion}\index{phase motion}:
    \begin{equation}
        (\abs{\gamma_i}, \uD \phi_i),\quad\text{with }
        \uD \phi_i \coloneqq
        \begin{cases}
            \arg\gamma_i - \arg\gamma_{i-1},    &\text{if }i > 0,\\
            0                                   &\text{otherwise}.
        \end{cases}
    \end{equation}


    %Ambiguities:
    %\begin{equation}\label{eq:intensity_phase_ambiguity}
    %    \Intensity(\tau) = \abs{\A(\tau)}^2 = \abs{\A^*(\tau)}^2 = \abs{\eu^{\iu\phi}\A(\tau)}^2,
    %    \quad
    %    \text{with }
    %    \phi\in\R.
    %\end{equation}


    %The first test fit I performed is to the data of the $\PDplus\to(\Pfnez\to\Ppiplus\Ppiminus)\Ppiplus$ decay, shown in Dalitz plot~\ref{fig:dalitz_f0_980_only_source}.
    %\begin{figure}
    %    \centering
    %    \begin{tikzpicture}
    \begin{axis}[
            xlabel = {$\Re (\Delta_{\Pfnez}^{\text{BW}}(s))$ [\si{1/(\giga\electronvolt/\c^2)^2}]},
            ylabel = {$\Im (\Delta_{\Pfnez}^{\text{BW}}(s))$ [\si{1/(\giga\electronvolt/\c^2)^2}]},
        ]

        \pgfmathsetmacro{\m}{.98}
        \pgfmathsetmacro{\T}{.07}

        \addplot+ [black, mark options={mark size = .75, color=black}]
          gnuplot [raw gnuplot] {
            set parametric;
            f(x) = 1. / (\m * \m - x * x - {0., 1.} * \m * \T);
            set samples 150;
            plot [.279:1.73] real(f(t)), imag(f(t));
        };
    \end{axis}
\end{tikzpicture}

    %    \caption{Argand-Gau\ss{} plot of a non-relativistic Breit-Wigner~\eqref{eq:bw}.}
    %    \label{fig:bw_argand_gauss}
    %\end{figure}
    %The decay proceeds through the \Pfnez{} resonance only with non-relativistic Breit-Wigner dynamic shape, whose plot in the Argand-Gau\ss{} plane is shown in figure~\ref{fig:bw_argand_gauss}.
    %\begin{equation}
    %    \Phi(\gamma) = (\abs{\gamma}, \arg\gamma)
    %\end{equation}
    %I can exploit the fact that the fit parameters have a fixed position in the vector:
    %\begin{equation}
    %    \Phi(\gamma_0) = (\abs{\gamma_0}, 0)
    %\end{equation}
    %\begin{equation}
    %    \Phi(\gamma_0,\dots,\gamma_j) = (\abs{\gamma_j}, \arg(\gamma_j) - \arg(\gamma_{j-1})), \text{ so that }\phi_j = \sum_{i=0}^j \uD \phi_i
    %\end{equation}



    %I am fitting the amplitude~\eqref{eq:freed_wave_amplitude} to the data.


    %In the $\Phi(\gamma) = (\abs{\gamma}, \arg\gamma)$ representation, there is the phase ambiguity and the conjugate ambiguity.


    %The first fit I performed is the decay of the $\Pfnez$.
    %The real-imaginary representation of the free amplitudes proved inconvenient as the data is ambiguous.

    %In the first fit attempt, I represented the free amplitudes by their real and imaginary parts.
    %With this choice, the pre-run did not converge as there is an ambiguity in the sign of the imaginary part of the parameters.
    %In fatc, for each set of parameters $\Set{(\Re,\Im)_i}$ that maximizes the likelihood, also $\Set{(\Re,-\Im)_i}$ does.


    %As the plot~\ref{fig:bw_argand_gauss} shows, the angle is increasing.

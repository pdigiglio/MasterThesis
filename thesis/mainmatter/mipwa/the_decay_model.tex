\subsection{The decay model}

    As of now, \pac{yap} does not yet provide the piecewise dynamic shape in equation~\eqref{eq:step_dynamic_shape}, which is needed to perform a \ac{mipwa} of a decay.
    However, such a dynamic shape can be constructed in \pac{yap} starting from the formal analogy between the model-independent decay amplitude in equation~\eqref{eq:freed_wave_amplitude}---the one that has to be fit to the data---and the model-dependent decay amplitude in equation~\eqref{eq:non_redundant_model_dependent_isobar_decomposition}.
    In fact, equation~\eqref{eq:freed_wave_amplitude} also applies to a decay proceeding through $N_\text{bins}$ intermediate states, $\xi_i$, whose dynamic shape is a characteristic function, such that
    \begin{equation}
        \PDplus\to\xi_i\Ppiplus,\quad%\text{ and }
        \xi_i\to\Ppiminus\Ppiplus,\quad
        \text{with } i \in \set{1,\dots, N_{\text{bins}}}.
    \end{equation}
    In the \pac{yap} model, each wave contains as many artificial resonances as the bins of the mass range.
    After the model is locked, each resonance will be assigned a non-fixed free amplitude to be handed over to \pac{bat} for the fit stage.


    I also some convenience functions to partition the mass range and create the partitioning.


    Thus, I implemented the \lstinline!MassBin! dynamic shape presented in listing~\ref{lst:mass_shape}, page~\pageref{lst:mass_shape}.
    Then I create a \lstinline!Model! for a decay of a $\PDplus$ to $\xi_i\Ppiminus$, with the $\xi_i$ decaying itself to $\Ppiplus\Ppiminus$ and having a \lstinline!MassBin! dynamic shape: $\Characteristic_{B_i}$.
    I make sure that $\set{B_i}$ is a partition of the allowed mass range of the $\Ppiplus\Ppiminus$, namely that
    \begin{equation}
        \bigcup_{i=0}^{N_\text{bins}-1} B_i = [2m_{\pi},m_{\PD}-m_{\pi})
    \end{equation}
    Locking the model will associate a free amplitude to each dynamic shape.

    
    It is very important for the fit to work correctly that the free amplitudes do not get internally displaced and can always be associated to the correct mass bin.
    While the mass bins can be sorted by checking at their boundaries, there is no natural way to sort the free amplitudes.
    So I placed them in a \lstinline!std::vector! and sorted them by their index in the vector.
    \begin{figure}
        \centering
        \subfloat[][]{\begin{tikzpicture}
    \begin{axis}[
        amplitude_plot,
        width = .7\textwidth,
        xlabel={$m_{\pi^+\pi^-}$ [\si{\giga\electronvolt^2\per\c^4}]},
    ]
        \pgfmathsetmacro{\m}{.98}
        \pgfmathsetmacro{\T}{.07}
        \addplot+ [guess, domain={.279:1.73}, samples=100, smooth] gnuplot {1./sqrt((\m * \m - x * x)**2 + (\m * \T)**2)};
        
        \addplot+ [fit, mark=none] table {data/dalitz_displaced/binned_breit-wigner_displaced.txt};
    \end{axis}
\end{tikzpicture}
}

        \subfloat[][]{\begin{tikzpicture}
    \begin{axis} [dalitz_plot]

        \addplot3 [dalitz]
          gnuplot [raw gnuplot] { splot "data/dalitz_displaced/f0_displaced_mcmc.txt" using 1:2:3 };
    \end{axis}
\end{tikzpicture}
}

        \caption{Effect of the displacement of the fourth and the fourteenth free amplitudes.}
        \label{fig:bw_binning_displaced}
    \end{figure}
    Figure~\ref{fig:bw_binning_displaced} shows the effect of the displacement of two free amplitudes.
    This can happen because the \lstinline!free_amplitudes! function in \pac{yap} returns an unsorted \lstinline!std::set! of pointers to the free amplitudes of the decay model.\footnote{The \lstinline!std::set! is a sorted container but---in this case---its elements are sorted by their memory address.}

    



    This means that every bin in figure~\ref{fig:step_function_approximation}, where I show a binning example of a Breit-Wigner distribution, corresponds to a different decay mode of the parent particle.


    For this reason, I implemented the \lstinline!MassBin! dynamic shape shown in listing~\ref{lst:mass_shape}.

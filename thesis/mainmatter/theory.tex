\chapter{\texorpdfstring{\Acl{pwa}}{Partial-wave analysis}}
\label{chap:theory}

    The partial-wave decomposition is a standard technique in scattering theory.
    It allows to write down the scattering probability amplitude in terms of Legendre polynomials, the eigenvectors of the angular momentum operator, each representing a partial\index{partial wave} wave~\cite[\S~11.2]{griffiths_intro_qm}.


    In this chapter, I will show how to perform a \ac{pwa}\index{partial-wave analysis} to heavy-meson decays to disentangle the various contributions to the decay amplitude.
    I will focus on the isobar decomposition of a spinless parent particle to three pseudo-scalar spinless daughter particles in the final state.
    This is the case of the $\PDplus\to\Ppiplus\Ppiminus\Ppiplus$ decay, to which I will refer throughout the whole thesis.


    In the last part of the chapter, I will briefly introduce the reader to the Dalitz plot, a particularly useful visualization tool in \acp{pwa} of three-body decays.

    \section{Isobar decomposition}

    In quantum mechanics, the quantity that describes the distribution of the observed events in an experiment is the event intensity\index{intensity}, $\Intensity(\tau)$, $\tau$ being the phase-space coordinate of the event.
    The intensity is defined as the squared magnitude of the event probability amplitude\index{amplitude}:
    \begin{equation}
        \Intensity(\tau) \coloneqq \abs{\A(\tau)}^2.
    \end{equation}
    In this section I will show how to write down the expression for the probability amplitude of a decay, $\A$, by means of the \ac{pwa} formalism, which---in this context---I will also call \emph{isobar formalism}\index{formalism!isobar}\index{isobar!formalism}.


    In the isobar formalism, a \emph{resonance}\index{resonance} is a bound state of particles with well-defined quantum numbers.\footnote{Please note that a bound state of resonances is a resonance too.}
    The quantum numbers that characterize a resonance are its isospin, its spin, $J$, its parity, $P$, and its charge-conjugation parity, $C$.
    The notation usually adopted to indicate the resonance quantum numbers is $J^{PC}$.
    Decaying resonances can dissociate into final-state particles in more than one way, \ie~the decay can proceed through different topologies\index{decay topology}.


    A set of bound states with the same quantum numbers is called \emph{isobar}\index{isobar}.
    Consequently, the decomposition of a decay amplitude in terms of isobars is called \emph{isobar model}\index{isobar!model} or formalism.
    In this formalism, the decay amplitude of an initial-state particle, $X$, into a set of final-state particles---whose coordinate in the phase space is $\tau$---can be written as the following coherent sum of all the contributions from the isobars that are allowed by the conservation laws:
    \begin{equation}\label{eq:isobar_decomposition}
        \A_X(m_X,\tau) = \sum_{I\in\set{(J, P, C)}} c_I \Psi_I(m_X, \tau),
    \end{equation}
    $c_I$ being a complex coefficient quantifying the relative weight of the $I$-th isobar decay amplitude, $\Psi_I$.
    I would like to stress that in equation~\eqref{eq:isobar_decomposition} I omitted the sum over the decay topologies, as I will always do in the following discussion.
    Please also note that, in the context of the isobar decomposition, the partial waves correspond to the isobars.


    \begin{figure}
        \centering
        \begin{tikzpicture}[
        particle/.style = {circle},
        P/.style        = {particle, ball color=black!20},
        a/.style        = {particle, ball color=blue!20},
        b/.style        = {particle, ball color=green!20},
        c/.style        = {particle, ball color=red!20},
    ]

    % Parent particle
    \node[P] (P) at (0,0) {$X$};
    \draw [<->, color=black!50] ($(P) + (-25:1.5)$) arc (-25:25:1.5) node [midway, label={right:$L$}] {};
    %\draw (330:1) arc (30:1);

    \pgfmathsetmacro{\xDist}{6}
    \pgfmathsetmacro{\yDist}{3}

    \node[a] (a) at ($(P) + (\xDist, \yDist)$) {$a$};
    \node[b] (b) at ($(P) + (\xDist, 0)$)      {$b$};
    \node[c] (c) at ($(P) + (\xDist,-\yDist)$) {$c$};
    % Resonance
    \node[ ] (r) at ($(P) + (.5*\xDist, .5*\yDist)$) {$\xi$};
    \draw [<->, color=black!50] ($(r) + (-25:1.5)$) arc (-25:25:1.5) node [midway, label={right:$J_{\xi}$}] {};

    % Segments
    \draw[->] (P) -- (r);
    \draw[->] (r) -- (a);
    \draw[->] (r) -- (b);
    \draw[->] (P) -- (c);
\end{tikzpicture}

        \caption[Decay of a spinless parent particle to pseudo-scalar final-state particles through a resonance.]%
                {Decay of a spinless parent particle, $X$, to pseudo-scalar final-state particles, $a$, $b$, and $c$, through a resonance, $\xi$, in the $ab$ channel.
                 Alternative topologies for the decay have a resonance in the $ac$ or $bc$ channels.}
        \label{fig:isobar_three_body_decay}
    \end{figure}
    To reduce the complexity of the decay model, I will assume that the initial-state particle decays to the final-state particles via subsequent two-body decays, as schematically depicted in figure~\ref{fig:isobar_three_body_decay}.
    This assumption has been verified in non-leptonic three-body \PD{} and \PB{} decays~\cite[\S~13.2]{Bevan:2014iga}.


    Allowing two-body sub-decays only, the $\PDplus \to \Ppiplus\Ppiminus\Ppiplus$ decay will proceed as:
    \begin{equation}
        \PDplus\to\xi\Ppiplus,\qquad
        \xi\to\Ppiplus\Ppiminus,
    \end{equation}
    $\xi$ being an intermediate resonance.
    The list of allowed intermediate states, $\xi$, includes, for example, the \Pfnez{} with $J^{PC}$ state $0^{++}$; %, such that $\Pfnez\Ppiplus$ is in the $0^{-+}$ state;
    the \Prhozero{} with $J^{PC}$ state $1^{--}$; %, such that $\Prhozero\Ppiplus$ is in the $1^{--}$ state;
    and the \Pfii{} with $J^{PC}$ state $2^{++}$. %, such that $\Pfii\Ppiplus$ is in the $2^{++}$ state.


    For the composite state of the intermediate resonance and the \emph{spectator}\index{spectator particle} (or \emph{bachelor}\index{bachelor particle}) particle, $\xi\Ppiplus$, I evaluate the state as follows. 
    The angular-momentum conservation reads
    \begin{equation}
        \vec{J}_{\PD} = \vec{L} + \vec{J}_{\xi} + \vec{J}_{\Ppi},
    \end{equation}
    where, as shown in figure~\ref{fig:isobar_three_body_decay}, $\vec{L}$ is the relative orbital angular-momentum operator between the bachelor particle, $\Ppiplus$, and the resonance, $\xi$. %; and I have dropped the particle superscripts to make the formula less cumbersome.
    As both the \Ppiplus{} and the \PDplus are spinless particles, $\vec{J}_{\PD} = \vec{J}_{\Ppi} = 0$; thus
    \begin{equation}\label{eq:angular_momentum_sum}
        \abs{J_{\xi} - L} \le 0 \le J_{\xi} + L.
    \end{equation}
    In particular, the left-hand inequality in equation~\eqref{eq:angular_momentum_sum} implies that---for the specific decay under examination---the spin of the resonance, $J_\xi$, must be equal to $L$, the relative orbital angular-momentum quantum number of the $\xi\Ppiplus$ system.


    The parity is a multiplicative quantum number, and depends on the angular momentum.
    So the parity of the $\xi\Ppiplus$ system is
    \begin{equation}
        P_{\xi\Ppi} = (-1)^L P_{\xi} P_{\Ppi} = (-1)^{L+1} P_{\xi}.
    \end{equation}


    Finally, the charge-conjugation parity is undefined for charged particles, so the $\xi\Ppiplus$ system will have the same value of $C$ as the resonance's. 
    \begin{table}
        \centering
        \caption{$L$, $P$, and $C$ values for the $\xi\Ppiplus$ system in function of the $J^{PC}$ state of some resonances, $\xi$.
                 The resonance states are the ones reported in~\cite{chinese_phisics}.}
        \label{table:composite_resonance_system}
        \begin{tabular}{lccccccc}
            \toprule
            \multicolumn{4}{c}{$\xi$ resonance}   & &\multicolumn{3}{c}{$\xi\Ppiplus$ system} \\ \cline{1-4} \cline{6-8}
            Name   &$J$ &$P$ &$C$                 & &$L$ &$P$ &$C$\\
            \midrule
            \Pfnez{}    &$0$ &$+$ &$+$            & &$0$ &$-$ &$+$\\
            \Prhozero{} &$1$ &$-$ &$-$            & &$1$ &$-$ &$-$\\
            \Pfii{}     &$2$ &$+$ &$+$            & &$2$ &$-$ &$+$\\
            \bottomrule
        \end{tabular}
    \end{table}
    Table~\ref{table:composite_resonance_system} shows the values of $L$, $P$, and $C$ for the composite $\xi\Ppiplus$ system for different resonances, $\xi$.


    Having set the context of the isobar formalism, I will now discuss how to write down the explicit form of the contributions to the decay amplitude~\eqref{eq:isobar_decomposition}.
    Each isobar amplitude can be decomposed in the following terms:
    \begin{equation}\label{eq:isobar_amplitude}
        \Psi_I(\tau) = \psi_J(\tau)\,\mathcal{F\!}_{J}(\tau)\,\Delta_I(s)\,\A_{d_1}\!(m_{d_1\!}; \tau)\,\A_{d_2}\!(m_{d_2\!}; \tau).
    \end{equation}
    Here, $\psi_J$ is the spin-dependent part of the decay amplitude;
    $\mathcal{F\!}_J$ is the Blatt-Wei\ss{}kopf penetration factor;
    $\Delta_I$ is the dynamic shape of the isobar;
    and $\A_{d_1}$ and $\A_{d_2}$ are the amplitudes of the decays of $d_1$ and $d_2$, the daughter particles (or resonances) of the sub-decay equation~\eqref{eq:isobar_amplitude} refers to.
    These amplitudes have to be evaluated recursively by means of equation~\eqref{eq:isobar_decomposition}.
    Please note again that I am only allowing the decay to proceed via two-body sub-decays.


    In what follows, unless explicitly stated, I will imply the sum over the Bose-Einstein symmetrizations of the indistinguishable particles---\ie~the final-state \Ppiplus{} in the \PDplus{} decay under consideration.


    \paragraph{Spin-dependent amplitude}
    The spin-dependent amplitude, $\psi_J(\tau)$, describes the angular distribution of the decay and is fully specified by the spin quantum numbers of the initial-state particle, the isobar, and the final-state particles.
    There are a number of formalisms to parametrize $\psi_J(\tau)$.
    

    In the Zemach formalism\index{formalism!Zemach}\index{Zemach formalism}~\cite[\S~V.1]{PhysRev.140.B97}, the form of the decay angular dependence in terms of the final-state particles three-momenta is
    \begin{equation}\label{eq:zemach_formalism}
        \psi_{J}^{(\xi)c}(\vec{p}_a,\vec{p}_c) = \frac{J!}{(2J-1)!!}\,P_J(\vec{\hat{p}}_a \cdot \vec{\hat{p}}_c)\,\abs{\vec{p}_a}^J\abs{\vec{p}_c}^J,
    \end{equation}
    where the superscript $(\xi)c$ means that $c$ is the spectator particle;
    $P_J$ is the $J$-th Legendre polynomial, which only depends on the \emph{helicity angle}\index{helicity angle}, namely the angle between $a$'s and $c$'s momenta ($\vec{\hat{p}}_a$ and $\vec{\hat{p}}_c$ are unit vectors);
    and the three-momenta $\vec{p}_a$ and $\vec{p}_c$ are evaluated in the rest frame of the $ab$ system.
    Please note that, unlike the generic spin-dependent amplitude in equation~\eqref{eq:isobar_amplitude}, $\psi_J(\tau)$, which applies to a two-body decay, equation~\eqref{eq:zemach_formalism} applies to a three-body decay.
    \begin{table}
        \centering
        \caption{Expressions of the spin-dependent part of the isobar amplitude~\eqref{eq:isobar_amplitude} in the Zemach formalism for the lowest three integer values of $J$.
                 The three-momenta $\vec{p}_a$ and $\vec{p}_c$ are to be evaluated in the rest frame of the $ab$ subsystem.}
        \label{tab:zemach_formalism}
        
        \begin{tabular}{lc}
            \toprule
            $J$ &$\psi_J^{(ab)c}$\\
            \midrule
            $0$ &$1$ \\
            $1$ &$-2(\vec{p}_a\cdot\vec{p}_c)$ \\
            $2$ &$4(\vec{p}_a\cdot\vec{p}_c)^2 - 4(\abs{\vec{p}_a}\abs{\vec{p}_c})^2\!/3$\\
            \bottomrule
        \end{tabular}
    \end{table}
    Table~\ref{tab:zemach_formalism} shows the explicit expressions of $\psi_J$ in the Zemach formalism for $J\in\set{0,1,2}$.
    Equation~\eqref{eq:zemach_formalism} can also be expressed in a Lorentz-invariant form, which I will not report here.


    Another possible angular formalism is the helicity formalism\index{helicity formalism}\index{formalism!helicity}~\cite{jacob1959404}, which I will not discuss here.
    It is worth stressing that the physical prediction, $\Intensity(\tau)$, must be independent on the particular angular formalism one chooses to describe the decay.


    \paragraph{Blatt-Wei\ss{}kopf factors}

    \begin{table}
        \centering
        \caption{Expressions of the Blatt-Wei\ss{}kopf penetration factors for the lowest three integer values of $J$. To simplify the notation, I define $z\coloneqq R^2 q^2$.}
        \label{table:blatt_weisskopf}
        \begin{tabular}{lc}
            \toprule
            $J$ &$\mathcal{F}_{\!J}$\\
            \midrule
            $0$ &$1$ \\
            $1$ &$\bigg(\displaystyle\frac{2z}{z + 1}\bigg)^{1/2}$ \\
            $2$ &$\bigg(\displaystyle\frac{13 z^2 }{z^2 + 3z + 9}\bigg)^{1/2}$ \\
            \bottomrule
        \end{tabular}
    \end{table}
    Table~\ref{table:blatt_weisskopf} summarizes the explicit expressions of the Blatt-Wei\ss{}kopf\index{Blatt-Weisskopf@Blatt-Wei\ss{}kopf} barrier factors, $\mathcal{F}_{\!J}(q; R)$.
    These functions depend on the resonance \emph{radial size}\index{radial size}, $R$---which is a phenomenological parameter---, and on the break-up momentum\index{break-up momentum} of the daughter particles of the sub-decay they refer to, $q$.
    \begin{figure}
        \centering
        \begin{tikzpicture}
    \begin{axis}[
            xlabel={$s_{\pi\pi}$ [\si{\giga\electronvolt}]},
            legend style={at={(.66,.85)}, anchor=west},
        ]
        \pgfmathsetmacro{\mPi}{0.13957018}
        \pgfmathsetmacro{\mD}{1.86962}
        \pgfmathsetmacro{\r}{3.}

        \addplot [domain={\mPi + \mPi:\mD - \mPi}, samples=150]
                 gnuplot [raw gnuplot] {
                     q2_pi(x)   = (.5 * x)^2 - \mPi * \mPi;
                     zeta_pi(x) = \r * \r * q2_pi(x);
                     plot [(2 * \mPi):(\mD - \mPi)] zeta_pi(x);
                 }; \addlegendentry{$R^2q_{\pi\pi}^2$}

        \addplot [domain={\mPi + \mPi:\mD - \mPi}, samples=150, densely dashed]
                 gnuplot [raw gnuplot] {
                     q2_D(x)   = .25 * (\mD * \mD - (x - \mPi) ** 2) * (\mD * \mD - (x + \mPi) ** 2) / (\mD * \mD);
                     zeta_D(x) = \r * \r * q2_D(x);
                     plot [(2 * \mPi):(\mD - \mPi)] zeta_D(x);
                 }; \addlegendentry{$R^2q_{\xi\pi}^2$}
    \end{axis}
\end{tikzpicture}

        \caption{Squared break-up momenta of the $\Ppiplus$ and $\Ppiminus$ in the $\xi$ decay, and of the $\xi$ and $\Ppiplus$ in the $\PDplus$ decay.}
        \label{fig:break_up_momenta}
    \end{figure}
    Figure~\ref{fig:break_up_momenta} shows the plot of the break-up momenta for the $\PDplus\to\Ppiplus\Ppiminus\Ppiplus$ decay:
    \begin{equation}
        q_{\pi\pi}^2(s) = \bigg(\frac{s}{2}\bigg)^2 - m_{\pi}^2,
        \text{ and }
        q_{\xi\pi}^2(s) = \frac{[m_{\PD}^2 - (s - m_\pi)^2][m_{\PD}^2 - (s+m_\pi)^2]}{4m_{\PD{}}^2},
    \end{equation}
    where $s \coloneqq p_{\pi\pi}^2$ is the measured squared mass of the intermediate state, $\xi$.
    \begin{figure}
        \centering
        \subfloat[][P wave.]{\begin{tikzpicture}
    \begin{axis}[
            legend style={at={(.92,.71)}, anchor=west},
            xlabel={$s$ [\si{\giga\electronvolt}]},
        ]
        \pgfmathsetmacro{\mPi}{0.13957018}
        \pgfmathsetmacro{\mD}{1.86962}
        \pgfmathsetmacro{\r}{3.}

        \addplot [domain={\mPi + \mPi:\mD - \mPi}, smooth]
                 gnuplot [raw gnuplot] {
                     set samples 300;
                     q2_pi(x)   = (.5 * x)^2 - \mPi * \mPi;
                     zeta_pi(x) = \r * \r * q2_pi(x);
                     q2_D(x)   = .25 * (\mD * \mD - (x - \mPi) ** 2) * (\mD * \mD - (x + \mPi) ** 2) / (\mD * \mD);
                     zeta_D(x) = \r * \r * q2_D(x);
                     BW(x) = sqrt(2 * x / (1 + x));
                     plot [(2 * \mPi):(\mD - \mPi)] BW(zeta_pi(x)) * BW(zeta_D(x));
                 }; \addlegendentry{$\mathcal{F}_1^{\PD{}}\!(s)\,\mathcal{F}_1^{\xi}(s)$}

        \addplot [domain={\mPi + \mPi:\mD - \mPi}, smooth, densely dashed]
                 gnuplot [raw gnuplot] {
                     set samples 300;
                     q2_pi(x)   = (.5 * x)^2 - \mPi * \mPi;
                     zeta_pi(x) = \r * \r * q2_pi(x);
                     q2_D(x)   = .25 * (\mD * \mD - (x - \mPi) ** 2) * (\mD * \mD - (x + \mPi) ** 2) / (\mD * \mD);
                     zeta_D(x) = \r * \r * q2_D(x);
                     BW(x) = sqrt(2 * x / (1 + x));
                     plot [(2 * \mPi):(\mD - \mPi)] BW(zeta_D(x));
                 }; \addlegendentry{$\mathcal{F}_1^{\PD{}}\!(s)$}

        \addplot [domain={\mPi + \mPi:\mD - \mPi}, smooth, densely dotted]
                 gnuplot [raw gnuplot] {
                     set samples 300;
                     q2_pi(x)   = (.5 * x)^2 - \mPi * \mPi;
                     zeta_pi(x) = \r * \r * q2_pi(x);
                     q2_D(x)   = .25 * (\mD * \mD - (x - \mPi) ** 2) * (\mD * \mD - (x + \mPi) ** 2) / (\mD * \mD);
                     zeta_D(x) = \r * \r * q2_D(x);
                     BW(x) = sqrt(2 * x / (1 + x));
                     plot [(2 * \mPi):(\mD - \mPi)] BW(zeta_pi(x));
                 }; \addlegendentry{$\mathcal{F}_1^{\xi}(s)$}

    \end{axis}
\end{tikzpicture}

}

        \subfloat[][D wave.]{\begin{tikzpicture}
    \begin{axis}[
            legend style={at={(.89,.85)}, anchor=west},
            xlabel={$s$ [\si{\giga\electronvolt}]},
        ]
        \pgfmathsetmacro{\mPi}{0.13957018}
        \pgfmathsetmacro{\mD}{1.86962}
        \pgfmathsetmacro{\r}{3.}

        \addplot [domain={\mPi + \mPi:\mD - \mPi}, smooth]
                 gnuplot [raw gnuplot] {
                     set samples 300;
                     q2_pi(x)   = (.5 * x)^2 - \mPi * \mPi;
                     zeta_pi(x) = \r * \r * q2_pi(x);
                     q2_D(x)   = .25 * (\mD * \mD - (x - \mPi) ** 2) * (\mD * \mD - (x + \mPi) ** 2) / (\mD * \mD);
                     zeta_D(x) = \r * \r * q2_D(x);
                     BW(x) = sqrt(13 * x * x / (9 + 3 * x + x * x));
                     plot [(2 * \mPi):(\mD - \mPi)] BW(zeta_pi(x)) * BW(zeta_D(x));
                 }; \addlegendentry{$\mathcal{F}_2^{\PD{}}\!(s)\,\mathcal{F}_2^{\xi}(s)$}

        \addplot [domain={\mPi + \mPi:\mD - \mPi}, smooth, densely dashed]
                 gnuplot [raw gnuplot] {
                     set samples 300;
                     q2_pi(x)   = (.5 * x)^2 - \mPi * \mPi;
                     zeta_pi(x) = \r * \r * q2_pi(x);
                     q2_D(x)   = .25 * (\mD * \mD - (x - \mPi) ** 2) * (\mD * \mD - (x + \mPi) ** 2) / (\mD * \mD);
                     zeta_D(x) = \r * \r * q2_D(x);
                     BW(x) = sqrt(13 * x * x / (9 + 3 * x + x * x));
                     plot [(2 * \mPi):(\mD - \mPi)] BW(zeta_D(x));
                 }; \addlegendentry{$\mathcal{F}_2^{\PD{}}\!(s)$}

        \addplot [domain={\mPi + \mPi:\mD - \mPi}, smooth, densely dotted]
                 gnuplot [raw gnuplot] {
                     set samples 300;
                     q2_pi(x)   = (.5 * x)^2 - \mPi * \mPi;
                     zeta_pi(x) = \r * \r * q2_pi(x);
                     q2_D(x)   = .25 * (\mD * \mD - (x - \mPi) ** 2) * (\mD * \mD - (x + \mPi) ** 2) / (\mD * \mD);
                     zeta_D(x) = \r * \r * q2_D(x);
                     BW(x) = sqrt(13 * x * x / (9 + 3 * x + x * x));
                     plot [(2 * \mPi):(\mD - \mPi)] BW(zeta_pi(x));
                 }; \addlegendentry{$\mathcal{F}_2^{\xi}(s)$}
    \end{axis}
\end{tikzpicture}
}

        \caption{Blatt-Wei\ss{}kopf factors for the P and D waves of the $\PDplus\to\Ppiplus\Ppiminus\Ppiplus$ decay.}
        \label{fig:blatt_weisskopf}
    \end{figure}
    Figure~\ref{fig:blatt_weisskopf} shows the Blatt-Wei\ss{}kopf factors for the same decay.


    Finally, figure~\ref{fig:dalitz_angular_parts} shows the plots of $\psi_J^{(\xi)\pi}\!(\tau)\,\mathcal{F}_J^{\PD}\!(s)\,\mathcal{F}_J^{\xi}(s)$ in the P and D waves on the phase space of the $\PDplus\to\Ppiplus\Ppiminus\Ppiplus$ decay.

    \paragraph{Dynamic shape}
    Unlike the spin-dependent part of the isobar amplitude and the Blatt-Wei\ss{}kopf factors, there is no way to derive an explicit expression for the dynamic shape from first principles.
    In the past, physicists have mostly adopted a heuristic model-dependent description to obtain the form of the dynamic shape.
    This approach is described in the following section; a model-independent approach is described in section~\ref{sec:model_independent_isobar_decomposition}.


    \subsection{Model-dependent isobar decomposition}



    In the model-dependent approach to the isobar decomposition, the dynamic shape of the decay amplitude is expanded in terms of the resonances that populate the isobar:
    \begin{equation}\label{eq:isobar_mass_shape_expansion}
        \Delta_I(s) = \sum_{\xi\in I} \alpha_{\xi}\Delta_{\xi}(s),
    \end{equation}
    being $s$ the squared mass of the resonance, \ie~the squared four-momentum of the resonance daughter-particle system; referring to the figure~\ref{fig:isobar_three_body_decay}, $s\coloneqq p_{ab}^2 = p_{\xi}^2$.
    Each parameter $\alpha_{\xi}$ is the complex weight of the corresponding resonance $\xi$.


    \begin{table}
        \centering
        \caption{Expressions of the Blatt-Wei\ss{}kopf penetration factors for the first three integer values of $J$. To simplify the notation, I define $z\coloneqq R^2 q_{ab}^2$.}
        \label{table:blatt_weisskopf}
        \begin{tabular}{lc}
            \toprule
            $J$ &$\mathcal{F}_{\!J}$\\
            \midrule
            $0$ &$1$ \\
            $1$ &$\bigg(\displaystyle\frac{2z}{z + 1}\bigg)^{1/2}$ \\
            $2$ &$\bigg(\displaystyle\frac{13 z^2 }{z^2 + 3z + 9}\bigg)^{1/2}$ \\
            \bottomrule
        \end{tabular}
    \end{table}
    One of the most common forms for the dynamic shape of a resonance is the relativistic Breit-Wigner function\index{relativistic Breit-Wigner}\index{Breit-Wigner!relativistic}
    \begin{equation}\label{eq:rbw}
        \Delta_{\xi}^{\text{RBW}}(s) \coloneqq \frac{1}{m_{\xi}^2 - s - \iu \sqrt{s}\, \Gamma_\xi(s)},
    \end{equation}
    being $m_{\xi}$ the nominal mass of the resonance.
    The mass-dependent width reads
    \begin{equation}
        \Gamma_\xi(s) \coloneqq \Gamma_{\xi} \, \mathcal{F}_{\!J_\xi}\!(q_{ab};R)\, \frac{m_{\xi}}{\sqrt{s}} \bigg(\frac{q_{ab}}{q_{\xi}}\bigg)^{2J_{\xi}+1},
    \end{equation}
    where $\Gamma_{\xi}$ and $J_{\xi}$ are the width and spin of the resonance;
    $\mathcal{F}$ is the Blatt-Wei\ss{}kopf form factor (see the table~\ref{table:blatt_weisskopf}), which depends on $R$, the \emph{radial size}\index{radial size};
    $q_{ab}$ is the measured break-up momentum;
    and $q_\xi$ is the break-up momentum at the nominal mass of $\xi$.


    If the resonance is narrow and all the relevant thresholds are far away, the term $\sqrt{s}\,\Gamma_\xi(s)$ in the denominator of the equation~\eqref{eq:rbw} can be replaced with the constant quantity $m_{\xi}\Gamma_{\xi}$~\cite[\S~47.2.1]{chinese_phisics}.
    With this substitution, one gets the non-relativistic Breit-Wigner\index{Breit-Wigner}
    \begin{equation}\label{eq:bw}
        \Delta_{\xi}^{\text{BW}}(s) \coloneqq \frac{1}{m_{\xi}^2 - s - \iu m_{\xi} \Gamma_{\xi}},
    \end{equation}
    which normalized such that $\Delta_{\xi}^{\text{BW}}(m_{\xi}^2) = \iu / m_\xi\Gamma_\xi$.


    Another possible dynamic shape is the pole-mass distribution
    \begin{equation}
        \Delta_{\xi}^{\text{PM}}(s) \coloneqq \frac{1}{m_{\xi}^2 - s},
    \end{equation}
    in which the nominal mass of the resonance is a complex number $m_\xi = \Re (m_\xi) + \iu\Im(m_\xi)$.
    \citeauthor{PhysRevD.71.054030}, for example, proposes the pole-mass shape as a parametrization for the \Psigma{} resonance~\cite{PhysRevD.71.054030}.
    \begin{figure}
        \centering
        \begin{tikzpicture}
    \begin{axis}[
            ylabel={Magnitude $[\si{1/(\giga\electronvolt\per\c^2)^2}]$},
            xlabel={$s$ $[\si{\giga\electronvolt^2\per\c^4}]$},
        ]
        \pgfmathsetmacro{\m}{.98}
        \pgfmathsetmacro{\T}{.07}
        \addplot [domain={.279:1.73}, samples=100, smooth] gnuplot {1./sqrt((\m * \m - x * x)**2 + (\m * \T)**2)};\addlegendentry{Breit-Wigner}

        \addplot [domain={.279:1.73}, samples=100, smooth, densely dashed]
                 gnuplot {1./sqrt((\m * \m - \T * \T - x * x)**2 + 4 * (\m * \T)**2)};\addlegendentry{Pole mass}
    \end{axis}
\end{tikzpicture}

        \caption{Comparison between the non-relativistic Breit-Wigner and the pole-mass dynamic shapes. The parameters I used are $(\SI{.98}{\giga\electronvolt/\c^2}, \SI{.07}{\giga\electronvolt/\c^2})$, corresponding to $(m_\xi,\Gamma_\xi)$ for the Breit-Wigner shape; and $(\Re(m_\xi),\Im(m_\xi))$ for the pole-mass shape.}

        \label{fig:bw_pm_comparison}
    \end{figure}
    The figure~\ref{fig:bw_pm_comparison} shows a comparison between the non-relativistic Breit-Wigner and the pole-mass dynamic shapes.

    {\color{red} Other parametrizations are the Flatté and the pole-mass.

    \begin{equation}
        \Delta_{\xi}^{\text{F}}(s) \coloneqq \frac{g_1}{m_{\xi}^2 - s - \iu(\rho_1 g_1^2 + \rho_2 g_2^2)}.
    \end{equation}
    Where the mass of the resonance $\xi$ is complex.
    }

    In the model-dependent isobar decomposition, the decay amplitude~\eqref{eq:isobar_decomposition} finally reads\marginpar{more efficient for caching, but wrong with my conventions}
    \begin{equation}\label{eq:redundant_model_dependent_isobar_decomposition}
        \Psi_I(\tau) =  \sum_{\xi\in I} \gamma_\xi \,\psi_{J_\xi}\!(\tau)\,\Delta_{\xi}(s),\quad
        \text{being }
        \gamma_\xi\coloneqq c_I \alpha_{\xi}.
    \end{equation}
%    \begin{equation}\label{eq:redundant_model_dependent_isobar_decomposition}
%        \A(\tau) = \sum_{I\in\set{(J, P, C)}} \psi_I(\tau) \sum_{\xi\in I} \gamma_\xi \Delta_{\xi}(s),\quad
%        \text{being }
%        \gamma_\xi\coloneqq c_I \alpha_{\xi}.
%    \end{equation}
    I have absorbed the coefficients $c_I$ and $\alpha_\xi$ into one coefficient $\gamma_{\xi}$, that I will call \emph{free amplitude}\index{free amplitude}.
    Please note that the free amplitude $\gamma_\xi$ in the equation~\eqref{eq:redundant_model_dependent_isobar_decomposition} does not need the isobar label: since each resonance is in a well-defined isobar, a double index like $\gamma_{I,\xi}$ would be redundant.
    For the same reason, the common angular part in the equation~\eqref{eq:redundant_model_dependent_isobar_decomposition} can be factored out of the sum:
    \begin{equation}\label{eq:non_redundant_model_dependent_isobar_decomposition}
        \Psi_I(\tau) =  \psi_{J_I}\!(\tau)\sum_{\xi\in I} \gamma_\xi \,\Delta_{\xi}(s).
    \end{equation}
    {\color{red} TODO point out that one is more suited for smart caching.}




    At this point, I would like to stress that in the expansion~\eqref{eq:non_redundant_model_dependent_isobar_decomposition} there are several sources of systematic uncertainties.
    The assumption about the resonance content of each isobar is not known and highly affects the quality of the model description. {\color{red} cite \cite{PhysRevD.76.012001}}\marginpar{some resonances are not yet well-established}
    The dynamic shape of each resonance---along with its parameters---has to be experimentally determined; so it has to be silently assumed that neither the dynamic shape of the resonance, nor its parameters are affected by the particular interaction in the decay.


    The limitations of the model-dependent isobar formalism are becoming critical due to the availability of larger and larger data sets.
    In the next section I will present how such limitations can be circumvented by means of the model-independent isobar formalism.


    {\color{red}
    Three-body heavy-meson decays have been analyzed via Dalitz plots.
    The isobars are decomposed in terms of the contributing resonances and the quality of the Dalitz-plot description is maximized by the appropriate choice of the intermediate-resonance set.
    An example is $ \PDplus\to\xi\Ppiplus, \xi\to\Ppiplus\Ppiminus$, with many possible resonances $\xi$.
    }




{\color{red}
It's difficult to find a heuristic model for the S wave, since it is the most populated.
For this, we implement a model-independent description in \ac{yap}.
}

    {\color{red}

    The accuracy on the parameters of resonances strongly depends on the quality of the decay model that has been adopted to perform the fit.
    In particular, the isobar formalism has some limits that must be overcome if the decay description is to be improved.
    The knowledge of isobars comes from past experiments that had to determine mass and width of intermediate resonances.
    This means that the resonance parameters (and, sometimes, shapes as in the case of ?) are only known with limited precision, thus introducing systematic effects in the decay modelling.
    }

    \subsection{Model-independent isobar decomposition}
\label{sec:model_independent_isobar_decomposition}

    The model-independent approach to the isobar decomposition aims to reduce the analysis' dependence on possibly incomplete or incorrect fit models, as it does not need any assumption on the resonant content of the isobars.
    In this approach, the dynamic shapes of the isobars are directly extracted from the available experimental events, provided the data set is large enough.
    

    A characteristic function\index{characteristic function} on a real interval, $M \coloneqq [m_{\text{low}}, m_{\text{up}})$, is defined as
    \begin{equation}
        \Characteristic_M(m) \coloneqq 
        \begin{cases}
            1 &\text{if }m \in M, \\
            0 &\text{otherwise}.
        \end{cases}
    \end{equation}
    Let $\mathcal{P}(M) \coloneqq \set{m_0, m_1,\dots, m_{N_\text{bins}}}$ be a partition of $M$---\ie~a set of points such that $m_{i} < m_{i+1}$; and whose extremes coincide with those of $M$, namely $(m_0, m_{N_\text{bins}}) = (m_{\text{low}}, m_{\text{up}})$.
    \begin{figure}
        \centering
        \begin{tikzpicture}
    \pgfmathtruncatemacro{\a}{3}
    \pgfmathtruncatemacro{\b}{1}
    \pgfmathtruncatemacro{\c}{2}

    \begin{scope}[color=black!50]
        \coordinate (o) at (-1, 0);
        \draw (o) -- +(\a + 2, 0) coordinate (a);
        \draw [densely dashed] (a) -- +(\b, 0) coordinate (b);
        \draw [->] (b) -- +(\c, 0) coordinate (c);
    \end{scope}

    \begin{scope}
        % vertical label shift
        \pgfmathsetmacro{\s}{-.5}

        \foreach \x in {0,1,...,\a} {
            \node [draw, inner sep=1pt, circle, fill=black] (x\x) at (\x, 0) {};
            \node (sx\x) at ($(x\x) + (0, \s)$)  {$m_\x$};
        }

        \node [draw, inner sep=1pt, circle, fill=black] (last) at (\a + \b + 2, 0) {};
        \node (slast) at ($(last) + (0, \s)$) {$m_{N_\text{bins}}$};

    \end{scope}

    \node () at (x0) [label={above:$m_a + m_b$}] {};
    \node () at (last) [label={above:$m_X - m_c$}] {};


\end{tikzpicture}

        \caption[Partition of the invariant-mass range into $N_\text{bins}$ right-open bins.]%
        {Partition of the invariant-mass range, $M \coloneqq [m_a+m_b,m_X - m_c)$, into $N_\text{bins}$ right-open bins, $B_i = [m_{i-1}, m_i)$.}
        \label{fig:invariant-mass-partition}
    \end{figure}
    Figure~\ref{fig:invariant-mass-partition} shows the partitioning of the interval $[m_a+m_b,m_X-m_c)$, the allowed invariant-mass range for the final-state $ab$ system.\footnote{The allowed mass range of the $ab$ system also includes the value $m_X - m_c$; I will anyway ignore this value as it is a null-measure interval, so leaving it out will not affect the results of the analysis.}
    The partition $\mathcal{P}(M)$ defines $N_\text{bins}$ bins on $M$, $\set{B_i}$, where $B_i$ is the right-open interval $[m_{i-1},m_i)$ for $i\in\set{1,\dots, N_\text{bins}}$.
    A step function\index{step function} on the interval $M$, given the partition $\mathcal{P}(M)$ is a finite linear combination of characteristic functions on the bins:
    \begin{equation}\label{eq:step_dynamic_shape}
        \Delta_{\mathcal{P}(M)}^{\text{SF}}(m) \coloneqq \sum_{i = 1}^{N_\text{bins}} \alpha_i\, \Characteristic_{B_i}\!(m),
    \end{equation}
    $\alpha_i$ being a complex coefficient.


    In the model-independent isobar decomposition, equation~\eqref{eq:step_dynamic_shape} is the approximate dynamic shape of the isobar, and its coefficients are determined from the analysis of the observed events.
    \begin{figure}
        \centering
        \begin{tikzpicture}
    \begin{axis}[
        amplitude_plot,
        width = .8\textwidth,
        xlabel={$s$ $[\si{\giga\electronvolt^2\per\c^4}]$},
    ]
        \pgfmathsetmacro{\m}{.98}
        \pgfmathsetmacro{\T}{.07}
        \addplot+ [guess, domain={.279:1.73}, samples=100, smooth] gnuplot {1./sqrt((\m * \m - x * x)**2 + (\m * \T)**2)};
        \addplot+ [fit, mark=none] table {data/binned_breit-wigner.txt};
    \end{axis}
\end{tikzpicture}

        \caption{Step-function approximation of the magnitude of the non-relativistic Breit-Wigner dynamic shape, equation~\eqref{eq:bw}.}
        \label{fig:step_function_approximation}
    \end{figure}
    Figure~\ref{fig:step_function_approximation} shows the basic idea behind the model-independent isobar decomposition: the piecewise function in equation~\eqref{eq:step_dynamic_shape} will approximate the isobar mass shape.


    With the piecewise dynamic shape, the isobar amplitude reads
    \begin{equation}\label{eq:freed_wave_amplitude}
        \Psi_I(\tau) = \psi_{J_I\!}(\tau)\,\mathcal{F\!}_{J_I\!}(\tau)\,\A_{d_1}\!(m_{d_1\!}; \tau)\,\A_{d_2}\!(m_{d_2\!}; \tau) \sum_{i=1}^{N_\text{bins}} \gamma_{(I,i)}\, \Characteristic_{B_i}\!(s),
        \quad
        \text{with }
        \gamma_{(I,i)} \coloneqq c_I \alpha_i. 
    \end{equation}
    The free amplitudes, $\gamma_{(I,i)}$, now need to have an isobar label too.
    I will call a wave whose decay amplitude is written in the form of equation~\eqref{eq:freed_wave_amplitude} \emph{freed wave}\index{freed wave}.


    At this point it is worth noting that, when fitting the decay model to the data, the piecewise description of the dynamic shape introduces $N_\text{bins}$ complex \acp{dof}---for a total of $2N_{\text{bins}}$ real \acp{dof}.
    This drastically increases the computational complexity of the fit.


    To contain the number of fit \acp{dof}, it is possible to mix the model-dependent and model-independent \ac{pwa} approaches: the isobars with well-known resonances may be described in a fixed way, while the other isobars may be freed.
    The \focus{} collaboration performed a \ac{pwa} of the $\PDplus \to \PKminus\Ppiplus\Ppiplus$ decay employing a fixed description of the P- and D-wave components of the $\PKminus\Ppiplus$ decay amplitude (described as a sum of Breit-Wigner distributions), and a model-independent description of the S-wave component~\cite{Link200914}.


    In chapter~\ref{chap:model_independent_pwa}, I exploit the information about the resonant content of the \ac{mc} data to make the bins finer in the surroundings of the resonance peaks.
    A non-uniform optimized partitioning of the mass range speeds up the fit convergence.
    Unfortunately, the information about a wave's resonant content is not always available.


        \subsubsection{Zero modes}
        Introduction of the isobar-freed fit leads to linear dipendencies among the waves in the fit model.
This can be seen by means of the integral matrix (now $w$ and $v$ are wave indices)
\begin{equation}
    I_{wv} \coloneqq \int_\Omega \psi_w^*(\tau)\,\psi_v(\tau)\ud\tau
\end{equation}
which represents the integrated overlap of the waves $w$ and $v$ over the phase space.
Since $I$ is hermitian, its eigenvalues are real.
So
\begin{equation}
    I^{\text{Tot}} \coloneqq \int_\Omega \abs{\A(\tau)}^2\!\ud \tau = \sum_{v,w \in \Set{\text{waves}}} T_w^* I_{wv} T_v.
\end{equation}


    {\color{red}
    (Quote Fabian's work as unpublished?)
    }


    \section{Dalitz-plot analysis}

    In a two-body decay, the energy-momentum conservation fully determines the energy of the final-state particles in the center-of-mass frame.
    In a three-body decay, after requiring energy and momentum conservation, there are still five remaining \acp{dof}.
    If the initial- and final-state particles are spinless, up to an arbitrary rotation of the decay plane, two \acp{dof} remain.
    Therefore, for the decay under consideration, the phase-space coordinate, $\tau$, is a two-element vector.

   
   There are several possible choices for the observable pair to parametrize $\tau$.
    It is thus convenient to choose a couple of observables such that the phase-space measure is constant within the kinematically-allowed region they span.
    Two such observables are the squared invariant masses of two pairs of final-state particles, i.e.~$m_{ab}^2 = p_{ab}^2$ and $m_{bc}^2 = p_{bc}^2$, with $a$, $b$, and $c$ the labels for the final-state particles (see figure~\ref{fig:isobar_three_body_decay}).


    With this parametrization of the phase-space coordinate, the differential decay probability is~\cite[\S~13.1.1]{Bevan:2014iga}
    \begin{equation}
        \ud\Gamma = \frac{1}{(2\pi)^3}\frac{1}{32 m_X^3}\abs{\A(m_{ab}^2, m_{bc}^2)}^2\!
        \ud m_{ab}^2 \!\ud m_{bc}^2.
    \end{equation}
    As the phase-space measure is a constant function throughout the whole phase space, any non-uniformity in the distribution of the events is to be ascribed to the decay amplitude only.
    \begin{figure}
        \centering
        \begin{tikzpicture}
    \begin{axis} [
        width = .7\textwidth,
        view={0}{90},%
        xlabel={$m_{bc}^2$},
        ylabel={$m_{ab}^2$},
        ticks=none,
        grid=none
    ]
        \addplot3 [dalitz]
          gnuplot [raw gnuplot] { splot "data/resonance_example/s_wave_resonance_mcmc.txt" using 1:2:3 };
    \end{axis}
\end{tikzpicture}

        \caption{S-wave resonance in the $ab$ channel.}
        \label{fig:s_wave_resonance_example}
    \end{figure}
    \begin{figure}
        \centering
        \begin{tikzpicture}
    \begin{axis} [
        width = .7\textwidth,
        view={0}{90},%
        xlabel={$m_{bc}^2$},
        ylabel={$m_{ab}^2$},
        ticks=none,
        grid=none
    ]
        \addplot3 [dalitz]
          gnuplot [raw gnuplot] { splot "data/resonance_example/p_wave_resonance_mcmc.txt" using 1:2:3 };
    \end{axis}
\end{tikzpicture}

        \caption{P-wave resonance in the $ab$ channel.}
        \label{fig:p_wave_resonance_example}
    \end{figure}
    \begin{figure}
        \centering
        \begin{tikzpicture}
    \begin{axis} [
        width = .7\textwidth,
        view={0}{90},%
        xlabel={$m_{bc}^2$},
        ylabel={$m_{ab}^2$},
        ticks=none,
        grid=none
    ]
        \addplot3 [dalitz]
          gnuplot [raw gnuplot] { splot "data/resonance_example/d_wave_resonance_mcmc.txt" using 1:2:3 };
    \end{axis}
\end{tikzpicture}

        \caption{D-wave resonance in the $ab$ channel.}
        \label{fig:d_wave_resonance_example}
    \end{figure}
    The resonances appear as bands within the allowed region, as shown in Dalitz plots~\ref{fig:s_wave_resonance_example}, \ref{fig:p_wave_resonance_example}, and \ref{fig:d_wave_resonance_example}.


    \begin{figure}
        \centering
        \begin{tikzpicture}
%    \pgfmathsetmacro{\mTwoA}{(.13957)^2} % pi+ mass squared [GeV/c^2]
%    \pgfmathsetmacro{\mTwoB}{(.13957)^2} % pi- mass squared [GeV/c^2d
%    \pgfmathsetmacro{\mTwoC}{(.13957)^2} % pi+ mass squared [GeV/c^2d
%    \pgfmathsetmacro{\mTwo}{(1.86963)^2} % D+  mass squared [GeV/c^2d      

    \pgfmathsetmacro{\mTwoA}{(.17)^2}
    \pgfmathsetmacro{\mTwoB}{(.17)^2}
    \pgfmathsetmacro{\mTwoC}{(.17)^2}
    \pgfmathsetmacro{\mTwo}{(1.)^2}

    \pgfmathsetmacro{\lBound}{(sqrt(\mTwoA) + sqrt(\mTwoB))^2}
    \pgfmathsetmacro{\uBound}{(sqrt(\mTwo)  - sqrt(\mTwoC))^2}
    \pgfmathsetmacro{\lyBound}{(sqrt(\mTwoC) + sqrt(\mTwoB))^2}
    \pgfmathsetmacro{\uyBound}{(sqrt(\mTwo)  - sqrt(\mTwoA))^2}

    \begin{axis}[
        ylabel={$m_{ab}^2$},
        xlabel={$m_{bc}^2$},
        ytick={\lBound,\uBound},
        yticklabels={$(m_a + m_b)^2$, $(M-m_c)^2$},
        xtick={\lyBound,\uyBound},
        xticklabels={$(m_b + m_c)^2$, $(M-m_a)^2$},
        yticklabel style={sloped like y axis},
        grid = major,
        declare function = { E_b(\t) = ((\t -\mTwoB + \mTwoA)/(2*sqrt(\t))); },
        declare function = { P_b(\t) = sqrt(E_b(\t)^2 -\mTwoB); },
        declare function = { E_c(\t) = ((\mTwo -\t -\mTwoC)/(2*sqrt(\t))); },
        declare function = { P_c(\t) = sqrt(E_c(\t)^2 -\mTwoC); },
        enlargelimits=.12,
    ]

        \addplot[
            name path = A,
            black,
            opacity=0,
            domain = {\lBound:\uBound},
            samples=250,
        ] {(E_b(x) + E_c(x))^2 - (P_b(x) - P_c(x))^2};

        \addplot[
            name path = B,
            black,
            opacity=0,
            domain = {\lBound:\uBound},
            samples=250,
        ] {(E_b(x) + E_c(x))^2 - (P_b(x) + P_c(x))^2};

        \addplot[blue!50, opacity=.3] fill between[of=A and B];

    \end{axis}
\end{tikzpicture}

        \caption{Kinematically-allowed region in the phase space of a three-body decay. The corners correspond to the configuration in which one of the daughter particles is produced at rest (in the frame of the parent particle).}
        \label{fig:dalitz_kinematically_allowed}
    \end{figure}
    The boundaries of the kinematically-allowed phase-space region, shown in figure~\ref{fig:dalitz_kinematically_allowed}, can be expressed in terms of a constraint on $m_{bc}^2$ as follows:
    \begin{equation}
        \begin{aligned}
            (m_{bc}^2)_{\text{max}} &= (E_b + E_c)^2 - (\abs{\vec{p}_b} - \abs{\vec{p}_c})^2,\\
            (m_{bc}^2)_{\text{min}} &= (E_b + E_c)^2 - (\abs{\vec{p}_b} + \abs{\vec{p}_c})^2,
        \end{aligned}
    \end{equation}
    where
    \begin{equation}
        E_b = \frac{m_{ab}^2 - m_a^2 + m_b^2}{2m_{ab}},\text{ and }
        E_c = \frac{m_X^2 - m_{ab}^2 - m_c^2}{2 m_{ab}}
    \end{equation}
    are the energies of the $b$ and $c$ final-state particles in the rest frame of the $ab$ system, and
    \begin{equation}
        \abs{\vec{p}_b} = \sqrt{E_b^2 - m_b^2},\text{ and }
        \abs{\vec{p}_c} = \sqrt{E_c^2 - m_c^2}
    \end{equation}
    the corresponding 3-momenta magnitudes.
    The shape of the phase space, in figure~\ref{fig:dalitz_kinematically_allowed}, depends on the values of the final-state-particle masses:
    if $m_a = m_b = m_c = 0$, it will stretch to a triangle; whereas, if $m_a + m_b + m_c = m_X$, it will shrink to a point---thus making the decay impossible.

    \begin{figure}
        \centering

        \subfloat[]%
                 [P wave.]%
                 {\begin{tikzpicture}
    \begin{axis} [dalitz_plot]

        \addplot3 [dalitz]
          gnuplot [raw gnuplot] { splot "data/angular_part/uniform_p_wave_mcmc.txt" using 1:2:3 };
    \end{axis}
\end{tikzpicture}
}

        \subfloat[]%
                 [D wave.]%
                 {\begin{tikzpicture}
    \begin{axis} [dalitz_plot]

        \addplot3 [dalitz]
          gnuplot [raw gnuplot] { splot "data/angular_part/uniform_d_wave_mcmc.txt" using 1:2:3 };
    \end{axis}
\end{tikzpicture}
}

        \caption{Angular part of the decay amplitude (spin part in the Zemach formalism times Blatt-Wei\ss{}kopf factor) for the P and D waves.}
        \label{fig:dalitz_angular_parts}
    \end{figure}


    \section{Summary}

    In analyses of experimental data, one has to take into account the events coming from non-resonant interactions.
    This is done by summing an amplitude term to the decay amplitude~\eqref{eq:isobar_decomposition}:
    \begin{equation}
        \A_{\text{data}}(\tau) = \A(m_X;\tau) + a_{\text{NR}} \A_{\text{NR}}(\tau),
    \end{equation}
    where $a_{\text{NR}}$ is a complex coefficient, and $\A_{\text{NR}}$ a Lorentz-invariant function of the phase-space coordinate.
    Moreover, also the detector acceptance\index{acceptance} has to be taken into account.
    This is usually done by multiplying a function of the phase-space coordinate, $\epsilon(\tau)$, to the data intensity:
    \begin{equation}
        \Intensity_{\text{exp}}(\tau) = \epsilon(\tau)\,\abs{\A_{\text{data}}(\tau)}^2.
    \end{equation}
    In what follows, I will ignore the non-resonant contribution to the decay amplitude and the detector effects as I will only work on \ac{mc}-generated data.


    The decay amplitude that I will use in the rest of the thesis is equation~\eqref{eq:isobar_decomposition}, which has the following expansion:
    \begin{equation}
        \A(m_{\pi\pi}^2, m_{\pi\pi}^2) = \sum_{I} c_I \,\psi_{J_I}^{(\xi)\pi}\!(m_{\pi\pi}^2, m_{\pi\pi}^2)\, \mathcal{F}_{\!J_I}^{\PD}(q_{\xi\pi}^2(m_{\pi\pi}^2);R) \, \mathcal{F}_{\!J_I}^{\xi}(q_{\pi\pi}^2(m_{\pi\pi}^2);R)\,\Delta_I(m_{\pi\pi}^2).
    \end{equation}
    Please note that there is only one spin-dependent function for two sub-decays---namely, $\PDplus\to\xi\Ppiplus$ and $\xi\to\Ppiplus\Ppiminus$---because the functions~\eqref{eq:zemach_formalism} apply to three-body decays.
    Moreover, since the pions are spinless particles, their Blatt-Wei\ss{}kop factors are constant; and, since they are final-state particles, their dynamic shape is $\delta(s_{\pi}-m_{\pi}^2)$.

\chapter{Introduction}
\label{chap:introduction}

    Hadronic heavy-meson decays into multi-body final states are a great source of information for charge-conjugation and parity violation studies.

    So far, the model-dependent approach to the \ac{pwa} has been the standard analysis technique for light- and heavy-meson decays.
    The success of this technique in providing a good description of the decay strongly depends on the quality of the decay model, \ie~on the correct assumption about the resonant and non-resonant content of the waves and its parametrization.


    Increasingly large experimental data sets are collected at both collider and fixed-target experiments.
    On the one hand, this led to the discovery of new bound states and gave physicists the possibility to test very refined models.
    On the other hand, the model-dependent approach to \ac{pwa} is now proving limited as the systematic uncertainties introduced by incomplete or incorrect decay modelling dominate the statistical uncertainties.
    %the systematic uncertainties introduced by the model-dependent \ac{pwa} are now dominating the statistical uncertainties.
    An extension of the \ac{pwa} technique is thus needed to achieve a higher precision in the phenomenological decay description.


    The approach I propose and test here is the \ac{mipwa}, which exploits the increased size of the available experimental data to get rid of unjustified model assumptions.
    While not being a widespread analysis technique, the \ac{mipwa} has already been successfully used in the past---for example by the \focus{} collaboration to parametrize the S-wave component of a $\PDplus\to\PKminus\Ppiplus\Ppiplus$ decay~\cite{Link200914}.


    In this thesis, I will discuss my implementation of a \ac{mipwa}-fit utility.
    I will show the difficulties that come with a model-independent decay description---most notably, the large increase in the number of fit parameters and the appearance of fit ambiguities---and the results of several tests of my utility on \ac{mc} data.
    The aim of such a study is to provide a general utility for \ac{mipwa} fits to be used on real data collected at the \textsl{Belle} and \textsl{Belle II} experiments.


    My fit utility is based on \pac{yap}, a novel \cpp{} toolkit for \ac{pwa} developed at the \textsl{Technische Universit\"at M\"unchen} by Dr.~Greenwald and Mr.~Rauch, to which I have also contributed.
    I will briefly describe \pac{yap} in chapter~\ref{chap:yap}, after a short introduction to the \ac{pwa} formalism in heavy-meson decays in chapter~\ref{chap:theory}, and finally discuss my fit utility in chapter~\ref{chap:model_independent_pwa}.


%    \begin{itemize}
%    \item Resolving ambiguities (cite some paper) due to overlapping states and not well-known resonances;
%    \end{itemize}
%
%
%    In this thesis, I will describe \pac{yap}, .
%    {\color{red} why YAP? Implement PWA}
%
%    {\color{red} On top of \pac{yap}} 
%    
%
%
%
%
%    
%    
%
%    Il mio studio si inserisce all'interno di un progetto più generale del dipartimento di fisica, che sta cercando di formalizzare il \ac{mipwa} (zero waves).
%
%
%    \Ac{pwa} is a common tools to analyze data.
%
%
%    Also, through the Dalitz-plot analysis one can:
%    \begin{itemize}
%        \item Searches for new states;
%        \item Measuging resonance properties;
%        \item $CP$ violation;
%        \item B and D mixing;
%        \item Resolving ambiguities.
%    \end{itemize}
%
%    A toolkit for \ac{pwa} is \pac{yap}.
%
%    On top of \pac{yap} and \pac{bat} I have built a utility that implements a model-independent fit for analysis of $\PDplus\to\Ppiplus\Ppiminus\Ppiplus$ decays.
%
%    \paragraph{Why model-independent analysis}
%
%    Due to the availability of larger and larger data sets, the limiting factor in the data analysis is the quality of the model.
%
%
%    Model-independent analysis can be useful to solve ambiguity like:
%
%    ovelapping states
%
%    not well-known resonances.
%
%    \paragraph{goals of the thesis}
%
%    I have contributed to the development of \pac{yap}, a novel toolkit for \ac{pwa}, briefly described in the chapter~\ref{chap:yap}.
%    I have developed a fitting routine based on \pac{yap} that implements a model-independent fit.
%    I have tested the routine on a variety of \ac{mc} data with increasing complexity (see the chapter~\ref{chap:model_independent_pwa}).
%
%
%
%
%
%    The aim of this thesis is the description of \pac{yap}, a novel \ac{pwa} toolkit that I have contributed to develop.
%    I also show the results of some model-independent fits performed using a self-developed piece of software relying on the external libraries \pac{yap} and \pac{bat}.
%
%
